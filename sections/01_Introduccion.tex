\chapter{Introducción, Contexto y Motivación}\label{cap:introduccion}

%Este apartado tiene un tono un poco más personal,
%al fin y al cabo relata mi vivencia y opinión subjetiva sobre el TFG

\section{Introducción}

En palabras de Malcom X, \textit{"la educación es el pasaporte hacia el futuro, ya que el mañana pertenece a aquellos que se preparan para el hoy"}, y tanto es así que han sido cuatro años de preparación y dedicación día a día los que me han ido formando personal y académicamente hasta alcanzar la realización de este Trabajo de Fin de Grado, que abre las puertas de mi futuro profesional.

%%MODIFICACIÓN--------------------------------------
En esta primera sección se presenta el contexto y la motivación que trascienden a la realización del Trabajo.
%%MODIFICACIÓN-------------------------------------------


\section{Contexto // por el que se selecciona este topico //}



%Los tiempos que corren ahora son tiempos cambiantes, Industria 4.0, el auge de las IA, la importancia de la interoprerabilidad y los estándares... 

%- Propuestas a nivel europeo? OHDSI?


%- Hablar de OHDSI EN SEVILLA

%    [Innodata2023]
    
%- La selección de este tópico se debe al creciente interés por el estándar de OHDSI en Sevilla (Hospital Macarena y/o Hospital V del Rocio) y en España.

%- Comentar Aplicaciones reales y actuales del estandar. Además del interés a nivel mundial (Ohdsi en Europa Y en america del norte). Ohdsi community.
    


\section{Motivación}

%Mi motivación personal de entrar en el mundo del %análisis de datos clínicos utilizando  esta %herramienta prometedora..






    






