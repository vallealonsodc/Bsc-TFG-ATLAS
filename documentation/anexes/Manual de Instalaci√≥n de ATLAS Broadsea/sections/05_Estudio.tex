\chapter{Estudio Previo}\label{cap:estudio}

\section{Introduccion}

En esta sección se muestra un estudio comprensivo del estandar OHDSI utilizado: qué es, su ......

\section{Qué es OHDSI}

Observational Health Data Science and Informatics

....

\subsection{Historia}

Surgimiento a partir del proyecto OMOP 2008. Proyecto acaba en 2013. Se forma OHDSI para seguir trabajando en ello.

\subsection{Misión, Visión y Valores}

-
-
-

\subsection{Red colaborativa}

- Collaborative network across XX countries, organizations...
- Github....
- Community calls...
-Symposium

\section{Cómo generar evidencia}

- A journey from data to evidence (buscar en la documentación de ohdsi)

\subsection{Building blocks of OHDSI}

(buscar en la documentación de ohdsi)

\subsection{Estandarización de los datos}

- OMOP CDM

- OMOP VOCABULARY

\subsection{Investigación metodológica}
%%ADEMAS DE SER LOS TRES MÉTODOS QUE OFRECE OHDSI PARA REALZIAR INVESTIGACIONNES SON 3 CASOS DE USOS
-Caracterizacion
-Estimacion a nivel de poblacion
-Prediccion a nivel de paciente

\section{Herramientas}
\subsection{ATLAS}

- Extensa descripción de ATLAS (versión actual, anteriores, uso, aspecto...)

- Diferentes tipos de ATLAS (demo, Broadsea, AWS..)

ATLAS ADEMÁS IMPLEMENTA INTRÍNSICAMENTE  DOS HERRAMIENTAS

-ATHENA (herramienta de busqueda en el vocabulario del CDM) actualemnte está implementada dentro de ATLAS/Search

- ACHILLES (data quality dashboard) también esta implementada actualmente dentro de ATLAS/Data source


\subsection{Otras herramientas}

breve descripción de cada una:

-HADES (herramientas de análisis pero en librerias R)

-WHITE-RABBIT y RABBIT-IN-A-HAND (para preparar las ETL)

USAGI (también para la ETL)
...

\section{Conclusiones}