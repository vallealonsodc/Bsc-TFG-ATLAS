% Plantilla TFG LaTeX LSI por:
%   Agustín Borrego <borrego@us.es>
%   Inma Hernández <inmahernandez@us.es>
% Su uso y modificación es libre.

% ̀¡Recuerda hacer copias de seguridad frecuentes durante la redacción del trabajo!
% Puedes descargar todo el código fuente del proyecto en zip en Menú > (Descargar) Fuente

\documentclass[12pt]{report}

% Paquetes LaTeX y estilos globales
\input{etc/pkgs}
\input{etc/style}
\setlength{\parindent}{0pt}
\newcommand{\code}[1]{\texttt{\textsc{#1}}}


%%%%%%%%%%%%%%%%%%%%%%%%%%%%%%%%%%%%%%%%%%%%%%%%%%%%%%%%%%%%%%%%%%%%%%%%%%%%%%%%%%%%%

% Variables para la portada
\setTitle{Manual de instalación de ATLAS Broadsea}

%%%%%%%% Extrayendo evidencia utilizando la herramienta ATLAS a partir de datos estandarizados según OMOP CDM


\setAuthor{Da. Maria del Valle Alonso de Caso Ortiz} % Si hay más de un autor, separarlos con \\
\setDegree{Grado en Ingeniería de la Salud} % Cambiar si es necesario
\setSupervisor{Da. Silvia Rodríguez Mejías \\ Dr. Carlos Parra Calderón} % Si hay más de un tutor, separarlos con \\
\setDepartment{Innovación Tecnológica del HUVR}
\setMonth{junio} % Dejar sólo el mes de la convocatoria en que se presenta el trabajo
\setYear{2023/24} % Por ejemplo, 2022/23

%%%%%%%%%%%%%%%%%%%%%%%%%%%%%%%%%%%%%%%%%%%%%%%%%%%%%%%%%%%%%%%%%%%%%%%%%%%%%%%%%%%%%

% Dedicatoria del trabajo
% Si no se desea incluir, comentar o borrar la línea siguiente para eliminar la página de dedicatoria
%\setDedication{A mi padre y a mi madre, por inculcarme la pasión por el estudio y acompañarme incondicionalmente en cada etapa del camino.}

%%%%%%%%%%%%%%%%%%%%%%%%%%%%%%%%%%%%%%%%%%%%%%%%%%%%%%%%%%%%%%%%%%%%%%%%%%%%%%%%%%%%%

% Comienzo del documento
\begin{document}

    % Portada y secciones no numeradas
    \thispagestyle{empty} % Impide que se incluya número de página en la portada
\begin{center}

\vspace*{1cm}

\includegraphics[width=\textwidth]{figures/etsii_us.png}

\vspace*{2.5cm}
\begin{large}
ANEXO A
\end{large}

\vspace*{0.1in}
\textbf{\huge \tfgTitle}

\vspace*{.5in}

{\large Realizado por}\\
\textbf{\Large \tfgAuthors}

\vspace*{2cm}

%\textbf{Para la obtención del título de}\\
%{\large \tfgDegree}

\vspace*{0.2in}

\textbf{Dirigido por}\\
{\large \tfgSupervisor}\\

\vspace*{0.2in}

\textbf{En el departamento de}\\
{\large \tfgDepartment}

\vspace*{.6in}
%\textbf{\Large Convocatoria de \tfgMonth, curso \tfgYear}

\end{center}

% Dedicatoria
\ifdefined\tfgDedication
    \newpage
    \thispagestyle{empty}
    
    \vspace*{\fill}
    \begin{center}
    \textit{\tfgDedication}
    \end{center}
    \vspace*{\fill}
\fi

\clearpage\setcounter{page}{1} % Comienza a incluir números de página a partir de aquí
\pagenumbering{roman} % En números romanos
    %\chapter*{Agradecimientos}

A mi familia, a mi padre Francisco José Alonso de Caso, a mi madre María del Valle Ortiz y a mis cuatro hermanos: Manuel, Ignacio, Quico y Juan Pablo; por haber sido apoyo incondicional e inspiración de los valores del trabajo, esfuerzo y sacrificio durante mis años de estudio y durante toda mi vida.

A todos mis compañeros y compañeras de clase, a las que me han acompañado y ayudado en algún momento durante el transcurso del grado de Ingeniería de la Salud y especialmente a aquellas que considero mis amigas, a Patricia, Angela, Marta, Gracia, Andrea y Paloma  que no solo me han acompañado sino que también han amenizado el camino, llenándolo de diversión y pasión por nosotras y por estos estudios que hemos disfrutado juntas.

A todos los profesores con los que he coincidido, especialmente a Julián y María José, que además han tutelado y supervisado este Trabajo Fin de Grado.

A todos los profesionales del departamento de Innovación Tecnológica del Hospital Universitario Virgen del Rocío, que me han guiado durante el período de las prácticas curriculares, apostando por esta iniciativa y ayudándome a llevarla a cabo tutorizando y supervisando su desarrollo, especialmente a Silvia y Carlos.

    %\chapter*{Resumen}

Este trabajo consiste en el estudio y aplicación de la herramienta de análisis de datos ATLAS perteneciente a la organización \textit{Observational Health Data Sciences and Informatics} (OHDSI) con la finalidad de promover la estandarización de la investigación observacional con datos de salud. El proyecto se ha desarrollado en colaboración con el grupo de Informática de la Salud Computacional del Hospital Universitario Virgen del Rocío.

La necesidad de estandarización entre los sistemas informáticos y datos de salud es un aspecto de cada vez mayor relevancia a nivel mundial. En este aspecto, OHDSI se alza como una comunidad de investigadores con la finalidad común de unificar la forma de conducir estudios observacionales, a través del Modelo de Datos Común de OMOP y un complejo ecosistema de herramientas de procesamiento y análisis de datos, entre las que destaca ATLAS, una herramienta de análisis de datos \textit{low-code} que permite ejecutar análisis siguiendo una metodología común.

Por la relevancia de la organización, el proyecto consta de una primera parte en la que se recopila información teórica sobre OHDSI, sus estándares y herramientas. Posteriormente, la investigación se complementa con un caso práctico en el que se reproduce con ATLAS un estudio realizado por el grupo de investigadores del hospital sobre posibles efectos adversos del tratamiento radioterápico en pacientes de cáncer de pulmón.

El proyecto en su totalidad confirma la relevancia de OHDSI en el sector de la informática clínica y los beneficios de la aplicación de la herramienta ATLAS y el Modelo de Datos Común de OMOP en la estandarización de la investigación observacional.


\vspace{.5cm}

\textbf{Palabras clave:} Observational Health Data Sciences and Informatics, ATLAS, Modelo de Datos Común de OMOP, Broadsea.
    %\chapter*{Abstract}

**Translation:**

This Bachelor's Thesis presents the study and application of the data analysis tool ATLAS, belonging to the organization \textit{Observational Health Data Sciences and Informatics} (OHDSI). The project, which addresses the need for standardization of health information systems and data at the national and international levels, has been developed in collaboration with the Computational Health Informatics Group (GIS) of the Virgen del Rocío University Hospital (HUVR).

In this context, the OHDSI organization stands out as the most important community of researchers with the common goal of unifying the way observational studies are conducted, through the OMOP Common Data Model and a complex ecosystem of data processing and analysis tools, among which ATLAS, the central focus of this work, stands out.

The \textit{low-code} data analysis tool allows for the execution of various analyses depending on the specific use case following a common methodology. Finally, this thesis is completed with a practical case in which, through the deployment of the ATLAS tool via Broadsea, a study on possible adverse effects of radiotherapy treatment in lung cancer patients conducted by HUVR is standardized. The project as a whole confirms the relevance of the OHDSI organization in the clinical informatics sector and the benefits of applying the ATLAS tool and the OMOP Common Data Model in the standardization of health data and observational research.


\vspace{.5cm}

\textbf{Keywords:} ATLAS, Broadsea, Common Data Model, Observational Health Data Sciences and Informatics (OHDSI), Observational Medical Outcomes Partnership (OMOP), Oncology.
   
    
    
    % Índice del documento y de figuras
    \begingroup
        % Los enlaces son normalmente azules, pero en los índices se configuran a negro para que no aparezca todo azul
        \hypersetup{linkcolor=black}
        \tableofcontents
        %\listoffigures
        %\listoftables
        %\lstlistoflistings
    \endgroup
    
    % Cambia el estilo de números de página de romanos a normal
    \clearpage\pagenumbering{arabic}
    
    % Capítulos del trabajo
    %\chapter{Ejemplos de uso de LaTeX}\label{cap:ejemplos}

\todo[inline]{Este capítulo se incluye únicamente como ayuda y referencia de uso de \LaTeX. No debe aparecer en el documento final.}

\section{Introducción}
En este capítulo se muestran ejemplos de uso de \LaTeX{} para operaciones comunes. 

\section{Estilos}\label{sec:estilos}
Se pueden aplicar estilos al texto como \textbf{negritas}, \textit{cursiva}, \underline{subrayado} y \texttt{monoespaciado}. También se \textcolor{red}{pueden} \textcolor{blue}{aplicar} \textcolor{green}{colores}, y \underline{\textit{combinar}} \textbf{\textcolor{red}{estilos}}. Se recomienda usar sólo negritas para hacer énfasis, y no abusar de este recurso.

Por comodidad para usuarios no habituados con LaTeX, esta plantilla define algunos alias de comandos más fáciles de recordar para estilos de texto comunes: \negritas{negritas}, \cursiva{cursiva} y \codigo{código}.

\section{Listados}
Con itemize se pueden crear listas no numeradas:

\begin{itemize}
    \item Fresas
    \item Melocotones
    \item Piñas
    \item Nectarinas
\end{itemize}

De manera similar, enumerate permite crear listas numeradas:

\begin{enumerate}
    \item Elaborar la memoria del TFG
    \item Elaborar la presentación
    \item Presentar el TFG
    \item Solicitar el título de Grado
\end{enumerate}

\section{Subsecciones}
Se pueden definir subsecciones con el comando \texttt{subsection}:

\subsection{Primera subsección}\label{sec:subseccion}
Esto es una subsección

\subsection{Segunda subsección}
Esto es otra subsección.

\subsubsection{Sub-sub-sección}
Este es un tercer nivel de profundidad, que no aparece en el índice general. Se recomienda no utilizarlo, si es posible.

\section{Imágenes y figuras}
Todas las imágenes y figuras del documento se incluirán en la carpeta ``figures''. Se pueden incluir de la siguiente manera:

\begin{figure}[htp]
    \centering
    \includegraphics[width=0.7\textwidth]{figures/ejemplo.png}
    \caption{Un feroz depredador}
    \label{fig:ejemplo}
\end{figure}

Observe que las figuras se numeran automáticamente según el capítulo y el número de figuras que hayan aparecido anteriormente en dicho capítulo. Existen muchas maneras de definir el tamaño de una figura, pero se aconseja utilizar la mostrada en este ejemplo: se define el ancho de la figura como un porcentaje del ancho total de la página, y la altura se escala automáticamente. De esta manera, el ancho máximo de una figura sería 1.0 * textwidth, lo que aseguraría que se muestra al máximo tamaño posible sin sobrepasar los márgenes del documento.

Tenga en cuenta que LaTeX intenta incluir las figuras en el mismo sitio donde se declaran, pero en ocasiones no es posible por motivos de espacio. En esos casos, LaTeX colocará la figura lo más cerca posible de su declaración, puede que en una página diferente. Esto es un comportamiento normal.

\section{Tablas}
Existe una gran variedad de formas de crear tablas en LaTeX puro, y todas ellas tienen cierta complejidad. A continuación se muestra un ejemplo simple de tabla nativa, en la Tabla \ref{table:ejemplo}. Se recomienda crear un archivo en la carpeta \textit{tables} por cada tabla nativa que se desee incluir, y enlazarla mediante el comando \texttt{input}.

\begin{table}[htp]
\centering

    % Esta primera línea define las columnas de la tabla. Los posibles tipos de columna son:
    % c: texto centrado
    % l: texto alineado a la izquierda
    % r: texto centrado a la derecha
    % p: columna de ancho fijo
    % Las columnas tienen ancho dinámico según la anchura máxima de los elementos que contengan.

    % Las columnas l/r/c no parten el texto en filas diferentes si éste es demasiado largo. Para ello, puede utilizar el tipo de columna de ancho fijo "p".
    
    % Las barras verticales | se usan para definir los bordes verticales de la tabla. Pruebe a eliminar algunas y observe qué ocurre.
    \begin{tabular}{ | l | c | r | p{2cm} | }
        
        % A continuación van las filas de la tabla. En cada fila, las columnas se separan con el carácter &
        % Para terminar una fila se usa \\
        % Para incluir un borde horizontal entre filas se usa \hline

        % Cabecera con textos en negrita:
        \hline
        \textbf{Columna L} & \textbf{Columna C} & \textbf{Columna R} & \textbf{Columna P}\\
        \hline
        
        % Cuerpo de la tabla:
        Texto de ejemplo & Texto de ejemplo & Texto de ejemplo & Texto de ejemplo\\
        \hline
        ABC & DEF & HIJ & KLM\\
        \hline
        
    \end{tabular} 
    
    \caption{Tabla LaTeX de ejemplo}
    \label{table:ejemplo} 
\end{table}


Para tablas con un formato más complejo, considere la posibilidad de diseñarla usando otro software externo (por ejemplo Excel) e incluirla de manera similar a una figura. \textbf{Observe en el código LaTeX a continuación cómo usar el comando \texttt{captionof\{table\}}, en lugar de simplemente \texttt{caption}, hace que se liste como una Tabla en lugar de como una Figura}:

\begin{figure}[htp]
    \centering
    \includegraphics[width=1.0\textwidth]{tables/complex_table.png}
    \captionof{table}{Tabla compleja introducida como figura}
    \label{table:ejemplo2}
\end{figure}

\section{Referencias}
Observe cómo en el código fuente de esta sección se ha usado varias veces el comando \texttt{label}. Este comando permite marcar un elemento, ya sea capítulo, sección, figura, etc. para hacer una referencia numérica al mismo. Para referenciar una label se usa el comando \texttt{ref} incluyendo el nombre de la referencia:

Este es el capítulo \ref{cap:ejemplos}.

En la sección \ref{sec:estilos} se muestran ejemplos de estilos.

La subsección \ref{sec:subseccion} explica...

En la Figura \ref{fig:ejemplo} vemos que...

Esto evita que tengamos que escribir directamente los índices de las secciones y figuras que queremos mencionar, ya que LaTeX lo hace por nosotros y además se encarga de mantenerlos actualizados en caso de que cambien (pruebe a mover este capítulo al final del documento y observe cómo se actualizan automáticamente todos los índices referenciados). Además, las referencias mediante ``ref'' actúan como hipervínculos dentro del documento que llevan al elemento referenciado al pulsar en ellas.

Es habitual nombrar las ``label'' con un prefijo que indica el tipo de elemento para encontrarlo luego más fácilmente, pero no es obligatorio.

\section{Extractos de código}

Se pueden incluir extractos de código mediante lstlisting:

\begin{lstlisting}[language=Python, caption={Código Python}, label={cod:python}, captionpos=b]
num = float(input("Enter a number: "))
if num > 0:
   print("Positive number")
elif num == 0:
   print("Zero")
else:
   print("Negative number")
\end{lstlisting}

Para evitar tener que incluir el código directamente en el texto del documento, se pueden guardar en archivos separados y referenciarlos:

\lstinputlisting[
    float,
    floatplacement=!htp,
    language=Java,
    label=cod:java,
    caption=Código Java
]{code/java_example.java}

\lstinputlisting[
    float,
    floatplacement=!htp,
    language=html,
    label=cod:html,
    caption=Código HTML
]{code/html_example.html}

\lstinputlisting[
    float,
    floatplacement=!htp,
    language=javascript,
    label=cod:js,
    caption=Código JavaScript
]{code/javascript_example.js}

Los extractos de código también se pueden referenciar mediante label/ref: Extractos de código \ref{cod:python}, \ref{cod:java}, \ref{cod:html}, \ref{cod:js}. 

\section{Enlaces}
Puede enlazar una web externa mediante el comando \texttt{url}: \url{https://www.example.com}. También se puede vincular un enlace a un texto mediante el comando href: \href{https://www.example.com}{dominio de ejemplo}.

\section{Citas y bibliografía}
En LaTeX, los elementos de la bibliografía se almacenan en un fichero bibliográfico en un formato llamado BibTeX, en el caso de este proyecto se encuentran en ``bibliografia.bib''. Para citar un elemento se usa el comando \texttt{cite}. Se pueden citar tanto artículos científicos \cite{borrego2019} como enlaces web \cite{webETSII}. 

También se puede usar el comando \texttt{citet} para incluir una referencia junto con el nombre de su autor o autores: \citet{borrego2021}. Todas las citas se numeran automáticamente y se incluyen en la sección de bibliografía del trabajo. El orden por defecto es según su orden de aparición en el documento. Para ordenarlas por orden alfabético del autor, puede modificar el comando \texttt{bibliographystyle} del archivo principal y reemplazar su valor por el estilo \texttt{plainnat} (orden alfabético, nombres completos) o \texttt{abbrvnat} (orden alfabético, nombres abreviados).

Observe cómo los elementos bibliográficos almacenados en ``bibliografia.bib'' tienen una etiqueta asociada, que es la que se usa al citarlos mediante cite. \textbf{Añadir una referencia al fichero bibliográfico no hace que ésta aparezca automáticamente en la sección de bibliografía del trabajo, es necesario citarla en algún lugar del mismo}.

\section{Ecuaciones}
LaTeX tiene un potente motor para mostrar ecuaciones matemáticas y un amplio catálogo de símbolos matemáticos. El entorno matemático se puede activar de muchas maneras. Para incluir ecuaciones simples en un texto se pueden rodear de símbolos dólar: $1 + 2 = 3$, $\sqrt{81} = 3^2 = 9$, $\forall x \in y~\exists~z : S_z < 4$.

Las ecuaciones más complejas pueden expresarse aparte y son numeradas: ecuación \ref{eq:ecuacion}.

\begin{equation}\label{eq:ecuacion}
\lim_{x\to 0}{\frac{e^x-1}{2x}}
 \overset{\left[\frac{0}{0}\right]}{\underset{\mathrm{H}}{=}}
 \lim_{x\to 0}{\frac{e^x}{2}}={\frac{1}{2}}
 +7 \int_0^2
  \left(
    -\frac{1}{4}\left(e^{-4t_1}+e^{4t_1-8}\right)
  \right)\,dt_1
\end{equation}

Dispone \href{http://www.yann-ollivier.org/latex/texsymbols.pdf}{aquí} de un amplio listado de símbolos que pueden usarse en modo matemático.

\section{Caracteres y símbolos especiales}
Algunos caracteres y símbolos deben ser escapados para poder representarse en el documento, ya que tienen un significado especial en LaTeX. Algunos de ellos son:

\begin{itemize}
    \item El símbolo dólar \$ se usa para ecuaciones.
    \item El tanto por ciento \% se usa para comentarios en el código fuente.
    \item El símbolo euro \euro{} suele dar problemas si se escribe directamente.
    \item El guión bajo \_ se usa para subíndices en modo matemático.
    \item Las comillas deben expresarse `así' para comillas simples y ``así'' para comillas dobles. Las comillas españolas pueden expresarse \textquote{así}.
    \item La barra invertida o contrabarra \textbackslash{} se usa para comandos LaTeX.
    \item Otros símbolos que deben escaparse son las llaves \{ \}, el ampersand \&, la almohadilla \# y los símbolos mayor que \textgreater{} y menor que \textless{}.
\end{itemize}
    \chapter{Introducción, Contexto y Motivación}\label{cap:introduccion}


\section{Introducción}

%Este Trabajo Fin de Grado está orientado a {}, trabaja con {}, pretende {} .....

En esta primera sección de Introducción se presenta el contexto y la motivación que trascienden a la realización del Trabajo Fin de Grado (TFG).
%%-------------------------------------------------------

\section{Contexto}
 
El contexto en el que se desarrolla este TFG, que trata el análisis de datos en el ámbito clínico, se encuentra altamente influenciado por el auge actual de la Industria 4.0 y su potente impacto en el sector sanitario.

%%1. INDUSTRIA 4.0

\subsubsection{Contexto general: Industria 4.0, aparición de nuevas tecnologías}

La Industria 4.0, o cuarta revolución industrial, es un concepto concebido por el gobierno alemán en noviembre de 2011 como una estrategia tecnológica para abordar el crecimiento industrial proyectado para 2020. Su uso internacional se popularizó en abril de 2013 durante la feria industrial de Hannover \textit{Hannover Messe}). Este concepto representa la cuarta fase de la industrialización, sucediendo a la mecanización, electrificación e informatización, y destaca la integración de tecnologías avanzadas \cite{lasi2014industry}..
Se centra principalmente en la digitalización y la necesaria convergencia entre los sistemas físicos y cibernéticos, en inglés \textit{Cyber-Physical Systems (CPS)}. Esta integración se busca a través de nuevas tecnologías de la información y teleocomunicación (TICs), como el internet de las cosas (\textit{Internet of Things o IoT}), la generación y análisis de de datos masivos (\textit{Big Data \& Big Data Analytics}), la computación en la nube (\textit{Cloud Computing}) y el auge de la Inteligencia Artificial (IA) \cite{lasi2014industry}.\cite{chen2020times}\cite{tortorella2020healthcare}


%%2. HEALTHCARE 4.0

\subsubsection{Contexto general aplicado: Sanidad 4.0, características}

La integración de los principios y tecnologías de la Industria 4.0 en el sector sanitario originó el concepto de Salud o Sanidad 4.0 (del inglés, \textit{Healthcare 4.0})\cite{tortorella2020healthcare}\cite{tortorella2021impacts}.  %
En este contexto, este nuevo término se presenta como un complejo desafío  destinado a abordar los nuevos escenarios generados por la creciente demanda de dispositivos y sistemas médicos más eficaces y alineados con las nuevas TICs y los avances ininterrumpidos en las ciencias como la biotecnología y la ingeniería genética. \cite{martin2021ehealth}. La Sanidad 4.0 origina un nuevo ecosistema interseccional del que destacan a lo largo del TFG  tres  características principales: (1) la provisión continua de cuidado sanitario, (2) la orientación de la medicina hacia el paciente y (3) la prevención y predicción de enfermedades.

%Provisión continua de cuidado sanitario - Tecnologías de I4.0 + HCE

(1) La provisión continua del cuidado sanitario se basa en el continuo de salud (\textit{continuum of care}) \cite{kouroubali2019new}. Gracias a las nuevas tecnologías de la Industria 4.0, mayoritariamente a las TICs y al IoT, la sociedad está estrechamente comunicada entre sí de forma prácticamente ininterrumpida. También a raíz de la pandemia del COVID-19 se han acelerado todas estas telecomunicaciones, que en el ámbito sanitario han potenciado el desarrollo de la telemedicina y la salud digital (o \textit{e-Health} \cite{martin2021ehealth}. Con la digitalización y el seguimiento remoto de la salud, los dispositivos médicos que monitorizan a los pacientes en su vida cotidiana generan enormes cantidades de datos médicos de distintas índoles que, además, se recogen con distintos propósitos según cada organización. Los sistemas de salud digital frecuentemente almacenan datos inconsistentes, incoherentes o inaccesibles entre sí, produciéndose registros electrónicos de salud muy extensos y dispares \cite{kouroubali2019new}. 

%Patient-centred - Modelos de datos más amplios. Medicina de precisión.

(2) La orientación de la medicina hacia el paciente se refiere a la priorización del paciente como objetivo central de la provisión de salud  \cite{tortorella2020healthcare}. La atención sanitaria cada vez es más específica a cada individuo, gracias al seguimiento remoto de su actividad diaria y al auge de la medicina de precisión, que es una nueva disciplina médica que considera que el estudio clínico de un paciente debe alcanzar niveles tan detallados como el estudio de su genoma, proteoma, condiciones medioambientales, rutina \cite{ruiz2023inteligencia}... La posición del foco de la salud en el paciente, fomentado por la Unión Europea,  implica reestructurar el sistema sanitario alrededor del mismo, pues debe ser el paciente el cliente final, juez y recibidor de todos los servicios y aplicaciones de la salud digital \cite{ntafi2022legal} \cite{katehakis2019framework}. En términos informáticos esto conlleva reestructurar los sistemas médicos de modo que se recoja de manera central para cada individuo su historial clínico electrónico (HCE) completo, que incluya tanto datos médicos, como farmaceúticos y otros datos de interés. 


%Preventiva y predictiva - Herramientas de big data, IA

(3) La última característica es que sea preventiva y predictiva en vez de reactiva. Esto quiere que decir, que a diferencia de cómo se ha estado realizando tradicionalmente, el enfoque de la medicina debe transicionar, de ser una medicina meramente curativa posterior a la aparición de una enfermedad, a proveer salud previamente a la aparición de una enfermedad de manera que esta enfermedad sea predicha, a través del análisis del HCE del paciente y/o exhaustivos análisis de precisión, y prevenida a través de monitorearización y provisión de tratamientos preventivos \cite{ruiz2023inteligencia}. El análisis del historial clínico de un paciente genera un desafío muy complejo por las características propias de los datos que se han comentado previamente, es decir, por su complejidad, desorden y extensión, de modo que las técnicas de análisis de datos tradicionales generalmente resultan insuficientes. La prevención y la predicción se alcanza gracias al constante desarrollo de técnicas y algoritmos cada vez más sofisticos de inteligencia artificial y aprendizaje automático y herramientas cada vez más poderosas de ciencia y análisis de datos masivos.


\subsubsection{Pilares fundamentales: Interoperabilidad y estandarización a nivel internacional y europeo}

Estas tres características de la Sanidad 4.0 se mantienen firmes sobre un principio fundamental de creciente interés internacional: (a) la estandarización y (b) interoperabilidad de los sistemas médicos. Ambos conceptos están relacionados entre sí mediante una relación causa-consecuencia, según el Institute of Electrical and Electronics Engineers (IEEE, 2013), "la interoperabilidad se hace posible mediante la implementación de estándares" \cite{berryman2013data}.

(a) La implementación de estándares o estandarización consiste principalmente en establecer acuerdos entre las grandes organizaciones de la salud para definir marcos específicos a través de los que estructurar los registros clínicos electrónicos de manera única, reduciendo el desorden y la disparidad de los datos y permitiendo el intercambio de mensajes entre sistemas pertenecientes a distintas organizaciones. La estandarización es un requisito fundamental para alcanzar la interoperabilidad \cite{katehakis2019framework}. Actualmente existen muchos estándares reconocidos internacionalmente, tales como HL7 (Health Level Seven), DICOM (Digital Imaging and Communications in Medicine), SNOMED CT (Systematized Nomenclature of Medicine - Clinical Terms) o IHE (Integrating the Healthcare Enterprise). Con los estándares nace también un concepto importante: el código abierto o \textit{Open Source}. Sin ir más lejos, HL7, la mayor de las organizaciones anteriores comenzó ofreciendo sus servicios de infraestructura de datos y mensajerías de manera privada hasta 2012 cuando se decidió a promover el código abierto liberando la mayor parte de su propiedad intelectual para que pudiera ser accesible de forma gratuita, lo que potenció la adopción de estándares y la interoperabilidad entre las organizaciones sanitarias \cite{berryman2013data}.

(b) Gracias a la popularización de los estándares médicos se está desarrollando cada vez más y mejor la interoperabilidad entre los diferentes sistemas, siendo este es el objetivo final de la revolución industrial, tecnológica y sanitaria actual. A principios de siglo la Comisión Europea identificó la necesidad de interoperabilidad entre las administraciones públicas y se comenzó a desarrollar programas para respaldarla y promoverla \cite{CEU1999ida}. En 2010 se adoptó el primer Marco Europeo de Interoperabilidad (\textit{European Interoperability Framework, EIF}) que junto a los programas Soluciones de interoperabilidad para las administraciones públicas europeas (ISA y ISA\textsuperscript{2}) han asentado las bases para las estrategias actuales. En 2013 El IEEE definió el concepto de interoperabilidad en como "la habilidad de los sistemas de intercambiar información y utilizar dicha información intercambiada de forma efectiva" \cite{berryman2013data} y en 2017 la Unión Europea adoptó el nuevo Marco de Interoperabilidad Europea \textit{(new EIF)}  a través del cual ofrecer recomendaciones, modelos y guianza a fin de mejorar la calidad de los servicios públicos europeos alegando que "la falta de interoperabilidad es el mayor obstáculo para progresar" \cite{kouroubali2019new}. También, en la Comisión Europea del mismo año, se actualizó la definición de interoperabilidad como "la habilidad de las organizaciones de interactuar hacia objetivos mutamente beneficiosos, involucrando el intercambio de información y conocimiento entre dichas organizaciones a través de los procesos empresariales que soportan, es decir, del intercambio de información entre sus sistemas de información TIC" \cite{katehakis2019framework}\cite{CEU2017eif} \cite{casiano2022towards}.

\subsubsection{Iniciativas interoperables a nivel europeo: El camino hacia OHDSI}

En el camino hacia la interoperabilidad en el ámbito sanitario, en noviembre de 2018 se lanzó la Red Europea de Datos y Evidencia en Salud (\textit{European Health Data \& Evidence Network, EHDEN}) con el objetivo de \"abordar los desafíos actuales en la generación de conocimientos y evidencia a partir de datos clínicos del mundo real a escala, para ayudar a los pacientes, médicos, pagadores, reguladores, gobiernos y la industria a comprender el bienestar, las enfermedades, los tratamientos, los resultados y Nuevas terapias y dispositivos". \cite{ehden}. En marzo de 2020 EHDEN comenzó a colaborar con la organización OHDSI (\textit{Observational Health Data Sciences and Informatics}) \cite{ohdsi}, de gran popularidad en el continente americano, para realizar estudios sobre el COVID-19 hasta la actualidad. Apartir de entonces la relación entre ambas entidades se ha estrechado enormente, adquiriendo OHDSI mayor popularidad europea y EHDEN numerosos beneficios por parte de su colaboración, de tal manera que el Catálogo de socios de datos del Portal EHDEN debutó en el Simposio Europeo de OHDSI en junio de 2022. 

Actualmente la presencia de OHDSI en europa es de un interés cada vez mayor y, por ende, también está en auge a nivel estatal \cite{ohdsiSpain}. España es uno de los nodos de colaboración con OHDSI más grandes de europa, y muchas organizaciones a lo largo del territorio español ya están colaborando con sus estándares como la Agencia Española de Medicamentos y Productos Sanitarios (AEMPS) o Quirónsalud entre otros. En Sevilla, la colaboración con OHDSI la llevan a cabo los hospitales universitarios Virgen Macarena y Virgen del Rocío, siendo este último sede del estudio práctico que ha acompañado a este TFG.


%- Hablar de OHDSI EN SEVILLA

%  [Innodata2023]
    
%- La selección de este tópico se debe al creciente interés por el estándar de OHDSI en Sevilla (Hospital Macarena y/o Hospital V del Rocio) y en España.


\section{Motivación}

%Mi motivación personal de entrar en el mundo del %análisis de datos clínicos utilizando  esta %herramienta prometedora..

%RECUERDOS PRINCIPIO DE CARRERA BIG DATA, POLITECNICO DI MILANO...






    







    \chapter{Despliegue por defecto.} \label{cap:02Despliegue}

El despliegue de ATLAS en Docker es muy sencillo y está muy bien documentado en el repositorio de github de Broadsea \cite{githubBroadsea}.  Igualmente, en este manual se detallan nuevamente los pasos para la configuración y despliegue de la herramienta, añadiendo algunos pasos relevantes adicionales.

\section{Requisitos para el despliegue}

\begin{enumerate}
    \item Descargar e instalar Docker. Lo más sencillo es seguir las instrucciones de la \href{https://docs.docker.com/engine/install/}{página web oficial} para la descarga y seguir la configuración por defecto para la instalación.
    
    \item Descargar e instalar Git. Lo más sencillo es seguir las instrucciones de la \href{https://git-scm.com/downloads}{página web oficial} para la descarga y seguir la configuración por defecto para la instalación.
\end{enumerate}

\section{Deployment} \label{cap:02Deployment}

En este primer despliegue rápido de ATLAS, se desplegarán las configuraciones por defecto de la herramienta, siguiendo la guía de implementación rápida \textit{(Quick Start)} del repositorio de github.

\begin{enumerate}
    \item Por tanto, el primer paso para desplegar ATLAS es clonar localmente el repositorio de github de Broadsea. Una forma rápida de hacerlo es copiar la siguiente línea de código en el \code{cdm}.

\begin{lstlisting}[language=sh]
        git clone https://github.com/OHDSI/Broadsea.git
\end{lstlisting}

    \item El segundo paso, es desplegar el contenedor docker. Para ello, situar el puntero del \code{cdm}, en la carpeta donde se ha copiado el repositorio de github de Broadsea.

\begin{lstlisting}[language=sh]
        cd ruta\del\repositorio\Broadsea\local
\end{lstlisting}

    Una vez situado en la carpeta raíz del repositorio, se jecuta el comando que instalará y desplegará el perfil por defecto del contenedor docker en la máquina local.

\begin{lstlisting}[language=sh]
    docker compose pull && docker-compose --profile default up -d
\end{lstlisting}

\end{enumerate}

%--------------

\section{Comprobación de despliegue correcto} 

Se puede comprobar que se ha instalado correctamente el contenedor de Broadsea en la máquina local de distintas formas, tal y como se presenta a continuación.

\begin{enumerate} 

    \item La forma más sencilla de interactuar con el contenedor de Broadsea es a través de Docker Desktop. Ejecutando dicho programa, en la sección \textit{''containers"} se muestran todos los contenedores que están corriendo en el equipo. En este caso, debe aparecer un multi-contenedor llamado "broadsea" que contenga seis contenedores, tal y como se muestra en la Figura \ref{fig:dockerDesktop}.
    
\begin{figure}[H]
    \centering
    \includegraphics[width=0.90\textwidth]{figures/dockerDesktop.png}
    \caption{Captura de pantalla del contenedor Broadsea en Docker Desktop}
    \label{fig:dockerDesktop}
\end{figure}

    Mediante el panel de control de Docker se puede iniciar, pausar o detener cada contenedor (o todos a la vez) fácilmente y en cualquier momento. Por esto se dice que Broadsea ofrece servicios \textit{a-la-carte}.

    \item Otra forma de interactuar con el contenedor Docker es a través del \code{cmd}, ejecutando el comando \code{docker ps}, que muestra un listado de todos los contenedores que está ejecutando la máquina local. Con esta estrategia deberían mostrarse igualmente los mismos seis contenedores pertenecientes a broadsea, tal y como se muestra en la Figura \ref{fig:dockerCMD}

\begin{figure}[H]
    \centering
    \includegraphics[width=0.90\textwidth]{figures/dockerCMD.png}
    \caption{Captura de pantalla del comando \code{docker ps } en el \code{cmd}}
    \label{fig:dockerCMD}
\end{figure}
    
    \item Por último, para acceder a los servicios de Broadsea hay que abrir en el navegador web (Chrome recomendado) el servidor en el que se alojan los servicios. Por defecto, Broadsea se aloja en el servidor 127.0.0.1, introducirlo en el navegador para explorar las herramientas del contenedor, tal y como se muestra en la Figura \ref{fig:broadseaCap}.

    \begin{figure}[H]
    \centering
    \includegraphics[width=0.90\textwidth]{figures/broadseaCap.png}
     \caption{Captura de pantalla del servidor Broadsea ejecutado en Chrome}
    \label{fig:broadseaCap}
\end{figure}
    
    Se puede comprobar o modificar el servidor exacto dónde se aloja el contenedor revisando el parámetro \code{BROADSEA\_HOST} de la sección 1 del archivo \code{.env} de la ruta local del repositorio. 

\begin{lstlisting}[language=sh]
###################################################
# Section 1:
# Broadsea Host
###################################################
DOCKER_ARCH="linux/amd64" # change this to linux/arm64 if using Mac Silicon, otherwise keep as-is
BROADSEA_HOST="127.0.0.1" # change to your host URL (without the http part)
HTTP_TYPE="http" # if using https, you need to add the crt and key files to the ./certs folder
BROADSEA_CERTS_FOLDER="./certs" 
\end{lstlisting}

    En este caso, no se modifica el servidor durante la configuración de Broadsea, por lo que de aquí en adelante la dirección del contenedor se aloja en el servidor \code{127.0.0.1}, tal y como se muestra en la Figura \ref{fig:broadseaCap}
   



    Es interesante notar que Broadsea permite el acceso interactivo a la herramienta Atlas, que es la que nos interesa en este caso, pero también a Ares y a Hades, otras dos herramientas muy relacionadas.

    La ejecución de ATLAS en Broadsea es similar a ATLAS demo, aunque con algunas diferencias. En primer lugar, Broadsea solo ejecuta, por defecto, una base de datos, que es la base de datos de Eunomia.

\begin{figure}[H]
    \centering
    \includegraphics[width=0.90\textwidth]{figures/atlasBroadseaDB.png}
     \caption{Captura de pantalla de base de datos que utiliza ATLAS Broadsea}
    \label{fig:atlasBroadseaDB}
\end{figure}

    Por otra parte, y en contraste con la versión demo, ya no aparecen las entradas y estructuras que generan otros usuarios. La herramienta se presenta vacía, para ser completada solo con la información que nosotros introduzcamos.

\begin{figure}[H]
    \centering
    \includegraphics[width=0.90\textwidth]{figures/atlasBroadseaDB.png}
     \caption{Captura de pantalla señalando el número de entradas de definición de cohorte que almacena ATLAS Broadsea}
    \label{fig:atlasBroadseaDB}
\end{figure}
    
\end{enumerate}

\section{Solución de posibles errores}

\subsubsection{Error: Application initialization failed. Unable to connect to an instance of the WebAPI. Please contact your administrator to resolve this issue.}

El error aparece al dirigirse al servidor de Broadsea en el navegador.

\begin{figure}[H]
    \centering
    \includegraphics[width=0.90\textwidth]{figures/Error02AppFailed.png}
     \caption{Captura de pantalla del error}
    \label{fig:Error02AppFailed}
\end{figure}

Solución: Comprobar que todos los contenedores de docker implicados están corriendo. Recargar varias veces la página.



    \chapter{Conexión local con la BD por defecto}

En ocasiones, puede resultar interesante acceder a la base de datos remota de Broadsea desde un administrador de bases de datos local.  Docker almacena las bases de datos que utilizan los contenedores en lo que se denominan \textit{''volúmenes''}. Para revisar los volúmenes que están ejecutándose en el equipo se presentan dos estrategias:

\section{Gestión de BD docker}

\begin{enumerate}

    \item La forma más sencilla de interactuar con los volúmenes es a través de Docker Desktop, en la sección \textit{''volumes''}. En esta sección se muestran todos los volúmenes que está utilizando el equipo. En este caso, deben aparecer tres volúmenes, entre ellos atlasdb-postgres-data es el volumen de interés, en el que se almacena la información de la base de datos de ATLAS.

\begin{figure}[H]
    \centering
    \includegraphics[width=0.90\textwidth]{figures/dockerVolumes.png}
     \caption{Captura de pantalla del panel ''volumes'' de Docker Desktop}
    \label{fig:dockerVolumes}
\end{figure}

    \item Otra forma de comprobar el estado de los volúmenes es a través de la terminal del sistema, ejecutando el comando \textit{''docker volumes ls''}, que devuelve un listado de los volúmenes que se están ejecutando.
    
\begin{figure}[H]
    \centering
    \includegraphics[width=0.90\textwidth]{figures/dockerVolumesCDM.png}
     \caption{Captura de pantalla del comando ''docker volume ls'' en la terminal}
    \label{fig:dockerVolumesCDM}
\end{figure}
    
\end{enumerate}

Puede ser importante conocer la información de los volúmenes que ejecuta Docker para identificar comportamientos inusuales y acceder a configuración e información avanzada sobre el virtualizador. 

Por otro lado, la información específica sobre la configuración de los contenedores que utiliza Docker para conformar Broadsea se encuentra en el archivo \textit{''docker-compose.yml''}, alojado en la carpeta local del repositorio de github de Broadsea. Buscando el nombre del contenedor que alberga la base de datos de ATLAS se puede acceder a toda la información relevante del mismo, tal como se muestra en la Figura \ref{fig:dockerComposeDB}. Esta información será necesaria posteriormente para realizar la conexión con la base de datos.

\begin{figure}[H]
    \centering
    \includegraphics[width=0.90\textwidth]{figures/dockerComposeDB.png}
     \caption{Captura de pantalla de la configuración del archivo \textit{docker-compose.yml}}
    \label{fig:dockerComposeDB}
\end{figure}

\section{Requisitos para establecer la conexión}

\begin{enumerate}

    \item Descargar e instalar la base de datos PostgreSQL. Lo más sencillo es seguir las instrucciones de la \href{https://www.postgresql.org/download/}{página web oficial} para la descarga y seguir la configuración por defecto para la instalación.

    \item Descargar e instalar una administrador de base de datos PostgreSQL. Se recomienda utilizar el administrador pgAdmin. Para ello, lo más sencillo es seguir las instrucciones de la \href{https://www.pgadmin.org/download/}{página web oficial} para la descarga y seguir la configuración por defecto para la instalación.
    
\end{enumerate}


\section{Deployment}

Para establecer la conexión con la base de datos que utiliza Broadsea, se debe seguir las siguientes instrucciones:

\begin{enumerate}

    \item En primer lugar, se debe comprobar los parámetros de configuración de la base de datos. Para ello, se han detallado varias estrategias a lo largo del manual, siendo la más recomendada para esta ocasión revisar el \textit{docker-compose.yml} (Figura \ref{fig:dockerComposeDB}). Este archivo alberga la información técnica de los contendores que ejecuta Docker. En este caso, el contenedor que interesa es \textit{''broadsea-atlasdb''}.

    En el archivo por defecto, se presenta la contraseña para acceder a la base de datos \textit{(password=mypass)} y el puerto que utiliza (\textit{port=5432}).

    \item La configuración por defecto de Broadsea se solapa con la configuración local por defecto de PostgreSQL porque ambos alojan sus bases de datos en el servidor local y en el puerto 5432. Por tanto, para evitar este solapamiento se debe detener el servicio local de PostgreSQL, de forma que el puerto quede libre para albergar la base de datos de Broadsea.

    Para detener el servicio local de PostgreSQL lo más sencillo es abrir la aplicación \textit{servicios} buscar el servidor de postgre y deterlo, tal y como se muestra en la Figura \ref{fig:serviciosConfig}. Así nos aseguramos de liberar el puerto para que pueda ser ocupado por la base de datos de Broadsea.

    \begin{figure}[H]
    \centering
    \includegraphics[width=0.90\textwidth]{figures/serviciosConfig.png}
     \caption{Captura de pantalla de la aplicación de servicios.}
    \label{fig:serviciosConfig}
    \end{figure}

    \item El último paso consiste en registrar el servidor a través del administrador de base de datos instalado en el equipo, en este caso pgAdmin. 
    
    Una vez que tengamos libre el puerto 5432, podemos registrar un nuevo servidor en dicho puerto que acceda a la base de datos de Broadsea. Para ello abrimos el administrador de la base de datos PostgreSQL, pgAdmin 4, y registramos un nuevo servidor. Los parámetros de configuración de este nuevo servidor se describen en el \textit{docker-compose.yml} y en la sección 3 del archivo \textit{.env}. Los parámetros fundamentales son:

    \begin{lstlisting}[language=sh]
        host = 127.0.0.1
        port = 5432
        user = postgres
        password = mypass
    \end{lstlisting}
    
    \item Tras registrar el servidor correctamente, debe aparecer una base de datos con cinco esquemas: ''demo\_cdm'', ''demo\_cdm\_results'', ''public'', ''webapi'', ''webapi\_security''. Para comprobar que se ha establecido una conexión correcta con la base de datos, sin pérdida de información, se puede comprobar el número de filas que recupera pgAdmin de la tabla ''person'' del esquema ''demo\_cdm'', tal y como se muestra en la Figura \ref{fig:pgAdmin}.

    \begin{figure}[H]
    \centering
    \includegraphics[width=0.90\textwidth]{figures/pgAdmin.png}
     \caption{Captura de pantalla de pgAdmin.}
    \label{fig:pgAdmin}
    \end{figure}

    El número de filas que recupere pgAdmin debe ser igual al número de personas que muestra ATLAS en la sección Data Sorce/Dashboard, en este caso son 2694 personas (Figura \ref{fig:dashboardEJ}).

    \begin{figure}[H]
    \centering
    \includegraphics[width=0.90\textwidth]{figures/dashboardEJ.png}
     \caption{Captura de pantalla de ATLAS Data Sources/Dashboard.}
    \label{fig:dashboardEJ}
    \end{figure}

    \item Por último, otra forma de comprobar que la conexión es correcta y una forma alternativa de realizarla, con el fin de detectar posibles problemas durante la implementación, es ejecutar el siguiente script de código en R, que realiza la conexión con la base de datos a través de RStudio:

    \begin{figure}[H]
    \centering
    \includegraphics[width=0.90\textwidth]{figures/RStudio.png}
     \caption{Captura de pantalla de script de RStudio.}
    \label{fig:RStudio}
    \end{figure}
    
\end{enumerate}

\section{Solución de posibles problemas}

\subsubsection{Error: Connection timeout expired}
Al entrar en la base de datos de Eunomia e introducir la contraseña aparece el error.

    \begin{figure}[H]
    \centering
    \includegraphics[width=0.90\textwidth]{figures/Error03ConnTime.png}
     \caption{Captura de pantalla de error.}
    \label{fig:Error03ConnTime}
    \end{figure}

Solución: Enciende el contenedor de docker.

\subsubsection{No permite establecer la conexión por puerto ocupado.}

 
    \chapter{Conexión local con otras BD}

\section{Obtener otras BD}

- synthea
- synpuf


\section{Establecer la conexión}

- Broadsea viene con su propia BD pero la configuracion de la webApi por defecto permite faciltemente conectar otras BD

- Hay dos formas:

\subsection{A través de pgAdmin}

- Configuración a través del esquema de la webAPi, es bastante intuitivo en verdad

Bibliografia fundamental: https://github.com/OHDSI/WebAPI/wiki/CDM-Configuration
https://forums.ohdsi.org/t/adding-a-new-ms-sql-cdm-database-to-atlas-using-broadsea/19404/3
https://forums.ohdsi.org/t/brodsea-3-0-quick-start-how-to-add-my-own-db-in-atlas-omop/20617/21

\subsection{A través del protocolo de seguridad de ATLAS}

- No procede porque da muchos erroes
- Ni siquiera lo recomiendan aqui: https://forums.ohdsi.org/t/adding-a-new-ms-sql-cdm-database-to-atlas-using-broadsea/19404/2 

- Entiendo que es cuando se va a ejecutar en un entonro empresarial que verdaderamente necesite distinguir usuarios que pueden añadir data sources y usuariosq ue solo pueden manipular la herramienta. En este caso, como es un departamento de investigación todos los investigadores tienen los mismos privilegios, es una tonteria establecer la seguridad. Nada mñas que es una posible fuente de problemas.
    \chapter{Configuración del vocabulario}

\textcolor{red}{ATLAS utiliza un buscador de vocabulario para realizar las definiciones de cohortes...}

El vocabualrio no viene intrinseco en ATLAS Broadsea solo viene una demo.  (see https://github.com/OHDSI/Broadsea-Atlasdb/tree/main)


\section{Descargar el vocabulario}

Para descargar el vocabulario hay que recurrir a la herramienta online de  \href{https://athena.ohdsi.org/}{ATHENA}. Este procedimiento se ha seguido gracias al forum \href{https://forums.ohdsi.org/t/downloading-omop-cdm-version-5-vocabulary-data/3321/3}{Downloading OMOP cdm version 5 vocabulary data}.

\begin{enumerate}

    \item La herramienta de ATHENA presenta un menú superior que oferta dos opciones: (a) búsqueda online en el vocabulario y (b) descarga del vocabulario. En este caso interesa descargar el vocabulario.

    \item Cuando se accede al menú de descarga \textit{Download} se presenta un listado de todos los vocabularios que contiene ATHENA entre los que algunos aparecen preseleccionados. Cada usuario puede seleccionar o deseleccionar los vocabularios que le interesen para el estudio que esté realizando. En este caso, se descargará el vocabulario que ATHENA sugiere por defecto.

    \begin{figure}[H]
        \centering
        \includegraphics[width=0.90\textwidth]{figures/athenaPreDownload.png}
        \caption{Captura de pantalla de la preselección de vocabularios para descargar.}
        \label{fig:athenaPreDownload}
    \end{figure}

    \item La descarga requiere un registro de usuario con un correo electrónico válido al que se enviará un link personal que permitirá la descarga de un archivo .zip con el vocabulario seleccionado. También desde ATHENA en la pestaña \textit{''SHOW HISTORY''} muestra el estado en el que se encuentra la descarga del vocabulario y, una vez que esté listo, permite la descarga directa del zip.

    \item El archivo zip que se descarga, una vez descomprimido, muestra varios archivos .csv \textcolor{red}{con las tablas que conforman la tabla vocabulario} y \textcolor{red}{otros archivos CPT4 que no son necesarios}. Todos estos archivos deben almacenarse en el directorio local de \code{Broadsea/omop\_vocab/files}. En caso de no existir la carpeta \code{}/files, crearla manualmente. 

      \begin{figure}[H]
        \centering
        \includegraphics[width=0.90\textwidth]{figures/omopVocabFiles.png}
        \caption{Captura de pantalla de archivos descargados del vocabulario.}
        \label{fig:omopVocabFiles}
    \end{figure}

    
\end{enumerate}

\section{Configurar el vocabulario}

Configurar el vocabulario requiere acceder a la configuración avanzada del contenedor Docker. Si bien, la primera vez que se inicializó el contenedor se ejecutó el perfil \code{default} ahora se va a ejecutar específicamente el perfil \code{omop-vocab-pg-load}. Esta opción se contempla y presenta en el repositorio de Github de \href{https://github.com/OHDSI/Broadsea}{Broadsea}. 

Para acceder a la información y configuración avanzada del perfil se puede acceder al \code{docker-compose.yml} y a la sección 9 del archivo \code{.env}, aunque este caso no será necesario realizar ninguna modificación sobre los mismos.

\begin{enumerate}

    \item Para comenzar la configuración del vocabulario es necesario inicializar el contenedor \code{omop-vocab-load}. Ejecutar la siguiente línea de código en el \code{cmd}:

    \begin{lstlisting}[language=sh]
    docker compose --profile omop-vocab-pg-load up -d\end{lstlisting}

    Este comando da lugar el siguiente resultado.

      \begin{figure}[H]
        \centering
        \includegraphics[width=0.90\textwidth]{figures/composeProfVocabLoad.png}
        \caption{Captura de pantalla de comando para iniciar el perfil docker.}
        \label{fig:composeProfVocabLoad}
    \end{figure}

    \item Durante la instalación del contenedor es muy importante tener abierto el \code{logs} de Docker. Se recomienda abrirlo desde Docker en vez de desde el \code{cmd} porque este último imprime el \code{logs} que se ha ejecutado hasta el momento en que se ejecuta el comando, no mantiene abierto el \code{logs} hasta que el contenedor se termine de inicializar. Sin embargo, Docker sí muestra el control en tiempo real del proceso. Proceso que ha durado 2h.

    \item La instalación 
    

    
\end{enumerate}

- Docker compose

\section{Solución de posibles errores}

- No haber creado la carpeta /files

- Haber creado la carpeta /files pero vacía

- Se crea el esquema pero se queda vacío: ESPERATE A QUE SE EJECUTE EL CONTENDOR MINIÑA

    \bibliographystyle{unsrtnat}
    \bibliography{bibliografia.bib}


    %\appendix

        

% Fin del documento
\end{document}
