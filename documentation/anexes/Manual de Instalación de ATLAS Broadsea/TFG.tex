% Plantilla TFG LaTeX LSI por:
%   Agustín Borrego <borrego@us.es>
%   Inma Hernández <inmahernandez@us.es>
% Su uso y modificación es libre.

% ̀¡Recuerda hacer copias de seguridad frecuentes durante la redacción del trabajo!
% Puedes descargar todo el código fuente del proyecto en zip en Menú > (Descargar) Fuente

\documentclass[12pt]{report}

% Paquetes LaTeX y estilos globales
\input{etc/pkgs}
\input{etc/style}

%%%%%%%%%%%%%%%%%%%%%%%%%%%%%%%%%%%%%%%%%%%%%%%%%%%%%%%%%%%%%%%%%%%%%%%%%%%%%%%%%%%%%

% Variables para la portada
\setTitle{Extrayendo conocimiento a partir de análisis clínico de datos CDM usando la herramienta Atlas}

%%%%%%%% Extrayendo evidencia utilizando la herramienta ATLAS a partir de datos estandarizados según OMOP CDM


\setAuthor{Da. Maria del Valle Alonso de Caso Ortiz} % Si hay más de un autor, separarlos con \\
\setDegree{Grado en Ingeniería de la Salud} % Cambiar si es necesario
\setSupervisor{Dr. Julián García García \\ Dra. María José Escalona Cuaresma} % Si hay más de un tutor, separarlos con \\
\setDepartment{Lenguajes y Sistemas Informáticos}
\setMonth{junio} % Dejar sólo el mes de la convocatoria en que se presenta el trabajo
\setYear{2023/24} % Por ejemplo, 2022/23

%%%%%%%%%%%%%%%%%%%%%%%%%%%%%%%%%%%%%%%%%%%%%%%%%%%%%%%%%%%%%%%%%%%%%%%%%%%%%%%%%%%%%

% Dedicatoria del trabajo
% Si no se desea incluir, comentar o borrar la línea siguiente para eliminar la página de dedicatoria
\setDedication{A mi padre y a mi madre, por inculcarme la pasión por el estudio y acompañarme incondicionalmente en cada etapa del camino.}

%%%%%%%%%%%%%%%%%%%%%%%%%%%%%%%%%%%%%%%%%%%%%%%%%%%%%%%%%%%%%%%%%%%%%%%%%%%%%%%%%%%%%

% Comienzo del documento
\begin{document}

    % Portada y secciones no numeradas
    \thispagestyle{empty} % Impide que se incluya número de página en la portada
\begin{center}

\vspace*{1cm}

\includegraphics[width=\textwidth]{figures/etsii_us.png}

\vspace*{2.5cm}
\begin{large}
ANEXO A
\end{large}

\vspace*{0.1in}
\textbf{\huge \tfgTitle}

\vspace*{.5in}

{\large Realizado por}\\
\textbf{\Large \tfgAuthors}

\vspace*{2cm}

%\textbf{Para la obtención del título de}\\
%{\large \tfgDegree}

\vspace*{0.2in}

\textbf{Dirigido por}\\
{\large \tfgSupervisor}\\

\vspace*{0.2in}

\textbf{En el departamento de}\\
{\large \tfgDepartment}

\vspace*{.6in}
%\textbf{\Large Convocatoria de \tfgMonth, curso \tfgYear}

\end{center}

% Dedicatoria
\ifdefined\tfgDedication
    \newpage
    \thispagestyle{empty}
    
    \vspace*{\fill}
    \begin{center}
    \textit{\tfgDedication}
    \end{center}
    \vspace*{\fill}
\fi

\clearpage\setcounter{page}{1} % Comienza a incluir números de página a partir de aquí
\pagenumbering{roman} % En números romanos
    \chapter*{Agradecimientos}

A mi familia, a mi padre Francisco José Alonso de Caso, a mi madre María del Valle Ortiz y a mis cuatro hermanos: Manuel, Ignacio, Quico y Juan Pablo; por haber sido apoyo incondicional e inspiración de los valores del trabajo, esfuerzo y sacrificio durante mis años de estudio y durante toda mi vida.

A todos mis compañeros y compañeras de clase, a las que me han acompañado y ayudado en algún momento durante el transcurso del grado de Ingeniería de la Salud y especialmente a aquellas que considero mis amigas, a Patricia, Angela, Marta, Gracia, Andrea y Paloma  que no solo me han acompañado sino que también han amenizado el camino, llenándolo de diversión y pasión por nosotras y por estos estudios que hemos disfrutado juntas.

A todos los profesores con los que he coincidido, especialmente a Julián y María José, que además han tutelado y supervisado este Trabajo Fin de Grado.

A todos los profesionales del departamento de Innovación Tecnológica del Hospital Universitario Virgen del Rocío, que me han guiado durante el período de las prácticas curriculares, apostando por esta iniciativa y ayudándome a llevarla a cabo tutorizando y supervisando su desarrollo, especialmente a Silvia y Carlos.

    \chapter*{Resumen}

Este trabajo consiste en el estudio y aplicación de la herramienta de análisis de datos ATLAS perteneciente a la organización \textit{Observational Health Data Sciences and Informatics} (OHDSI) con la finalidad de promover la estandarización de la investigación observacional con datos de salud. El proyecto se ha desarrollado en colaboración con el grupo de Informática de la Salud Computacional del Hospital Universitario Virgen del Rocío.

La necesidad de estandarización entre los sistemas informáticos y datos de salud es un aspecto de cada vez mayor relevancia a nivel mundial. En este aspecto, OHDSI se alza como una comunidad de investigadores con la finalidad común de unificar la forma de conducir estudios observacionales, a través del Modelo de Datos Común de OMOP y un complejo ecosistema de herramientas de procesamiento y análisis de datos, entre las que destaca ATLAS, una herramienta de análisis de datos \textit{low-code} que permite ejecutar análisis siguiendo una metodología común.

Por la relevancia de la organización, el proyecto consta de una primera parte en la que se recopila información teórica sobre OHDSI, sus estándares y herramientas. Posteriormente, la investigación se complementa con un caso práctico en el que se reproduce con ATLAS un estudio realizado por el grupo de investigadores del hospital sobre posibles efectos adversos del tratamiento radioterápico en pacientes de cáncer de pulmón.

El proyecto en su totalidad confirma la relevancia de OHDSI en el sector de la informática clínica y los beneficios de la aplicación de la herramienta ATLAS y el Modelo de Datos Común de OMOP en la estandarización de la investigación observacional.


\vspace{.5cm}

\textbf{Palabras clave:} Observational Health Data Sciences and Informatics, ATLAS, Modelo de Datos Común de OMOP, Broadsea.
    \chapter*{Abstract}

**Translation:**

This Bachelor's Thesis presents the study and application of the data analysis tool ATLAS, belonging to the organization \textit{Observational Health Data Sciences and Informatics} (OHDSI). The project, which addresses the need for standardization of health information systems and data at the national and international levels, has been developed in collaboration with the Computational Health Informatics Group (GIS) of the Virgen del Rocío University Hospital (HUVR).

In this context, the OHDSI organization stands out as the most important community of researchers with the common goal of unifying the way observational studies are conducted, through the OMOP Common Data Model and a complex ecosystem of data processing and analysis tools, among which ATLAS, the central focus of this work, stands out.

The \textit{low-code} data analysis tool allows for the execution of various analyses depending on the specific use case following a common methodology. Finally, this thesis is completed with a practical case in which, through the deployment of the ATLAS tool via Broadsea, a study on possible adverse effects of radiotherapy treatment in lung cancer patients conducted by HUVR is standardized. The project as a whole confirms the relevance of the OHDSI organization in the clinical informatics sector and the benefits of applying the ATLAS tool and the OMOP Common Data Model in the standardization of health data and observational research.


\vspace{.5cm}

\textbf{Keywords:} ATLAS, Broadsea, Common Data Model, Observational Health Data Sciences and Informatics (OHDSI), Observational Medical Outcomes Partnership (OMOP), Oncology.
    
    % Índice del documento y de figuras
    \begingroup
        % Los enlaces son normalmente azules, pero en los índices se configuran a negro para que no aparezca todo azul
        \hypersetup{linkcolor=black}
        \tableofcontents
        \listoffigures
        \listoftables
        \lstlistoflistings
    \endgroup
    
    % Cambia el estilo de números de página de romanos a normal
    \clearpage\pagenumbering{arabic}
    
    % Capítulos del trabajo
    %\chapter{Ejemplos de uso de LaTeX}\label{cap:ejemplos}

\todo[inline]{Este capítulo se incluye únicamente como ayuda y referencia de uso de \LaTeX. No debe aparecer en el documento final.}

\section{Introducción}
En este capítulo se muestran ejemplos de uso de \LaTeX{} para operaciones comunes. 

\section{Estilos}\label{sec:estilos}
Se pueden aplicar estilos al texto como \textbf{negritas}, \textit{cursiva}, \underline{subrayado} y \texttt{monoespaciado}. También se \textcolor{red}{pueden} \textcolor{blue}{aplicar} \textcolor{green}{colores}, y \underline{\textit{combinar}} \textbf{\textcolor{red}{estilos}}. Se recomienda usar sólo negritas para hacer énfasis, y no abusar de este recurso.

Por comodidad para usuarios no habituados con LaTeX, esta plantilla define algunos alias de comandos más fáciles de recordar para estilos de texto comunes: \negritas{negritas}, \cursiva{cursiva} y \codigo{código}.

\section{Listados}
Con itemize se pueden crear listas no numeradas:

\begin{itemize}
    \item Fresas
    \item Melocotones
    \item Piñas
    \item Nectarinas
\end{itemize}

De manera similar, enumerate permite crear listas numeradas:

\begin{enumerate}
    \item Elaborar la memoria del TFG
    \item Elaborar la presentación
    \item Presentar el TFG
    \item Solicitar el título de Grado
\end{enumerate}

\section{Subsecciones}
Se pueden definir subsecciones con el comando \texttt{subsection}:

\subsection{Primera subsección}\label{sec:subseccion}
Esto es una subsección

\subsection{Segunda subsección}
Esto es otra subsección.

\subsubsection{Sub-sub-sección}
Este es un tercer nivel de profundidad, que no aparece en el índice general. Se recomienda no utilizarlo, si es posible.

\section{Imágenes y figuras}
Todas las imágenes y figuras del documento se incluirán en la carpeta ``figures''. Se pueden incluir de la siguiente manera:

\begin{figure}[htp]
    \centering
    \includegraphics[width=0.7\textwidth]{figures/ejemplo.png}
    \caption{Un feroz depredador}
    \label{fig:ejemplo}
\end{figure}

Observe que las figuras se numeran automáticamente según el capítulo y el número de figuras que hayan aparecido anteriormente en dicho capítulo. Existen muchas maneras de definir el tamaño de una figura, pero se aconseja utilizar la mostrada en este ejemplo: se define el ancho de la figura como un porcentaje del ancho total de la página, y la altura se escala automáticamente. De esta manera, el ancho máximo de una figura sería 1.0 * textwidth, lo que aseguraría que se muestra al máximo tamaño posible sin sobrepasar los márgenes del documento.

Tenga en cuenta que LaTeX intenta incluir las figuras en el mismo sitio donde se declaran, pero en ocasiones no es posible por motivos de espacio. En esos casos, LaTeX colocará la figura lo más cerca posible de su declaración, puede que en una página diferente. Esto es un comportamiento normal.

\section{Tablas}
Existe una gran variedad de formas de crear tablas en LaTeX puro, y todas ellas tienen cierta complejidad. A continuación se muestra un ejemplo simple de tabla nativa, en la Tabla \ref{table:ejemplo}. Se recomienda crear un archivo en la carpeta \textit{tables} por cada tabla nativa que se desee incluir, y enlazarla mediante el comando \texttt{input}.

\begin{table}[htp]
\centering

    % Esta primera línea define las columnas de la tabla. Los posibles tipos de columna son:
    % c: texto centrado
    % l: texto alineado a la izquierda
    % r: texto centrado a la derecha
    % p: columna de ancho fijo
    % Las columnas tienen ancho dinámico según la anchura máxima de los elementos que contengan.

    % Las columnas l/r/c no parten el texto en filas diferentes si éste es demasiado largo. Para ello, puede utilizar el tipo de columna de ancho fijo "p".
    
    % Las barras verticales | se usan para definir los bordes verticales de la tabla. Pruebe a eliminar algunas y observe qué ocurre.
    \begin{tabular}{ | l | c | r | p{2cm} | }
        
        % A continuación van las filas de la tabla. En cada fila, las columnas se separan con el carácter &
        % Para terminar una fila se usa \\
        % Para incluir un borde horizontal entre filas se usa \hline

        % Cabecera con textos en negrita:
        \hline
        \textbf{Columna L} & \textbf{Columna C} & \textbf{Columna R} & \textbf{Columna P}\\
        \hline
        
        % Cuerpo de la tabla:
        Texto de ejemplo & Texto de ejemplo & Texto de ejemplo & Texto de ejemplo\\
        \hline
        ABC & DEF & HIJ & KLM\\
        \hline
        
    \end{tabular} 
    
    \caption{Tabla LaTeX de ejemplo}
    \label{table:ejemplo} 
\end{table}


Para tablas con un formato más complejo, considere la posibilidad de diseñarla usando otro software externo (por ejemplo Excel) e incluirla de manera similar a una figura. \textbf{Observe en el código LaTeX a continuación cómo usar el comando \texttt{captionof\{table\}}, en lugar de simplemente \texttt{caption}, hace que se liste como una Tabla en lugar de como una Figura}:

\begin{figure}[htp]
    \centering
    \includegraphics[width=1.0\textwidth]{tables/complex_table.png}
    \captionof{table}{Tabla compleja introducida como figura}
    \label{table:ejemplo2}
\end{figure}

\section{Referencias}
Observe cómo en el código fuente de esta sección se ha usado varias veces el comando \texttt{label}. Este comando permite marcar un elemento, ya sea capítulo, sección, figura, etc. para hacer una referencia numérica al mismo. Para referenciar una label se usa el comando \texttt{ref} incluyendo el nombre de la referencia:

Este es el capítulo \ref{cap:ejemplos}.

En la sección \ref{sec:estilos} se muestran ejemplos de estilos.

La subsección \ref{sec:subseccion} explica...

En la Figura \ref{fig:ejemplo} vemos que...

Esto evita que tengamos que escribir directamente los índices de las secciones y figuras que queremos mencionar, ya que LaTeX lo hace por nosotros y además se encarga de mantenerlos actualizados en caso de que cambien (pruebe a mover este capítulo al final del documento y observe cómo se actualizan automáticamente todos los índices referenciados). Además, las referencias mediante ``ref'' actúan como hipervínculos dentro del documento que llevan al elemento referenciado al pulsar en ellas.

Es habitual nombrar las ``label'' con un prefijo que indica el tipo de elemento para encontrarlo luego más fácilmente, pero no es obligatorio.

\section{Extractos de código}

Se pueden incluir extractos de código mediante lstlisting:

\begin{lstlisting}[language=Python, caption={Código Python}, label={cod:python}, captionpos=b]
num = float(input("Enter a number: "))
if num > 0:
   print("Positive number")
elif num == 0:
   print("Zero")
else:
   print("Negative number")
\end{lstlisting}

Para evitar tener que incluir el código directamente en el texto del documento, se pueden guardar en archivos separados y referenciarlos:

\lstinputlisting[
    float,
    floatplacement=!htp,
    language=Java,
    label=cod:java,
    caption=Código Java
]{code/java_example.java}

\lstinputlisting[
    float,
    floatplacement=!htp,
    language=html,
    label=cod:html,
    caption=Código HTML
]{code/html_example.html}

\lstinputlisting[
    float,
    floatplacement=!htp,
    language=javascript,
    label=cod:js,
    caption=Código JavaScript
]{code/javascript_example.js}

Los extractos de código también se pueden referenciar mediante label/ref: Extractos de código \ref{cod:python}, \ref{cod:java}, \ref{cod:html}, \ref{cod:js}. 

\section{Enlaces}
Puede enlazar una web externa mediante el comando \texttt{url}: \url{https://www.example.com}. También se puede vincular un enlace a un texto mediante el comando href: \href{https://www.example.com}{dominio de ejemplo}.

\section{Citas y bibliografía}
En LaTeX, los elementos de la bibliografía se almacenan en un fichero bibliográfico en un formato llamado BibTeX, en el caso de este proyecto se encuentran en ``bibliografia.bib''. Para citar un elemento se usa el comando \texttt{cite}. Se pueden citar tanto artículos científicos \cite{borrego2019} como enlaces web \cite{webETSII}. 

También se puede usar el comando \texttt{citet} para incluir una referencia junto con el nombre de su autor o autores: \citet{borrego2021}. Todas las citas se numeran automáticamente y se incluyen en la sección de bibliografía del trabajo. El orden por defecto es según su orden de aparición en el documento. Para ordenarlas por orden alfabético del autor, puede modificar el comando \texttt{bibliographystyle} del archivo principal y reemplazar su valor por el estilo \texttt{plainnat} (orden alfabético, nombres completos) o \texttt{abbrvnat} (orden alfabético, nombres abreviados).

Observe cómo los elementos bibliográficos almacenados en ``bibliografia.bib'' tienen una etiqueta asociada, que es la que se usa al citarlos mediante cite. \textbf{Añadir una referencia al fichero bibliográfico no hace que ésta aparezca automáticamente en la sección de bibliografía del trabajo, es necesario citarla en algún lugar del mismo}.

\section{Ecuaciones}
LaTeX tiene un potente motor para mostrar ecuaciones matemáticas y un amplio catálogo de símbolos matemáticos. El entorno matemático se puede activar de muchas maneras. Para incluir ecuaciones simples en un texto se pueden rodear de símbolos dólar: $1 + 2 = 3$, $\sqrt{81} = 3^2 = 9$, $\forall x \in y~\exists~z : S_z < 4$.

Las ecuaciones más complejas pueden expresarse aparte y son numeradas: ecuación \ref{eq:ecuacion}.

\begin{equation}\label{eq:ecuacion}
\lim_{x\to 0}{\frac{e^x-1}{2x}}
 \overset{\left[\frac{0}{0}\right]}{\underset{\mathrm{H}}{=}}
 \lim_{x\to 0}{\frac{e^x}{2}}={\frac{1}{2}}
 +7 \int_0^2
  \left(
    -\frac{1}{4}\left(e^{-4t_1}+e^{4t_1-8}\right)
  \right)\,dt_1
\end{equation}

Dispone \href{http://www.yann-ollivier.org/latex/texsymbols.pdf}{aquí} de un amplio listado de símbolos que pueden usarse en modo matemático.

\section{Caracteres y símbolos especiales}
Algunos caracteres y símbolos deben ser escapados para poder representarse en el documento, ya que tienen un significado especial en LaTeX. Algunos de ellos son:

\begin{itemize}
    \item El símbolo dólar \$ se usa para ecuaciones.
    \item El tanto por ciento \% se usa para comentarios en el código fuente.
    \item El símbolo euro \euro{} suele dar problemas si se escribe directamente.
    \item El guión bajo \_ se usa para subíndices en modo matemático.
    \item Las comillas deben expresarse `así' para comillas simples y ``así'' para comillas dobles. Las comillas españolas pueden expresarse \textquote{así}.
    \item La barra invertida o contrabarra \textbackslash{} se usa para comandos LaTeX.
    \item Otros símbolos que deben escaparse son las llaves \{ \}, el ampersand \&, la almohadilla \# y los símbolos mayor que \textgreater{} y menor que \textless{}.
\end{itemize}
    \chapter{Introducción, Contexto y Motivación}\label{cap:introduccion}


\section{Introducción}

%Este Trabajo Fin de Grado está orientado a {}, trabaja con {}, pretende {} .....

En esta primera sección de Introducción se presenta el contexto y la motivación que trascienden a la realización del Trabajo Fin de Grado (TFG).
%%-------------------------------------------------------

\section{Contexto}
 
El contexto en el que se desarrolla este TFG, que trata el análisis de datos en el ámbito clínico, se encuentra altamente influenciado por el auge actual de la Industria 4.0 y su potente impacto en el sector sanitario.

%%1. INDUSTRIA 4.0

\subsubsection{Contexto general: Industria 4.0, aparición de nuevas tecnologías}

La Industria 4.0, o cuarta revolución industrial, es un concepto concebido por el gobierno alemán en noviembre de 2011 como una estrategia tecnológica para abordar el crecimiento industrial proyectado para 2020. Su uso internacional se popularizó en abril de 2013 durante la feria industrial de Hannover \textit{Hannover Messe}). Este concepto representa la cuarta fase de la industrialización, sucediendo a la mecanización, electrificación e informatización, y destaca la integración de tecnologías avanzadas \cite{lasi2014industry}..
Se centra principalmente en la digitalización y la necesaria convergencia entre los sistemas físicos y cibernéticos, en inglés \textit{Cyber-Physical Systems (CPS)}. Esta integración se busca a través de nuevas tecnologías de la información y teleocomunicación (TICs), como el internet de las cosas (\textit{Internet of Things o IoT}), la generación y análisis de de datos masivos (\textit{Big Data \& Big Data Analytics}), la computación en la nube (\textit{Cloud Computing}) y el auge de la Inteligencia Artificial (IA) \cite{lasi2014industry}.\cite{chen2020times}\cite{tortorella2020healthcare}


%%2. HEALTHCARE 4.0

\subsubsection{Contexto general aplicado: Sanidad 4.0, características}

La integración de los principios y tecnologías de la Industria 4.0 en el sector sanitario originó el concepto de Salud o Sanidad 4.0 (del inglés, \textit{Healthcare 4.0})\cite{tortorella2020healthcare}\cite{tortorella2021impacts}.  %
En este contexto, este nuevo término se presenta como un complejo desafío  destinado a abordar los nuevos escenarios generados por la creciente demanda de dispositivos y sistemas médicos más eficaces y alineados con las nuevas TICs y los avances ininterrumpidos en las ciencias como la biotecnología y la ingeniería genética. \cite{martin2021ehealth}. La Sanidad 4.0 origina un nuevo ecosistema interseccional del que destacan a lo largo del TFG  tres  características principales: (1) la provisión continua de cuidado sanitario, (2) la orientación de la medicina hacia el paciente y (3) la prevención y predicción de enfermedades.

%Provisión continua de cuidado sanitario - Tecnologías de I4.0 + HCE

(1) La provisión continua del cuidado sanitario se basa en el continuo de salud (\textit{continuum of care}) \cite{kouroubali2019new}. Gracias a las nuevas tecnologías de la Industria 4.0, mayoritariamente a las TICs y al IoT, la sociedad está estrechamente comunicada entre sí de forma prácticamente ininterrumpida. También a raíz de la pandemia del COVID-19 se han acelerado todas estas telecomunicaciones, que en el ámbito sanitario han potenciado el desarrollo de la telemedicina y la salud digital (o \textit{e-Health} \cite{martin2021ehealth}. Con la digitalización y el seguimiento remoto de la salud, los dispositivos médicos que monitorizan a los pacientes en su vida cotidiana generan enormes cantidades de datos médicos de distintas índoles que, además, se recogen con distintos propósitos según cada organización. Los sistemas de salud digital frecuentemente almacenan datos inconsistentes, incoherentes o inaccesibles entre sí, produciéndose registros electrónicos de salud muy extensos y dispares \cite{kouroubali2019new}. 

%Patient-centred - Modelos de datos más amplios. Medicina de precisión.

(2) La orientación de la medicina hacia el paciente se refiere a la priorización del paciente como objetivo central de la provisión de salud  \cite{tortorella2020healthcare}. La atención sanitaria cada vez es más específica a cada individuo, gracias al seguimiento remoto de su actividad diaria y al auge de la medicina de precisión, que es una nueva disciplina médica que considera que el estudio clínico de un paciente debe alcanzar niveles tan detallados como el estudio de su genoma, proteoma, condiciones medioambientales, rutina \cite{ruiz2023inteligencia}... La posición del foco de la salud en el paciente, fomentado por la Unión Europea,  implica reestructurar el sistema sanitario alrededor del mismo, pues debe ser el paciente el cliente final, juez y recibidor de todos los servicios y aplicaciones de la salud digital \cite{ntafi2022legal} \cite{katehakis2019framework}. En términos informáticos esto conlleva reestructurar los sistemas médicos de modo que se recoja de manera central para cada individuo su historial clínico electrónico (HCE) completo, que incluya tanto datos médicos, como farmaceúticos y otros datos de interés. 


%Preventiva y predictiva - Herramientas de big data, IA

(3) La última característica es que sea preventiva y predictiva en vez de reactiva. Esto quiere que decir, que a diferencia de cómo se ha estado realizando tradicionalmente, el enfoque de la medicina debe transicionar, de ser una medicina meramente curativa posterior a la aparición de una enfermedad, a proveer salud previamente a la aparición de una enfermedad de manera que esta enfermedad sea predicha, a través del análisis del HCE del paciente y/o exhaustivos análisis de precisión, y prevenida a través de monitorearización y provisión de tratamientos preventivos \cite{ruiz2023inteligencia}. El análisis del historial clínico de un paciente genera un desafío muy complejo por las características propias de los datos que se han comentado previamente, es decir, por su complejidad, desorden y extensión, de modo que las técnicas de análisis de datos tradicionales generalmente resultan insuficientes. La prevención y la predicción se alcanza gracias al constante desarrollo de técnicas y algoritmos cada vez más sofisticos de inteligencia artificial y aprendizaje automático y herramientas cada vez más poderosas de ciencia y análisis de datos masivos.


\subsubsection{Pilares fundamentales: Interoperabilidad y estandarización a nivel internacional y europeo}

Estas tres características de la Sanidad 4.0 se mantienen firmes sobre un principio fundamental de creciente interés internacional: (a) la estandarización y (b) interoperabilidad de los sistemas médicos. Ambos conceptos están relacionados entre sí mediante una relación causa-consecuencia, según el Institute of Electrical and Electronics Engineers (IEEE, 2013), "la interoperabilidad se hace posible mediante la implementación de estándares" \cite{berryman2013data}.

(a) La implementación de estándares o estandarización consiste principalmente en establecer acuerdos entre las grandes organizaciones de la salud para definir marcos específicos a través de los que estructurar los registros clínicos electrónicos de manera única, reduciendo el desorden y la disparidad de los datos y permitiendo el intercambio de mensajes entre sistemas pertenecientes a distintas organizaciones. La estandarización es un requisito fundamental para alcanzar la interoperabilidad \cite{katehakis2019framework}. Actualmente existen muchos estándares reconocidos internacionalmente, tales como HL7 (Health Level Seven), DICOM (Digital Imaging and Communications in Medicine), SNOMED CT (Systematized Nomenclature of Medicine - Clinical Terms) o IHE (Integrating the Healthcare Enterprise). Con los estándares nace también un concepto importante: el código abierto o \textit{Open Source}. Sin ir más lejos, HL7, la mayor de las organizaciones anteriores comenzó ofreciendo sus servicios de infraestructura de datos y mensajerías de manera privada hasta 2012 cuando se decidió a promover el código abierto liberando la mayor parte de su propiedad intelectual para que pudiera ser accesible de forma gratuita, lo que potenció la adopción de estándares y la interoperabilidad entre las organizaciones sanitarias \cite{berryman2013data}.

(b) Gracias a la popularización de los estándares médicos se está desarrollando cada vez más y mejor la interoperabilidad entre los diferentes sistemas, siendo este es el objetivo final de la revolución industrial, tecnológica y sanitaria actual. A principios de siglo la Comisión Europea identificó la necesidad de interoperabilidad entre las administraciones públicas y se comenzó a desarrollar programas para respaldarla y promoverla \cite{CEU1999ida}. En 2010 se adoptó el primer Marco Europeo de Interoperabilidad (\textit{European Interoperability Framework, EIF}) que junto a los programas Soluciones de interoperabilidad para las administraciones públicas europeas (ISA y ISA\textsuperscript{2}) han asentado las bases para las estrategias actuales. En 2013 El IEEE definió el concepto de interoperabilidad en como "la habilidad de los sistemas de intercambiar información y utilizar dicha información intercambiada de forma efectiva" \cite{berryman2013data} y en 2017 la Unión Europea adoptó el nuevo Marco de Interoperabilidad Europea \textit{(new EIF)}  a través del cual ofrecer recomendaciones, modelos y guianza a fin de mejorar la calidad de los servicios públicos europeos alegando que "la falta de interoperabilidad es el mayor obstáculo para progresar" \cite{kouroubali2019new}. También, en la Comisión Europea del mismo año, se actualizó la definición de interoperabilidad como "la habilidad de las organizaciones de interactuar hacia objetivos mutamente beneficiosos, involucrando el intercambio de información y conocimiento entre dichas organizaciones a través de los procesos empresariales que soportan, es decir, del intercambio de información entre sus sistemas de información TIC" \cite{katehakis2019framework}\cite{CEU2017eif} \cite{casiano2022towards}.

\subsubsection{Iniciativas interoperables a nivel europeo: El camino hacia OHDSI}

En el camino hacia la interoperabilidad en el ámbito sanitario, en noviembre de 2018 se lanzó la Red Europea de Datos y Evidencia en Salud (\textit{European Health Data \& Evidence Network, EHDEN}) con el objetivo de \"abordar los desafíos actuales en la generación de conocimientos y evidencia a partir de datos clínicos del mundo real a escala, para ayudar a los pacientes, médicos, pagadores, reguladores, gobiernos y la industria a comprender el bienestar, las enfermedades, los tratamientos, los resultados y Nuevas terapias y dispositivos". \cite{ehden}. En marzo de 2020 EHDEN comenzó a colaborar con la organización OHDSI (\textit{Observational Health Data Sciences and Informatics}) \cite{ohdsi}, de gran popularidad en el continente americano, para realizar estudios sobre el COVID-19 hasta la actualidad. Apartir de entonces la relación entre ambas entidades se ha estrechado enormente, adquiriendo OHDSI mayor popularidad europea y EHDEN numerosos beneficios por parte de su colaboración, de tal manera que el Catálogo de socios de datos del Portal EHDEN debutó en el Simposio Europeo de OHDSI en junio de 2022. 

Actualmente la presencia de OHDSI en europa es de un interés cada vez mayor y, por ende, también está en auge a nivel estatal \cite{ohdsiSpain}. España es uno de los nodos de colaboración con OHDSI más grandes de europa, y muchas organizaciones a lo largo del territorio español ya están colaborando con sus estándares como la Agencia Española de Medicamentos y Productos Sanitarios (AEMPS) o Quirónsalud entre otros. En Sevilla, la colaboración con OHDSI la llevan a cabo los hospitales universitarios Virgen Macarena y Virgen del Rocío, siendo este último sede del estudio práctico que ha acompañado a este TFG.


%- Hablar de OHDSI EN SEVILLA

%  [Innodata2023]
    
%- La selección de este tópico se debe al creciente interés por el estándar de OHDSI en Sevilla (Hospital Macarena y/o Hospital V del Rocio) y en España.


\section{Motivación}

%Mi motivación personal de entrar en el mundo del %análisis de datos clínicos utilizando  esta %herramienta prometedora..

%RECUERDOS PRINCIPIO DE CARRERA BIG DATA, POLITECNICO DI MILANO...






    








    \chapter{Objetivos del proyecto}\label{cap:objetivos}

\section{Objetivos del TFG}

- Extraer evidencia relevante a partir de datos clínicos observacionales


\section{Objetivos Personales}

- Aumentar mi conocimiento del estándar de OHDSI

- AUmentar mi experiencia en el manejo de datos clínicos

- Aumentar mi conocimeinto en el mundo de análisis de datos

    \chapter{Gestion del Proyecto}\label{cap:gestión}

\section{Introducción}
Breve introducción al capítulo

\section{Participantes del Proyecto}

- Yo M.V ALONSO
- Julian
- M. Jose 

%\section{Estructura de desglose de trabajo}


\section{Estimación de recursos}

- PC, licencias windows, office; recursos open-source de OHDSI a través de youtube, github, docker...

\section{Planificación temporal}

- Scrum, planificación por sprints, estimación del tiempo, desviación...

\section{Evaluación de costes}

- PC, licencias windows, office, teams, ATLAS, OHDSI; gastos indirectos (luz)...

%\section{Identificación de riesgos y planes de contingencia}


    \chapter{Metolodología}\label{cap:04metodologia}

Metodología usada para el proyecto


- sofIA (requisitos)

Metodología usada para el desarrollo del proyecto

- Docker, Github, Postgre

%Servidores, bases de datos del Hospital

- Herramientas de OHDSI


    
    \chapter{Estudio Previo}\label{cap:05EstudioPrevio}

En esta sección se muestra un estudio comprensivo del estandar OHDSI utilizado: qué es, su ......

\section{Introducción}

El trabajo fin de grado se basa en OHDSI

\section{¿Qué es OHDSI?}

OHDSI, pronunciado en inglés ''Odyseey'', son las siglas de Observational Health Data Science and Informatics. OHDSI es una organización colaborativa de ciencia abierta cuyo propósito, de forma muy resumida, es mejorar la investigación cientifico-sanitaria a través de la ciencia de datos y la informática clínica. No obstante, no es solo una organización, sino una comunidad global abierta a todo el que esté interesado y alineado con su misión, visión y objetivos. 

La comunidad se asigna por tanto la misión de ''mejorar la salud empoderando a una comunidad para generar de manera colaborativa evidencia que promueva mejores decisiones de salud y una mejor atención'', y comparte la visión de ''un mundo en el que la investigación observacional produzca una comprensión integral de la salud y la enfermedad'' \cite{OHDSIwebsite}\cite{OHDSIbook}. 

Por otra parte, en \textit{El Libro de OHDSI} la organización se define así misma como ''una comunidad de ciencia abierta que tiene como objetivo mejorar la salud empoderando a la comunidad para generar de manera colaborativa evidencia que promueva mejores decisiones de salud y mejor atención'' \cite{OHDSIbook}. La web oficial presenta otra definición algo diferente, se presenta como ''una colaboración de ciencia abierta, interdisciplinaria y de múltiples partes interesadas para resaltar el valor de los datos de salud a través de análisis a gran escala'' \cite{OHDSIwebsite}.

\begin{figure}[H]
    \centering
    \includegraphics[width=0.90\textwidth]{figures/OHDSIbanner.png}
    \caption{Banner de OHDSI. Extraído de web oficial \cite{OHDSIwebsite}}
    \label{fig:enter-label}
\end{figure}

De esta forma se pueden inferir tres características fundamentales de la organización: (i) ser una comunidad o red colaborativa, (ii) ser de ciencia abierta y (iii) tener la finalidad de promover la extracción de evidencia a partir de datos clínicos.

\begin{enumerate}[label=\roman*.]
    \item La organización es una comunidad, siempre abierta a la incorporación de nuevos colaboradores, lo que muestran con el eslogan \textit{''Join the Journey''}, en español, ''únete a la aventura''. En los capítulos 2.1 y 2.2 del \textit{Libro de OHDSI} se presenta una guía de cómo unirse a la comunidad y participar de sus múltiples eventos. Pero además, OHDSI no es solo una comunidad localizada en un único punto, sino una red colaborativa con múltiples nodos en diferentes países y continentes que comparten el mismo objetivo. Por tanto, los eventos que realiza OHDSI generalmente son de carácter internacional como grupos de trabajo, llamadas comunitarias semanales o los symposium anuales, entre otros.

    \item Además, todos estos eventos y el trabajo que elabora la red son abiertos, puesto que OHDSI es una organización de ''ciencia abierta''. Todos los eventos, publicaciones, herramientas y documentación están disponibles públicamente y de forma gratuita en internet, para que pueda unirse quien quiera (en el caso de los eventos) o consultarse y usarse en cualquier momento (en caso de las herramientas e información). La organización se mantiene económicamente a través del Centro de Coordinación Central, situado en el Centro Médico Irving de la Universidad de Columbia, que es quien asume los costes asociados a la infraestructura central y la coordinación comunitaria por medio del apoyo de los miembros de la comunidad y del patrocinio \cite{OHDSIwebsite}.

    \item Por último, destacar la finalidad de OHDSI de promover la importancia de, no solo recopilar y almacenar la información clínica sino también extraer información o evidencia de ella, que es lo que se denomina comunmente ''el uso secundario de los datos''. Para ello, promueve también la importancia de establecer una metodología estándar para la realización de dichos estudios e investigaciones científicas, persiguiendo \textcolor{red}{los principios FAIR }y favoreciendo la reciclabilidad y reproducidad de los estudios. Gracias a la ciencia abierta, OHDSI promueve a sus colaboradores llevar a cabos sus estudios mediante las herramientas de tratamiento de datos que ofrece y el modelo común de OMOP, fundamentalmente.
    
    
\end{enumerate}

\subsection{Historia}

Es común encontrar en internet los términos OHDSI y OMOP (\textit{Observational Medical Outcomes Partnership}), utilizados de forma casi indistintiva. Si bien es verdad que OMOP se suele asociar mayoritariamente al CDM (\textit{Common Data Model}) también OHDSI mantiene gran relación con este modelo común de datos. Entonces, ¿cuál es la relación entre estas dos entidades? 

La iniciativa de OHDSI se origina en 2014, posterior al proyecto OMOP, que finalizó en 2013, pues la relación que guardan estas dos entidades es parental, OHDSI es la sucesora de OMOP.

OMOP nació en 2008 como una asociación público-privada presidida por la Administración de Alimentos y Medicamentos de EE. UU. con el objetivo de establecer buenas prácticas en estudios observacionales retrospectivos. El proyecto además fue administrado por la Fundación de los Institutos Nacionales de Salud y financiado por un consorcio de compañías farmacéuticas en colaboración con otros investigadores académicos y socios de datos de salud \cite{stang2010advancing}. El propósito inicial de OMOP era impulsar la ciencia de la vigilancia activa de la seguridad de los productos médicos mediante el análisis de datos observacionales de atención médica \cite{stang2010advancing}. Sin embargo, durante su desarrollo, se enfrentó a los desafíos técnicos de llevar a cabo investigaciones en bases de datos observacionales muy heterogéneas entre sí.

El resultado fue el desarrollo de un Modelo Común de Datos (CDM) como un mecanismo para estandarizar la estructura, el contenido y la semántica de los datos observacionales y hacer posible escribir código de análisis estadístico que fuera reutilizable para estudios en distintas fuentes de datos \cite{overhage2012validation}. Los experimentos de OMOP demostraron la viabilidad de establecer un CDM que además reuniese diferentes vocabularios estandarizados, reuniendo en un mismo estándar diversos tipos de datos de diferentes entornos de atención y representados por diferentes vocabularios de origen. Esta característica facilitó la colaboración y aumentó el interés entre diferentes instituciones lo que promovió o un enfoque de ciencia abierta \cite{OHDSIbook}. OMOP puso todo su trabajo a disposición del público, incluidos diseños de estudio, estándares de datos, código de análisis y hallazgos empíricos, para mejorar la transparencia y fomentar la confianza en su investigación. 

Al término del proyecto, el Modelo Común de Datos (CDM) de OMOP había evolucionado hasta respaldar un abanico  amplísimo de aplicaciones analíticas, incluida la efectividad comparativa de intervenciones médicas y políticas de todo el sistema de salud, no solo de la industria farmacéutica, por tanto, el equipo de investigación acordó que el fin de dicho proyecto debería ser el origen de uno nuevo. a partir de esta idea nació OHDSI \cite{OHDSIbook}.

\subsection{Actualidad}

Por tanto, lo que nació en 2014 como la continuación del proyecto OMOP ha evolucionado hasta convertirse en una extensa red colaborativa global.  En la actualidad, cuenta con la participación de más de tres mil colaboradores distribuidos en 80 países.

\begin{figure}[H]
    \centering
    \includegraphics[width=0.90\textwidth]{figures/OHDSIcollaborators.png}
     \caption{Mapa de colaboradores de OHDSI. Extraído de la web oficial \cite{OHDSIwebsite}}
    \label{fig:OHDSIcollaborators}
\end{figure}

La colaboración con OHDSI se realiza a través de las diferentes fuentes de información que aporta la organización. 

Por su característica de ciencia abierta, la información sobre OHDSI está espacida por toda la red de internet mediante publicaciones científicas \cite{OHDSIpublications}, tutoriales para principiantes, grabaciones de las reuniones semanales de la comunidad o las conferencias anuales a través de su canal de youtube \cite{OHDSIyt}, canales de mensajería abierta como discord \cite{OHDSIdiscordInvitation} o MS Teams \cite{OHDSIofficeForm}, cientos de repositorios de github con información técnica de cada herramienta \cite{OHDSIgithub} y los foros de la comunidad para solventar dudas y preguntas \cite{OHDSIforums}, entre otros. No obstante, las fuentes de mayor rigor para acceder a la información sobre la organización son la web oficial \cite{OHDSIwebsite} y el Libro de OHDSI \cite{OHDSIbook}.

Todo el mundo está inivitdo a colaborar. Join the journey. 

- Collaborative network across XX countries, organizations...
- Github....
- Community calls...
-  Symposium

- Importancia europea Tal y como se presenta en \ref{sec:01EstadoArte}: proyectos europeos

\section{¿Cómo generar evidencia?}

- A journey from data to evidence (buscar en la documentación de ohdsi)

\subsection{Building blocks of OHDSI}

(buscar en la documentación de ohdsi)


\subsection{Estandarización de los datos}

- OMOP CDM

- OMOP VOCABULARY

\subsection{Investigación metodológica}
%%ADEMAS DE SER LOS TRES MÉTODOS QUE OFRECE OHDSI PARA REALZIAR INVESTIGACIONNES SON 3 CASOS DE USOS
-Caracterizacion
-Estimacion a nivel de poblacion
-Prediccion a nivel de paciente

\section{Herramientas}
\subsection{ATLAS}

- Extensa descripción de ATLAS (versión actual, anteriores, uso, aspecto...)

- Diferentes tipos de ATLAS (demo, Broadsea, AWS..)

ATLAS ADEMÁS IMPLEMENTA INTRÍNSICAMENTE  DOS HERRAMIENTAS

-ATHENA (herramienta de busqueda en el vocabulario del CDM) actualemnte está implementada dentro de ATLAS/Search

- ACHILLES (data quality dashboard) también esta implementada actualmente dentro de ATLAS/Data source


\subsection{Otras herramientas}

breve descripción de cada una:

-HADES (herramientas de análisis pero en librerias R)

-WHITE-RABBIT y RABBIT-IN-A-HAND (para preparar las ETL)

USAGI (también para la ETL)
...

\section{Conclusiones}

    
    \chapter{Documento de Requisitos}\label{cap:06requisitos}

Este capítulo se divide en cuatro secciones: \ref{sec:06intro} Introducción, \ref{sec:06rf} Requisitos funcionales, \ref{sec:06rnf} Requisitos no funcionales y \ref{sec:06conclusiones} Conclusiones.

\section{Introducción} \label{sec:06intro}

Por la naturaleza informática del proyecto, bajo el convenio del departamento de Lenguajes y Sistemas Informáticos de la US, se presenta este capítulo donde se realiza un catálogo de requisitos del sistema.

En este caso, se ha realizado una adaptación de la ingeniería de requisitos debido a que no se está diseñando una herramienta o sistema de cero, sino que se está modelando un sistema ya existente, el ecosistema de ATLAS Broadsea. La arquitectura del sistema se presenta más detalladamente en el capítulo \ref{cap:08arquitectura} ''Arquitectura del Sistema''.

En este capítulo, en la sección \ref{sec:06rf} ''Requisitos Funcionales'' se presenta el catálogo de requisitos funcionales del sistema.

En la sección \ref{sec:06rnf} ''Requisitos no Funcionales del Sistema'' se presenta el catálogo de requisitos no funcionales del sistema.

Por últirmo, la sección \ref{sec:06conclusiones} ''Conclusiones'' recoge brevemente lo visto en el capítulo.

%R.I%%%%%%%%%%%%%%%%%%%%%%%%%%%%%%%%%%%%%%%%%%%%%%%%%%%%%%%%%%%%%%%%%%%%%
%\section{Requisitos de Información}


%R.F%%%%%%%%%%%%%%%%%%%%%%%%%%%%%%%%%%%%%%%%%%%%%%%%%%%%%%%%%%%%%%%%%%%%%
\section{Requisitos funcionales} \label{sec:06rf}

Los requisitos funcionales son declaraciones que especifican las acciones que un sistema debe realizar en respuesta a entradas específicas del usuario o del sistema. 

Para el sistema de ATLAS Broadsea se han definido seis requisitos funcionales y dos actores o usuarios del sistema: el desarrollador y el analista de datos. Los requisitos funcionales hacen referencia a las tareas que puede realizar el usuario a la hora de conducir un análisis utilizando el sistema. A continuación se presentan: \ref{subsec:06FRdiagrama} ''Diagrama de casos de uso'' y \ref{subsec:06casosUso} ''Casos de uso''.

\subsection{Diagrama de casos de uso} \label{subsec:06FRdiagrama}

El sistema distingue entre dos actores y las actividades que puede realizar cada uno de ellos. Mientras que desarrollador es el usuario encargado principalmente de gestionar el backend del sistema completo de Broadsea, el analista se encarga más específicamente de realizar las tareas de análisis a través de la herramienta de ATLAS. 

Debido a que el proyecto pone el foco mayoritariamente en el uso de la herramienta ATLAS, de los siete requisitos funcionales definidos, seis guardan relación con el analista y las tareas de análisis.


\begin{figure}[H]
    \centering
    \includegraphics[width=0.90\textwidth]{figures/FRdiagram.jpg}
    \caption{Diagrama de casos de uso}
    \label{fig:FRdiagram}
\end{figure}

\subsection{Casos de uso del sistema} \label{subsec:06casosUso}

A continuación se detalla un diagrama de actividad y una tabla descriptiva para caso de uso presentado en la anterior Figura \ref{fig:FRdiagram} ''Diagrama de casos de uso''

\begin{figure}[H]
    \centering
    \includegraphics[width=0.60\textwidth]{figures/FR01.jpg}
    \caption{Diagrama de actividad de RF-01:Añadir base de datos}
    \label{fig:FR01}
\end{figure}

\begin{figure}[H]
    \centering
    \includegraphics[width=0.80\textwidth]{tables/RF01tab.png}
    \captionof{table}{Caso de uso de RF-01:Añadir base de datos}
    \label{table:RF01tab}
\end{figure}

\begin{figure}[H]
    \centering
    \includegraphics[width=0.65\textwidth]{figures/FR02.png}
    \caption{Diagrama de actividad de RF-02: Visualizar Reporte}
    \label{fig:FR02}
\end{figure}

\begin{figure}[H]
    \centering
    \includegraphics[width=0.80\textwidth]{tables/RF02tab.png}
    \captionof{table}{Caso de uso de RF-02:Visualizar Reporte}
    \label{table:RF02tab}
\end{figure}

\begin{figure}[H]
    \centering
    \includegraphics[width=0.60\textwidth]{figures/FR03.png}
    \caption{Diagrama de actividad de RF-03: Definir una cohorte}
    \label{fig:FR03}
\end{figure}

\begin{figure}[H]
    \centering
    \includegraphics[width=0.80\textwidth]{tables/RF03tab.png}
    \captionof{table}{Caso de uso de RF-03:Definir una cohorte}
    \label{table:RF03tab}
\end{figure}

\begin{figure}[H]
    \centering
    \includegraphics[width=0.65\textwidth]{figures/FR04.png}
    \caption{Diagrama de actividad de RF-04: Definir un grupo de conceptos}
    \label{fig:FR04}
\end{figure}

\begin{figure}[H]
    \centering
    \includegraphics[width=0.80\textwidth]{tables/RF04tab.png}
    \captionof{table}{Caso de uso de RF-04:Definir un grupo de conceptos}
    \label{table:RF04tab}
\end{figure}

\begin{figure}[H]
    \centering
    \includegraphics[width=0.65\textwidth]{figures/FR05.png}
    \caption{Diagrama de actividad de RF-05: Realizar Caracterización}
    \label{fig:FR05}
\end{figure}

\begin{figure}[H]
    \centering
    \includegraphics[width=0.80\textwidth]{tables/RF05tab.png}
    \captionof{table}{Caso de uso de RF-05: Realizar caracterización}
    \label{table:RF05tab}
\end{figure}

\begin{figure}[H]
    \centering
    \includegraphics[width=0.65\textwidth]{figures/FR06.png}
    \caption{Diagrama de actividad de RF-06: Realizar caracterización}
    \label{fig:FR06}
\end{figure}

\begin{figure}[H]
    \centering
    \includegraphics[width=0.80\textwidth]{tables/RF06tab.png}
    \captionof{table}{Caso de uso de RF-06: Realizar Estimación a nivel de Población }
    \label{table:RF06tab}
\end{figure}

\begin{figure}[H]
    \centering
    \includegraphics[width=0.80\textwidth]{figures/FR07.png}
    \caption{Diagrama de actividad de RF-07: Realizar Predicción a nivel de Paciente}
    \label{fig:FR07}
\end{figure}

\begin{figure}[H]
    \centering
    \includegraphics[width=0.80\textwidth]{tables/RF07tab.png}
    \captionof{table}{Caso de uso de RF-07: Realizar Predicción a nivel de Paciente }
    \label{table:RF07tab}
\end{figure}


\section{Requisitos no funcionales} \label{sec:06rnf}

Los requisitos no funcionales son restricciones o criterios de calidad que definen cómo debe comportarse un sistema, sin describir funciones específicas.

En base a lo aprendido sobre las características del sistema de ATLAS Broadsea, se han definido seis requisitos no funcionales.

A continuación se muestran estos requisitos de forma general en la Figura \ref{fig:RNFdiagram} ''Diagrama de requisitos no funcionales'' y posteriormente, se añade una tabla descriptiva para cada uno. 

\begin{figure}[H]
    \centering
    \includegraphics[width=0.90\textwidth]{figures/RNFdiagram.jpg}
    \caption{Diagrama de requisitos no funcionales}
    \label{fig:RNFdiagram}
\end{figure}

\begin{figure}[H]
    \centering
    \includegraphics[width=0.90\textwidth]{tables/RNF01.png}
    \captionof{table}{RNF-01: Rendimiento}
    \label{fig:RNF01}
\end{figure}

\begin{figure}[H]
    \centering
    \includegraphics[width=0.90\textwidth]{tables/RNF02.png}
    \captionof{table}{RNF-02: Seguridad }
    \label{fig:RNF02}
\end{figure}

\begin{figure}[H]
    \centering
    \includegraphics[width=0.90\textwidth]{tables/RNF03.png}
    \captionof{table}{RNF-03: Usabilidad }
    \label{fig:RNF03}
\end{figure}

\begin{figure}[H]
    \centering
    \includegraphics[width=0.90\textwidth]{tables/RNF04.png}
    \captionof{table}{RNF-04: Portabilidad}
    \label{fig:RNF04}
\end{figure}

\begin{figure}[H]
    \centering
    \includegraphics[width=0.90\textwidth]{tables/RNF05.png}
    \captionof{table}{RNF-05: Interoperabilidad}
    \label{fig:RNF05}
\end{figure}

\begin{figure}[H]
    \centering
    \includegraphics[width=0.90\textwidth]{tables/RNF06.png}
    \captionof{table}{RNF-06: Mantenimiento}
    \label{fig:RNF06}
\end{figure}

\section{Conclusiones} \label{sec:06conclusiones}

De este capítulo se concluye que, aunque el objetivo del proyecto no sea específicamente diseñar un sistema, el análisis de requisitos es de gran relevancia y utilidad  para esquematizar y comprender las funcionalidades del sistema y sus propiedades.

Gracias a este análisis se abstrae de forma más sencilla el funcionamiento y las tareas que realiza el sistema de Broadsea, que en realidad es bastante más complejo.


    \chapter{Entorno de Trabajo}\label{cap:07diseño}

\section{Introducción}
En este capítulo explicaremos...

\section{Arquitectura de Broadsea}

Presentación del sistema Broadsea. Imagen de Broadsea

\begin{figure}[H]
    \centering
    \includegraphics[width=0.90\textwidth]{figures/OHDSIBroadsea3.0.png}
    \caption{Vista general de todos los componentes de Broadsea}
    \label{fig:OHDSIBroadsea3.0}
\end{figure}

-{WebAPI}

- {ATLAS}

\section{Entorno tecnológico} %Herramientas

\subsection{Postgre}

\subsection{Docker}

\subsection{Repositorio github}



\section{Conclusiones}
En este capítulo concluimos que...

    \chapter{Plan de pruebas}\label{cap:08pruebas}
%%%%%%%%%%%%%%%55

Este capítulo podría ser más bien "Casos prácticos"

\section{Introducción}

\section{Caso práctico}

Reproducción del estudio oncológico realizado por los investigadores del HUVR pero utilizando herramienta ATLAS

%\section{Pruebas Unitarias}
%\section{Pruebas de Integración}
%\section{Pruebas de Carga o Rendimiento}
%\section{Pruebas Funcionales}

\subsection{Comprobación calidad datos}

- Datos omopizados por TFG Paco 
- Calidad previa y post comprobada tb en TFG Paco

\subsection{Realización del estudio}

Check de los casos de uso/requisitos en el estudio real.

ej de la obtención dle reporte de la BD,
ej de la creación de un cohorte concreto para un estudio concreto,
ej de la predicción a nivel de paciente para un estudio concreto....

todos los ejemplos anteriores seguirían un mismo hilo conductor en cuanto al ESTUDIO CONCRETO

\section{Conclusiones}

En este capítulo concluimos que...

    \input{sections/09_Resultados}

    %\input{sections/XX_Implementacion}

    \chapter{Conclusiones}\label{cap:10conclusiones}




    \bibliographystyle{unsrtnat}
    \bibliography{bibliografia.bib}


    \appendix
    \chapter{Manual de usuario}\label{anexo:manual}

El anexo es un documento aparte

La estructura es...

El repositorio github..
    \chapter{Glosario}\label{anexo:glosario}


        

% Fin del documento
\end{document}
