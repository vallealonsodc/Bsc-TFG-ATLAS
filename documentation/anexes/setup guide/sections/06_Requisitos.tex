\chapter{Documento de Requisitos}\label{cap:requisitos}

\section{Introducción}


%R.I%%%%%%%%%%%%%%%%%%%%%%%%%%%%%%%%%%%%%%%%%%%%%%%%%%%%%%%%%%%%%%%%%%%%%
%\section{Requisitos de Información}


%R.F%%%%%%%%%%%%%%%%%%%%%%%%%%%%%%%%%%%%%%%%%%%%%%%%%%%%%%%%%%%%%%%%%%%%%
\section{Requisitos Funcionales}

- RF00: Cargar datasets //De hecho este aún no sabemos como hacerlo. Estamos trabajando con datasets que ya vienen cargados

- RF01: Obtener un reporte del data set (Data source)

- RF02 : Definir un conjunto de conceptos del Vocabulario (Concept set)

- RF03: Configurar la muestra de trabajo (cohort definition)

- RF04: Caracterizar el cohort (characterization - primer gran bloque de metodología de OHDSI)

- RF05: Definir una etsimacion a nivel de poblacion (estimation - segundo gran bloque de metodologia de OHDSI)

- RF06: Hacer una prediccion a nivel de paciente (prediction - tercer gran bloque de metodologia de OHDSI)



Más cosas que se pueden hacer y no definimos la otra vez:

- Obtener un reporte de la ruta del cohorte (cohort pathway)

- Analizar los ratios de incidencia de un outcome (INcidence rate)

- Obteenr un reporte de los datos para un paciente concreto (profile)

\subsection{Diagramas de casos de uso}


\subsection{Descripción del requisito}


%\subsection{Diagramas de casos de uso}
%\subsection{¿¿¿¿¿Descripción del requisito funcional????}


%\section{Modelo conceptual}%%%%%%%%%%%%%%%%%%%%%%%%%%%%%%%%%%%%%%%%%%%%%%

%\section{Mockups}%%%%%%%%%%%%%%%%%%%%%%%%%%%%%%%%%%%%%%%%%%%%%%%%%%%%%%%%

%R.N.F%%%%%%%%%%%%%%%%%%%%%%%%%%%%%%%%%%%%%%%%%%%%%%%%%%%%%%%%%%%%%%%%%%%%
\section{Requisitos no Funcionales}





\section{Conclusiones}
En este capítulo concluimos que...
