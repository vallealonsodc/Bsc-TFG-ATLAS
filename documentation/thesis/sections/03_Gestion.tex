\chapter{Gestión del Proyecto}\label{cap:03gestión}

%\section{Introducción}
En este capítulo se presenta toda la información relacionada con la gestión del proyecto de la elaboración del TFG. El capítulo se divide en seis secciones: \ref{sec:03Participantes} Participantes del Proyecto, \ref{sec:03EDT} Estructura de Desglose de Trabajo, \ref{sec:03Recursos} Estimación de recursos, \ref{sec:03Temporal} Planificación temporal, \ref{sec:03Costes} Evaluación de costes y \ref{sec:03Riesgos} Identificación de riesgos y planes de contingencia.

\section{Participantes del Proyecto} \label{sec:03Participantes}

Los participantes del proyecto TFG se presentan a continuación mediante una tabla que recoge su nombre, institución a la que pertenece, rol asignado durante la elaboración del proyecto, tareas asignadas durante la elaboración del proyecto e información de contacto. 

Es importante destacar que los tres primeros participantes corresponden a la propia alumna y tutores de la Escuela Técnica Superior de Ingeniería Informática de la Universidad de Sevilla y los dos últimos participantes, a los tutores de las prácticas realizadas en el Departamento de Innovación Tecnológica del Hospital Universitario Virgen del Rocío.

\begin{table}[H]
    \centering
    \begin{tabular}{|c|c|}
    \hline
    \textbf{Participante} & María del Valle Alonso de Caso Ortiz \\
    \hline
    \textbf{Institución} & Universidad de Sevilla \\
    \hline
    \textbf{Rol} & Jefe de Proyecto \& Developer \& Analista \\
    \hline
    \textbf{Información de contacto} & vallealonsodecaso@icloud.com \\ 
    \hline
    \end{tabular}
\caption{Descripción del primer participante del proyecto}
\label{tab:primerParticipante}
\end{table}

\begin{table}[H]
    \centering
    \begin{tabular}{|c|c|}
    \hline
    \textbf{Participante} & Julián García García \\
    \hline
    \textbf{Institución} & Universidad de Sevilla \\
    \hline
    \textbf{Rol} & Tutor del TFG \& Supervisor \\
    \hline
    \textbf{Información de contacto} & juliangg@us.es \\
    \hline
    \end{tabular}
\caption{Descripción del segundo participante del proyecto}
\label{tab:segundoParticipante}
\end{table}

\begin{table}[H]
    \centering
    \begin{tabular}{|c|c|}
    \hline
    \textbf{Participante} & María José Escalona Cuaresma \\
    \hline
    \textbf{Institución} & Universidad de Sevilla \\
    \hline
    \textbf{Rol} &  \\
    \hline
    \textbf{Tareas asignadas} & \\
    \hline
    \textbf{Información de contacto} & \\
    \hline
    \end{tabular}
\caption{Descripción del tercer participante del proyecto}
\label{tab:trazabilidadAlum}
\end{table}

\begin{table}[H]
    \centering
    \begin{tabular}{|c|c|}
    \hline
    \textbf{Participante} & Silvia Rodríguez Mejías \\
    \hline
    \textbf{Institución} & Hospital Universitario Virgen del Rocío \\
    \hline
    \textbf{Rol} &  \\
    \hline
    \textbf{Tareas asignadas} & \\
    \hline
    \textbf{Información de contacto} & \\
    \hline
    \end{tabular}
\caption{Descripción del cuarto participante del proyecto}
\label{tab:cuartoParticipante}
\end{table}

\begin{table}[H]
    \centering
    \begin{tabular}{|c|c|}
    \hline
    \textbf{Participante} & Carlos Luis Parra Calderón \\
    \hline
    \textbf{Institución} & Hospital Universitario Virgen del Rocío \\
    \hline
    \textbf{Rol} & Supervisor de prácticas en empresa  \\
    \hline
    \textbf{Información de contacto} & \\
    \hline
    \end{tabular}
\caption{Descripción del quinto participante del proyecto}
\label{tab:quintoParticipante}
\end{table}

\section{Estructura de desglose de trabajo} \label{sec:03EDT}



\section{Estimación de recursos} \label{sec:03Recursos}

- PC, licencias windows, office; recursos open-source de OHDSI a través de youtube, github, docker...

\section{Planificación temporal} \label{sec:03Temporal}

- Scrum, planificación por sprints, estimación del tiempo, desviación...

\section{Evaluación de costes} \label{sec:03Costes}

- PC, licencias windows, office, teams, ATLAS, OHDSI; gastos indirectos (luz)...

\section{Identificación de riesgos y planes de contingencia} \label{sec:03Riesgos}
