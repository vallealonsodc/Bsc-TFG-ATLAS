\chapter{Introducción, Estado del Arte y Motivación}\label{cap:introduccion}

Este primer capítulo del Trabajo Fin de Grado (TFG) se divide en tres secciones: se presenta en \ref{sec:01Intro} una introducción descriptiva del mismo, en \ref{sec:01EstadoArte} el contexto original y el estado del arte actual del estudio y en \ref{sec:01Motivacion} la motivación que trasciende a la realización del trabajo.

\section{Introducción} \label{sec:01Intro}

Este Trabajo Fin de Grado ha sido realizado por María del Valle Alonso de Caso Ortiz, alumna del grado de Ingeniería de la Salud por la Universidad de Sevilla (US), de la promoción 2020-2024 y bajo la tutela de D. Julián A. García García y Da. Maria J. Escalona Cuaresma, ambos pertenecientes al departamento de Lenguajes y Sistemas Informáticos de la Escuela Técnica Superior de Ingeniería Informática (ETSII) de la misma universidad. Además se realiza en conjunto con el Departamento de Innovación Tecnológica del Hospital Universitario Virgen del Rocío, mediante un convenio de prácticas curriculares de 337 horas, donde han ejercido la tutela D. Silvia Rodríguez Mejías y D. Carlos Luis Parra Calderón. Toda la información técnica relativa al TFG como los Objetivos del proyecto, la Gestión del proyecto y la Metodología empleada se presentan en los capítulos \ref{cap:02objetivos}, \ref{cap:03gestión} y \ref{cap:04metodologia}, respectivamente.

Por otra parte, el trabajo abarca una perspectiva teórica muy amplia sobre la importancia y las estrategias de estandarización y análisis de datos clínicos a nivel global y europeo, que se presentan en el Estado del Arte (véase \ref{sec:01EstadoArte}). Además, el trabajo se especializa en las herramientas y estándares propuestos por la organización Observational Health Data Sciences and Informatics (OHDSI), de creciente relevancia a nivel global, y que se presenta en profundidad en el Estudio Previo (véase \ref{cap:05EstudioPrevio}).

Una vez se presentan los conocimientos teóricos generales sobre la organización OHDSI y las herramientas más relevantes, el trabajo se especializa en la implementación y estudio de datos clínicos a través de ATLAS. Concretamente, la herramienta se despliega e implementa a través de la iniciativa \textit{Broadsea}, que origina la denominación \textit{ATLAS Broadsea} de la herramienta a lo largo del TFG. El marco teórico en el que se presenta la herramienta abarca los capítulos \ref{cap:05EstudioPrevio}, \ref{cap:06requisitos} y \ref{cap:07diseño}.

En tercer lugar, debido a la naturaleza práctica del TFG, por haber sido desarrollado en colaboración con el HUVR, se realiza \textcolor{red}{un estudio real mediante ATLAS Broadsea de una base de datos clínica proporcionada por el hospital} (véase \ref{cap:08pruebas}). Además, se desarrolla el Anexo A como un documento de gran extensión y relevancia que describe en profundidad el proceso de instalación, despliegue y configuración de ATLAS Broadsea. Este Anexo es de gran interés porque no existe en internet ninguna guía completa de estas características sobre la herramienta, aportando un contenido a la comunidad científica y a los compañeros del hospital de gran valor.

Por último, los últimos dos capítulos \ref{cap:09resultados} y \ref{cap:10conclusiones} presentan una recopilación de resultados y conclusiones, respectivamente, obtenidos al término del desarrollo del TFG. También se adjunta el Anexo B, que consiste en un Glosario de Términos técnicos relevantes para la comprensión del trabajo. 

Adicionalmente, por su naturaleza informática, este TFG se ha desarrollado paralelamente a un repositorio de github del proyecto \cite{vallealonsodc}, que ha servido como controlador de versiones y como administrador de archivos en la nube, permitiendo almacenar y compartir con el lector archivos relevantes al TFG, ya sea archivos necesarios para el despliegue de la herramienta, archivos producidos durante el análisis o los propios documentos en sí mismos.

\section{Estado del Arte} \label{sec:01EstadoArte} 
 
El contexto en el cual se desarrolla el TFG se encuentra notablemente influido por el surgimiento de la Industria 4.0 y las nuevas tecnologías que la acompañan, así como por su impacto significativo y transformador en el sector sanitario que ha generado nuevas necesidades a nivel mundial y europeo, más concretamente en el mundo del análisis de datos, entre otros. %que ha desembocado en Sevilla para ser el tópico trascendental de este trabajo.

A continuación se realiza un recorrido por todos los aspectos teóricos fundamentales que dan origen y sentido a este TFG. Como punto de partida se presenta el concepto de Industria 4.0 seguido de su aplicación en la sanidad, que da lugar a la Sanidad 4.0. A raíz de este concepto se desarrollan las características más relevantes del panorama tecnológico-sanitario emergente, incidiendo especialmente en la necesidad de interoperabilidad y estandarización de las tecnologías sanitarias. Frente a estas necesidades se presentan soluciones y estándares actuales ampliamente aceptados a nivel global, destacando en particular la creciente importancia del estándar de OHDSI, que constituye el foco central del trabajo.

%%1. INDUSTRIA 4.0

\subsubsection{Contexto general: Industria 4.0, aparición de nuevas tecnologías y aplicación en el ámbito sanitario}

La Industria 4.0, o cuarta revolución industrial, es un concepto concebido por el gobierno alemán en noviembre de 2011 como una estrategia tecnológica para abordar el crecimiento industrial proyectado para 2020. Su uso internacional se popularizó en abril de 2013 durante la feria industrial de Hannover \textit{Hannover Messe}). Este concepto representa la cuarta fase de la industrialización, sucediendo a la mecanización, electrificación e informatización, y destaca la integración digital de tecnologías avanzadas \cite{lasi2014industry}..
Se centra principalmente en la digitalización y la necesaria convergencia entre los sistemas físicos y cibernéticos (\textit{Cyber-Physical Systems, CPS}). Esta integración se busca a través de nuevas tecnologías de la información y telecomunicación (TICs), como el internet de las cosas (\textit{Internet of Things, IoT}), la generación y análisis de datos masivos (\textit{Big Data \& Big Data Analytics}), la computación en la nube (\textit{Cloud Computing}) y el auge de la Inteligencia Artificial (IA) \cite{lasi2014industry}.\cite{chen2020times}\cite{tortorella2020healthcare}


%%2. HEALTHCARE 4.0

%\subsubsection{Contexto general aplicado: Sanidad 4.0, características}

La integración de los principios y tecnologías de la Industria 4.0 en el sector sanitario originó el concepto de Salud o Sanidad 4.0 (del inglés, \textit{Healthcare 4.0})\cite{tortorella2020healthcare}\cite{tortorella2021impacts}.  %
En este contexto, este nuevo término se presenta como un complejo desafío  destinado a abordar los nuevos escenarios generados por la creciente demanda de dispositivos y sistemas médicos más eficaces y alineados con las nuevas TICs y los avances ininterrumpidos en ciencias como la biotecnología y la ingeniería genética. \cite{martin2021ehealth}. La Sanidad 4.0 origina un nuevo ecosistema interseccional del que se destacan, y se exploran a lo largo del desarrollo del TFG,  tres  características principales: (1) la provisión continua de cuidado sanitario, (2) la orientación de la medicina hacia el paciente y (3) la prevención y predicción de enfermedades.


\begin{enumerate}

    %Provisión continua de cuidado sanitario - Telemedicina, salud digital
    \item La provisión continua del cuidado sanitario se basa en el cuidado continuo (\textit{continuum of care}) \cite{kouroubali2019new}. Gracias a las nuevas tecnologías de la Industria 4.0, mayoritariamente a las TICs y al IoT, la sociedad se encuentra estrechamente comunicada entre sí de forma prácticamente ininterrumpida. También a raíz de la pandemia del COVID-19 se han acelerado las telecomunicaciones, que en el ámbito sanitario han potenciado el desarrollo de la telemedicina y la salud digital (o \textit{e-Health} \cite{martin2021ehealth} a través del desarrollo de programas informáticos para teleconsultas, monitores de actividad mediante pulseras o relojes, nuevos implantes inteligentes y un largo etcétera. Con la digitalización y el seguimiento continuo de la salud, los dispositivos médicos que monitorizan a los pacientes en su vida cotidiana generan enormes cantidades de datos médicos de distintas índoles que, además, cada organización recoge con distintos propósitos y forma lo que conlleva que los sistemas de salud digital frecuentemente almacenen grandísimas cantidades de datos inconsistentes, incoherentes o inaccesibles entre sí, produciéndose registros electrónicos de salud muy extensos y dispares. \cite{kouroubali2019new}. 

    %Patient-centred - Modelos de datos más amplios. Medicina de precisión.
    \item La orientación de la medicina hacia el paciente se refiere a la priorización del paciente como objeto central de la provisión de salud  \cite{tortorella2020healthcare}. La atención sanitaria cada vez es más específica para cada individuo, gracias al seguimiento remoto de su actividad diaria y al auge de la medicina de precisión. Esta última constituye una nueva disciplina médica que aboga por un estudio clínico detallado que incluya aspectos como genoma, proteoma, condiciones medioambientales o rutina de vida del paciente \cite{ruiz2023inteligencia}. La posición del foco de la salud en el paciente, fomentado por la Unión Europea,  implica reestructurar el sistema sanitario alrededor del mismo, pues el paciente debe ser el cliente final, juez y recibidor de todos los servicios y aplicaciones de la salud digital \cite{ntafi2022legal} \cite{katehakis2019framework}. En términos informáticos esto implica la reconfiguración de los sistemas médicos de modo que se recoja de manera central para cada individuo su historial clínico electrónico (HCE) completo, que incluya tanto datos médicos, como farmaceúticos y otros datos de interés.  

    %Preventiva y predictiva - Herramientas de big data, IA
    \item La última característica es que sea preventiva y predictiva en lugar de reactiva. Esto quiere que decir, que a diferencia del enfoque tradicional en el que la medicina es meramente curativa posterior a la aparición de una enfermedad, se debe transicionar hacia la provisión de salud de manera previa a la aparición de una enfermedad de modo que esta enfermedad sea predicha, a través del análisis del HCE del paciente y/o exhaustivos análisis de precisión, y prevenida a través de monitorearización y provisión de tratamientos preventivos en el cuidado continuo de la salud \cite{ruiz2023inteligencia}. En esta línea el análisis del historial clínico de un paciente genera un desafío muy complejo por las características inherentes a los datos ya comentadas, es decir, por su complejidad, desorden y extensión, de modo que las técnicas de análisis de datos tradicionales habitualmente resultan insuficientes. La prevención y la predicción se alcanza gracias al constante desarrollo de técnicas y algoritmos cada vez más sofisticados de inteligencia artificial y aprendizaje automático y herramientas cada vez más poderosas de ciencia y análisis de datos masivos.

\end{enumerate}


%%3. INTEROPERABILIDAD Y ESTANDARES

\subsubsection{Pilares fundamentales: Interoperabilidad y estandarización a nivel internacional y europeo}

Estas tres características de la Sanidad 4.0 se mantienen firmes sobre dos principios fundamentales de creciente interés internacional: (a) la estandarización y (b) la interoperabilidad de los sistemas médicos. Ambos conceptos están relacionados entre sí mediante una relación causa-consecuencia, según el Institute of Electrical and Electronics Engineers (IEEE, 2013), "la interoperabilidad se hace posible mediante la implementación de estándares" \cite{berryman2013data}.

\begin{enumerate}[label=\alph*.]

    \item La implementación de estándares o estandarización consiste principalmente en establecer acuerdos entre las grandes organizaciones de la salud para definir marcos específicos a través de los que estructurar los registros clínicos electrónicos de manera única, reduciendo el desorden y la disparidad de los datos y permitiendo el intercambio de mensajes entre sistemas pertenecientes a distintas organizaciones. La estandarización es un requisito fundamental para alcanzar la interoperabilidad \cite{katehakis2019framework}. Actualmente existen muchos estándares reconocidos y utilizados internacionalmente, tales como HL7 (Health Level Seven), DICOM (Digital Imaging and Communications in Medicine), SNOMED CT (Systematized Nomenclature of Medicine - Clinical Terms) o IHE (Integrating the Healthcare Enterprise). Con los estándares nace también un concepto importante: el código abierto o \textit{Open Source}. Sin ir más lejos, HL7, la mayor de las organizaciones anteriores comenzó ofreciendo sus servicios de infraestructura de datos y mensajerías de manera privada hasta 2012 cuando se decidió a promover el código abierto liberando la mayor parte de su propiedad intelectual para que pudiera ser accesible de forma gratuita, lo que potenció la adopción de estándares y la consecuente interoperabilidad entre las organizaciones sanitarias \cite{berryman2013data}.

    \item  Gracias a la popularización de los estándares médicos se está desarrollando cada vez más y mejor la interoperabilidad entre los diferentes sistemas, siendo este es el objetivo final de la revolución industrial, tecnológica y sanitaria actual. A principios de siglo la Comisión Europea identificó la necesidad de interoperabilidad entre las administraciones públicas y se comenzó a desarrollar programas para respaldarla y promoverla \cite{CEU1999ida}. En 2010 se adoptó el primer Marco Europeo de Interoperabilidad (\textit{European Interoperability Framework, EIF}) que junto a los programas Soluciones de interoperabilidad para las administraciones públicas europeas (ISA y ISA\textsuperscript{2}) han asentado las bases para las estrategias actuales. En 2013 el IEEE definió el concepto de interoperabilidad como "la habilidad de los sistemas de intercambiar información y utilizar dicha información intercambiada de forma efectiva" \cite{berryman2013data}. Recientemente, en 2017 la Unión Europea adoptó el nuevo Marco de Interoperabilidad Europea (\textit{new EIF})  a través del cual ofrecer recomendaciones, modelos y guianza a fin de mejorar la calidad de los servicios públicos europeos alegando que "la falta de interoperabilidad es el mayor obstáculo para progresar" \cite{kouroubali2019new}. También, en la Comisión Europea del mismo año, se actualizó la definición de interoperabilidad como "la habilidad de las organizaciones de interactuar hacia objetivos mutamente beneficiosos, involucrando el intercambio de información y conocimiento entre dichas organizaciones a través de los procesos empresariales que soportan, es decir, del intercambio de información entre sus sistemas de información TIC otorgando una importancia cada vez mayor al concepto \cite{katehakis2019framework}\cite{CEU2017eif} \cite{casiano2022towards}.

\end{enumerate}


%%4. CAMINO HACIA OHDSI

\subsubsection{El camino hacia OHDSI}

En el camino hacia la interoperabilidad y estandarización en el ámbito sanitario a nivel europeo, en noviembre de 2018 se lanzó la Red Europea de Datos y Evidencia en Salud (\textit{European Health Data \& Evidence Network, EHDEN}) con el objetivo de ''abordar los desafíos actuales en la generación de conocimientos y evidencia a partir de datos clínicos del mundo real a escala, para ayudar a los pacientes, médicos, pagadores, reguladores, gobiernos y la industria''. \cite{ehden}. En marzo de 2020 EHDEN comenzó a colaborar con la organización OHDSI (\textit{Observational Health Data Sciences and Informatics}) \cite{ohdsi}, de gran popularidad en el continente americano, para realizar estudios sobre el COVID-19. Apartir de entonces y hasta la actualidad la relación entre ambas entidades se ha estrechado enormente, adquiriendo OHDSI mayor popularidad europea y EHDEN numerosos beneficios por parte de su colaboración, de hecho, el Catálogo de socios de datos del Portal EHDEN debutó en el Simposio Europeo de OHDSI en junio de 2022. 

\textcolor{red}{DARWIN EU} https://www.ohdsi.org/darwin-eu-initiative-presentation/

Actualmente la presencia de OHDSI en Europa es cada vez mayor y, a nivel estatal España es uno de los nodos de colaboración con OHDSI más grandes de Europa, y muchas organizaciones a lo largo del territorio español ya están colaborando con sus estándares como la Agencia Española de Medicamentos y Productos Sanitarios (AEMPS) o Quirónsalud entre otros \cite{ohdsiSpain}. En Sevilla, la colaboración con OHDSI la llevan a cabo el IBIS (Instituto de Biomedicina de Sevilla), la fundación FISEVI (Fundación para la Gestión de la Investigación en Salud en Sevilla) y los hospitales universitarios Virgen Macarena y Virgen del Rocío. Sin ir más lejos, el pasado octubre de 2023 el hospital Macarena celebró el 'Innodata 2023', un congreso nacional sobre investigación de datos en salud, en la que se presentó una ponencia que trató las herramientas y experiencias de OHDSI. Por otro lado, el Virgen del Rocio también está desarrollando proyectos innovadores alineados con esta temática a cargo del departamento de Innovación Tecnológica del hospital, siendo esta la sede del estudio práctico que ha acompañado al desarrollo del TFG, concluyendo así el marco contextual del trabajo.


%%---------------------------------------------------------------------------

\section{Motivación} \label{sec:01Motivacion}

%Mi motivación personal de entrar en el mundo del %análisis de datos clínicos utilizando  esta %herramienta prometedora..

%RECUERDOS PRINCIPIO DE CARRERA BIG DATA, POLITECNICO DI MILANO...

Puedo afirmar que mi curiosidad e interés por el mundo de la ciencia de datos ha sido una constante a lo largo de mis cuatro años de estudio y la principal motivación para realizar este trabajo de fin de grado. El origen se sitúa en el primer año de carrera, allá en el 2020, cuando por primera vez el profesor de estadística nos habló a mi y a mis compañeros sobre el 'Big Data' como una disciplina emergente de gran interés a nivel laboral. Esta primera toma de contacto, fue la que me llevó a continuar investigando sobre dicha disciplina y todo lo relacionado con ella. En tercero de carrera tuve la oportunidad de realizar el programa de movilidad ERASMUS al Politecnico di Milano, una de las mejores universidades de ingeniería del mundo \cite{QSPolimi}, por lo que opté a seleccionar el mayor número de asignaturas de Data Science que mi convenio de estudios me permitió. Este año de estudio en Milán confirmó que, lo que había nacido como una mera curiosidad, se había convertido en una pasión, por lo que a mi regreso del Erasmus me decidí a orientar mi carrera profesional y mi TFG hacia el mundo del análisis de datos clínicos, hasta el día de hoy en que este trabajo es escrito.

También ha sido de gran importancia la motivación de mis profesores y tutores de la Escuela Técnica Superior de Ingeniería Informática de la Universidad de Sevilla y la colaboración, mediante el convenio de prácticas, del grupo científico del Departamento de Innovación Tecnológica del Hospital Universitario Virgen del Rocío, quienes confiando en mi me han apoyado, motivado y dado las herramientas y conocimientos necesarios para completar mi formación sobre ATLAS y OHDSI y la informática clínica en general.
