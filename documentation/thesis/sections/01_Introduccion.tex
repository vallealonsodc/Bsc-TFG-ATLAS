\chapter{Introducción, Contexto y Motivación}\label{cap:introduccion}

%%-------------------------------------------------------------------------

\section{Introducción}

\textit{Este Trabajo Fin de Grado está orientado a ..., trabaja con ..., pretende .....}

En esta primera sección de Introducción se presenta el contexto y la motivación que trascienden a la realización del Trabajo Fin de Grado (TFG).


%%------------------------------------------------------------------------

\section{Contexto}
 
El contexto en el cual se desarrolla el presente TFG, dedicado al análisis de datos clínicos a través de herramientas OHDSI, se ve notablemente influido por la actual expansión de la Industria 4.0 y las nuevas teconologías que la acompañan, así como por su impacto significativo y transofrmador en el ámbito sanitario. Este contexto ha generado nuevas necesidades a nivel mundial y europeo, que han desembocado en sevilla para ser tópico trascendental de este trabajo.

%%1. INDUSTRIA 4.0

\subsubsection{Contexto general: Industria 4.0, aparición de nuevas tecnologías}

La Industria 4.0, o cuarta revolución industrial, es un concepto concebido por el gobierno alemán en noviembre de 2011 como una estrategia tecnológica para abordar el crecimiento industrial proyectado para 2020. Su uso internacional se popularizó en abril de 2013 durante la feria industrial de Hannover \textit{Hannover Messe}). Este concepto representa la cuarta fase de la industrialización, sucediendo a la mecanización, electrificación e informatización, y destaca la integración digital de tecnologías avanzadas \cite{lasi2014industry}..
Por tanto, se centra principalmente en la digitalización y la necesaria convergencia entre los sistemas físicos y cibernéticos (\textit{Cyber-Physical Systems, CPS}). Esta integración se busca a través de nuevas tecnologías de la información y telecomunicación (TICs), como el internet de las cosas (\textit{Internet of Things, IoT}), la generación y análisis de de datos masivos (\textit{Big Data \& Big Data Analytics}), la computación en la nube (\textit{Cloud Computing}) y el auge de la Inteligencia Artificial (IA) \cite{lasi2014industry}.\cite{chen2020times}\cite{tortorella2020healthcare}


%%2. HEALTHCARE 4.0

\subsubsection{Contexto general aplicado: Sanidad 4.0, características}

La integración de los principios y tecnologías de la Industria 4.0 en el sector sanitario originó el concepto de Salud o Sanidad 4.0 (del inglés, \textit{Healthcare 4.0})\cite{tortorella2020healthcare}\cite{tortorella2021impacts}.  %
En este contexto, este nuevo término se presenta como un complejo desafío  destinado a abordar los nuevos escenarios generados por la creciente demanda de dispositivos y sistemas médicos más eficaces y alineados con las nuevas TICs y los avances ininterrumpidos en ciencias como la biotecnología y la ingeniería genética. \cite{martin2021ehealth}. La Sanidad 4.0 origina un nuevo ecosistema interseccional del que se destacan a lo largo del desarrollo completo del TFG  tres  características principales: (1) la provisión continua de cuidado sanitario, (2) la orientación de la medicina hacia el paciente y (3) la prevención y predicción de enfermedades.

%Provisión continua de cuidado sanitario - Telemedicina, salud digital

(1) La provisión continua del cuidado sanitario se basa en el cuidado continuo (\textit{continuum of care}) \cite{kouroubali2019new}. Gracias a las nuevas tecnologías de la Industria 4.0, mayoritariamente a las TICs y al IoT, la sociedad se encuentra estrechamente comunicada entre sí de forma prácticamente ininterrumpida. También a raíz de la pandemia del COVID-19 se han acelerado todas las telecomunicaciones, que en el ámbito sanitario han potenciado el desarrollo de la telemedicina y la salud digital (o \textit{e-Health} \cite{martin2021ehealth} a través del desarrollo de programas informáticos para teleconsultas, monitores de actividad mediante pulseras o relojes, nuevos implantes inteligentes... Con la digitalización y el seguimiento continuo de la salud, los dispositivos médicos que monitorizan a los pacientes en su vida cotidiana generan enormes cantidades de datos médicos de distintas índoles que, además, cada organización recoge con distintos propósitos y forma. Esto conlleva que los sistemas de salud digital frecuentemente almacenen grandísimas cantidades de datos inconsistentes, incoherentes o inaccesibles entre sí, produciéndose registros electrónicos de salud muy extensos y dispares. \cite{kouroubali2019new}. 

%Patient-centred - Modelos de datos más amplios. Medicina de precisión.

(2) La orientación de la medicina hacia el paciente se refiere a la priorización del paciente como objeto central de la provisión de salud  \cite{tortorella2020healthcare}. La atención sanitaria cada vez es más específica para cada individuo, gracias al seguimiento remoto de su actividad diaria y al auge de la medicina de precisión. Esta última constituye una nueva disciplina médica que aboga por un estudio clínico detallado que incluya aspectos como genoma, proteoma, condiciones medioambientales o rutina de vida del paciente \cite{ruiz2023inteligencia}. La posición del foco de la salud en el paciente, fomentado por la Unión Europea,  implica reestructurar el sistema sanitario alrededor del mismo, pues el paciente debe ser el cliente final, juez y recibidor de todos los servicios y aplicaciones de la salud digital \cite{ntafi2022legal} \cite{katehakis2019framework}. En términos informáticos esto implica la reconfiguración de los sistemas médicos de modo que se recoja de manera central para cada individuo su historial clínico electrónico (HCE) completo, que incluya tanto datos médicos, como farmaceúticos y otros datos de interés.  


%Preventiva y predictiva - Herramientas de big data, IA

(3) La última característica es que sea preventiva y predictiva en vez de reactiva. Esto quiere que decir, que a diferencia del enfoque tradicional en el que la medicina es meramente curativa posterior a la aparición de una enfermedad, se debe transicionar hacia la provisión de salud de manera previa a la aparición de una enfermedad de manera que esta enfermedad sea predicha, a través del análisis del HCE del paciente y/o exhaustivos análisis de precisión, y prevenida a través de monitorearización y provisión de tratamientos preventivos \cite{ruiz2023inteligencia}. En esta línea el análisis del historial clínico de un paciente genera un desafío muy complejo por las características inherentes a los datos ya comentadas, es decir, por su complejidad, desorden y extensión, de modo que las técnicas de análisis de datos tradicionales habitualmente resultan insuficientes. La prevención y la predicción se alcanza gracias al constante desarrollo de técnicas y algoritmos cada vez más sofisticos de inteligencia artificial y aprendizaje automático y herramientas cada vez más poderosas de ciencia y análisis de datos masivos.


%%3. INTEROPERABILIDAD Y ESTANDARES

\subsubsection{Pilares fundamentales: Interoperabilidad y estandarización a nivel internacional y europeo}

Estas tres características de la Sanidad 4.0 se mantienen firmes sobre un principio fundamental de creciente interés internacional compuesto por dos conceptos clave: (a) la estandarización y (b) interoperabilidad de los sistemas médicos. Ambos conceptos están relacionados entre sí mediante una relación causa-consecuencia, según el Institute of Electrical and Electronics Engineers (IEEE, 2013), "la interoperabilidad se hace posible mediante la implementación de estándares" \cite{berryman2013data}.

(a) La implementación de estándares o estandarización consiste principalmente en establecer acuerdos entre las grandes organizaciones de la salud para definir marcos específicos a través de los que estructurar los registros clínicos electrónicos de manera única, reduciendo el desorden y la disparidad de los datos y permitiendo el intercambio de mensajes entre sistemas pertenecientes a distintas organizaciones. La estandarización es un requisito fundamental para alcanzar la interoperabilidad \cite{katehakis2019framework}. Actualmente existen muchos estándares reconocidos y utilizados internacionalmente, tales como HL7 (Health Level Seven), DICOM (Digital Imaging and Communications in Medicine), SNOMED CT (Systematized Nomenclature of Medicine - Clinical Terms) o IHE (Integrating the Healthcare Enterprise). Con los estándares nace también un concepto importante: el código abierto o \textit{Open Source}. Sin ir más lejos, HL7, la mayor de las organizaciones anteriores comenzó ofreciendo sus servicios de infraestructura de datos y mensajerías de manera privada hasta 2012 cuando se decidió a promover el código abierto liberando la mayor parte de su propiedad intelectual para que pudiera ser accesible de forma gratuita, lo que potenció la adopción de estándares y la interoperabilidad entre las organizaciones sanitarias \cite{berryman2013data}.

(b) Gracias a la popularización de los estándares médicos se está desarrollando cada vez más y mejor la interoperabilidad entre los diferentes sistemas, siendo este es el objetivo final de la revolución industrial, tecnológica y sanitaria actual. A principios de siglo la Comisión Europea identificó la necesidad de interoperabilidad entre las administraciones públicas y se comenzó a desarrollar programas para respaldarla y promoverla \cite{CEU1999ida}. En 2010 se adoptó el primer Marco Europeo de Interoperabilidad (\textit{European Interoperability Framework, EIF}) que junto a los programas Soluciones de interoperabilidad para las administraciones públicas europeas (ISA y ISA\textsuperscript{2}) han asentado las bases para las estrategias actuales. En 2013 el IEEE definió el concepto de interoperabilidad como "la habilidad de los sistemas de intercambiar información y utilizar dicha información intercambiada de forma efectiva" \cite{berryman2013data}. Recientemente, en 2017 la Unión Europea adoptó el nuevo Marco de Interoperabilidad Europea (\textit{new EIF})  a través del cual ofrecer recomendaciones, modelos y guianza a fin de mejorar la calidad de los servicios públicos europeos alegando que "la falta de interoperabilidad es el mayor obstáculo para progresar" \cite{kouroubali2019new}. También, en la Comisión Europea del mismo año, se actualizó la definición de interoperabilidad como "la habilidad de las organizaciones de interactuar hacia objetivos mutamente beneficiosos, involucrando el intercambio de información y conocimiento entre dichas organizaciones a través de los procesos empresariales que soportan, es decir, del intercambio de información entre sus sistemas de información TIC" otorgando una importancia cada vez mayor al concepto \cite{katehakis2019framework}\cite{CEU2017eif} \cite{casiano2022towards}.


%%4. CAMINO HACIA OHDSI

\subsubsection{Iniciativas interoperables a nivel europeo: El camino hacia OHDSI}

En el camino hacia la interoperabilidad en el ámbito sanitario a nivel europeo, en noviembre de 2018 se lanzó la Red Europea de Datos y Evidencia en Salud (\textit{European Health Data \& Evidence Network, EHDEN}) con el objetivo de \"abordar los desafíos actuales en la generación de conocimientos y evidencia a partir de datos clínicos del mundo real a escala, para ayudar a los pacientes, médicos, pagadores, reguladores, gobiernos y la industria a comprender el bienestar, las enfermedades, los tratamientos, los resultados y Nuevas terapias y dispositivos". \cite{ehden}. En marzo de 2020 EHDEN comenzó a colaborar con la organización OHDSI (\textit{Observational Health Data Sciences and Informatics}) \cite{ohdsi}, de gran popularidad en el continente americano, para realizar estudios sobre el COVID-19. Apartir de entonces y hasta la actualidad la relación entre ambas entidades se ha estrechado enormente, adquiriendo OHDSI mayor popularidad europea y EHDEN numerosos beneficios por parte de su colaboración, hasta tal punto que el Catálogo de socios de datos del Portal EHDEN debutó en el Simposio Europeo de OHDSI en junio de 2022. 

Actualmente la presencia de OHDSI en Europa es de un interés cada vez mayor y, a nivel estatal España es uno de los nodos de colaboración con OHDSI más grandes de Europa, y muchas organizaciones a lo largo del territorio español ya están colaborando con sus estándares como la Agencia Española de Medicamentos y Productos Sanitarios (AEMPS) o Quirónsalud entre otros \cite{ohdsiSpain}. En Sevilla, la colaboración con OHDSI la llevan a cabo los hospitales universitarios Virgen Macarena y Virgen del Rocío, siendo este último sede del estudio práctico que ha acompañado al desarrollo de este TFG, concluyendo así el marco contextual del trabajo.


%- Hablar de OHDSI EN SEVILLA
%  [Innodata2023]


%%---------------------------------------------------------------------------

\section{Motivación}

%Mi motivación personal de entrar en el mundo del %análisis de datos clínicos utilizando  esta %herramienta prometedora..

%RECUERDOS PRINCIPIO DE CARRERA BIG DATA, POLITECNICO DI MILANO...

Personalmente, podría afirmar que mi curiosidad e interés por el mundo de la ciencia de datos ha sido constante a lo largo de mis cuatro años de estudios ingenieriles. Un pensamiento persistente que me ha guiado en este trayecto ha sido el recuerdo de una clase de estadística del primer año de carrera, en el 2020, donde por primera vez escuché hablar sobre el Big Data como una disciplina emergente que el profesor nos invitó a considerar como una posible salida laboral futura. Este fue mi primer contacto con estas ciencias que despertaron en mi una gran curiosidad, aunque fue al segundo contacto cuando finalmente me decidí a estudiarlas.
En el tercer año de carrera, ya pensando en orientar mis futuros estudios de máster o especializaciones en esta temática, aproveché la oportundiad que me brindó haber sido seleccionada para realizar el programa de movilidad Erasmus en el Politecnico di Milano para tener un contacto más directo con la disciplina. Ser estudiante erasmus me otorgó la flexibilidad de realizar mi propio convenio de estudios, brindándome así la oportunidad de escoger entre asignaturas pertenecientes al master de de Data Science que ofrecía el Politecnico. Después de un año, a mi regreso del Erasmus, pude afirmar con convicción que quería orientar mi carrera profesional hacia el mundo del análisis de datos, especialmente en el ámbito clínico, que siempre ha sido también una de mis pasiones.

Estas curiosidades convertidas en pasiones han sido los principales motores que me han motivado para realizar este Trabajo Fin de Grado manteniendo la alegría de quien verdaderamente disfruta lo que estudia. 






    






