\chapter{Objetivos del Proyecto}\label{cap:02objetivos}

En este capítulo se presentan los objetivos del Trabajo Fin de Grado, consensuados por el alumno, los tutores de la Universidad de Sevilla y los del Hospital Universitario Virgen del Rocío. Se presentan en \ref{sec:02objTFG} los objetivos generales para el desarrollo del TFG y en \ref{sec:02objPersonal} los objetivos personales del alumno.

\section{Objetivos del TFG} \label{sec:02objTFG}

Los objetivos relativos al desarrollo teórico y práctico del TFG son tres y se presentan a continuación:

%\begin{table}[H]
    \resizebox{\columnwidth}{!}{%
    \centering
    \begin{tabular}{|c|c|c|}
    \hline
    \textbf{ID} & \textbf{Descripción} \\
    \hline
    Obj-001 & Instalación, configuración y despliegue de ATLAS Broadsea \\
    Obj-002 & Estudio teórico minucioso de funcionalidades y arquitectura de OHDSI y ATLAS Broadsea \\
    Obj-003 & Estudio de caso práctico de análisis de datos clínicos proporcionados por el hospital\\
    \hline
    \end{tabular}
    }
\caption{Objetivos del Trabajo Fin de Grado}
\label{tab:objetivosTFG}
\end{table}

\begin{enumerate}

    \item \textbf{Obj-001: Estudio teórico de organización OHDSI y herramienta ATLAS Broadsea.} Este objetivo proporciona al alumno un marco de fundamentación y comprensión necesario para poder extraer verdadero valor del uso de ATLAS y de todo el ecosistema de la comunidad científica de OHDSI.

    \item \textbf{Obj-002: Instalación, configuración y despliegue de ATLAS Broadsea.} Este objetivo, acompañado de la redacción del Anexo A, es de importancia trascendental, puesto que el anexo reúne en un único documento inédito información dificilmente accesible y desperdigada en la red, constituyendo un documento de gran relevancia para toda la comunidad científica, especialmente para el equipo del Hospital, que contará con mayor facilidad a la hora de realizar estas tareas sobre Broadsea.

    \item \textbf{Obj-003: Reproducción de caso práctico de análisis de datos clínicos proporcionados por el hospital.} Este objetivo está ligado en igual medida al TFG y a las prácticas realizadas en el Hospital, pues consiste en reproducir un estudio ya realizado previamente sobre unos datos proporcionados por el HUVR pero utilizando, en este caso, ATLAS. La colaboración con el hospital en este caso es crucial para el alcance de este objetivo que de forma práctica complementa a la documentación teórica del TFG.

\end{enumerate}

\section{Objetivos Personales} \label{sec:02objPersonal}

Los objetivos personales, relativos a la ambición, interés y curiosidad de la alumna son tres y se presentan a continuación:

%%Aumentar mi conocimiento del estándar de OHDSI
%Aumentar mi experiencia en el manejo de datos clínicos
%Aumentar mi conocimeinto en el mundo de análisis de datos

\begin{table}[H]
    \resizebox{\columnwidth}{!}{%
    \centering
    \begin{tabular}{|c|c|c|}
    \hline
    \textbf{ID} & \textbf{Descripción} \\
    \hline
    Obj-Pers-001 & Aumentar mi conocimiento del estándar OHDSI y sus herramientas\\
    Obj-Pers-002 & Aumentar mi conocimiento del mundo del análisis de datos \\
    Obj-Pers-003 & Aumentar mi experiencia en el mundo del análisis de datos \\
    \hline
    \end{tabular}
    }
\caption{Objetivos personales del alumno}
\label{tab:objetivosAlum}
\end{table}

\begin{enumerate}

    \item \textbf{Obj-Pers-001: Aumentar mi conocimiento de la comunidad OHDSI y sus herramientas.} Este objetivo se debe a que inicialmente mi desconocimiento sobre  OHDSI era absoluto. Por tanto, aumentar mi conocimiento sobre la organización es importante para comprender la utilidad de la misma y de las herramientas que proporciona y poder realizar un trabajo coherente y bien fundamentado.

    \item \textbf{Obj-Pers-002: Aumentar mi conocimiento del mundo del análisis de datos.} Este objetivo se debe a que, aunque es cierto que durante mis estudios de grado he aprendido y obtenido grandes conocimientos sobre las ciencias de datos, de este trabajo final también se espera aumentar en mayor profundidad los conocimientos teóricos, generales y específicos a una herramienta de gran interés europeo como es ATLAS para el análisis de datos.

    \item \textbf{Obj-Pers-003: Aumentar mi experiencia laboral analizando datos clínicos.} Este objetivo pretende aumentar la experiencia de analizar datos clínicos fuera del marco meramente académico, sino en un entorno de trabajo real, con datos clíncos reales, gracias a la colaboración con el grupo de Innovación Tecnológica del Hospital.
    
\end{enumerate}



