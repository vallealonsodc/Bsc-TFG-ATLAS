\chapter{Objetivos del proyecto}\label{cap:02objetivos}

En este capítulo se presentan los objetivos del Trabajo Fin de Grado, consensuados por el alumno, los tutores de la Universidad de Sevilla y los del Hospital Universitario Virgen del Rocío. Se presentan en \ref{sec:02objTFG} los objetivos generales para el desarrollo del TFG, en \ref{sec:02objPersonal} los objetivos personales del alumno y en \ref{sec:02trazabilidad} la trazabilidad de los objetivos definidos.

\section{Objetivos del TFG} \label{sec:02objTFG}

Los objetivos relativos al desarrollo teórico y práctico del TFG son tres y se presentan a continuación en la siguiente tabla:

\begin{table}[H]
    \resizebox{\columnwidth}{!}{%
    \centering
    \begin{tabular}{|c|c|c|}
    \hline
    \textbf{ID} & \textbf{Descripción} \\
    \hline
    Obj-001 & Instalación, configuración y despliegue de ATLAS Broadsea \\
    Obj-002 & Estudio teórico minucioso de funcionalidades y arquitectura de OHDSI y ATLAS Broadsea \\
    Obj-003 & Estudio de caso práctico de análisis de datos clínicos proporcionados por el hospital\\
    \hline
    \end{tabular}
    }
\caption{Objetivos del Trabajo Fin de Grado}
\label{tab:objetivosTFG}
\end{table}

El Obj-001 consiste en el ''Estudio teórico minucioso de funcionalidades y arquitectura de OHDSI y ATLAS Broadsea''. Este objetivo proporciona al alumno un marco de fundamentación y comprensión necesario para poder extraer verdadero valor del uso de ATLAS y de todo el ecosistema de la comunidad científica de OHDSI.

El Obj-002 consiste en la ''Instalación, configuración y despliegue de ATLAS Broadsea'' y la redacción de toda la documentación relativa al proceso en el Anexo A del TFG. Este objetivo es de importancia trascendental, puesto que el anexo reúne en un único documento inédito información dificilmente accesible y desperdigada en la red, constituyendo un documento de gran relevancia para toda la comunidad científica, especialmente para el equipo del Hospital, que contará con mayor facilidad a la hora de realizar estas tareas sobre Broadsea.

El Obj-003 consiste en el ''Estudio de caso práctico de análisis de datos clínicos proporcionados por el hospital''. Este objetivo está ligado en igual medida al TFG y a las prácticas realizadas en el Hospital, pues consiste en \textcolor{red}{replicar un estudio ya realizado previamente sobre unos datos proporcionados por el HUVR} pero utilizando, en este caso, las herramientas OHDSI. La colaboración con el hospital en este caso es crucial para el alcance de este objetivo que de forma práctica complementa a la documentación teórica del TFG.


\section{Objetivos Personales} \label{sec:02objPersonal}

Los objetivos personales, relativos a la ambición, interés y curiosidad de la alumna son tres y se presentan a continuación en la siguiente tabla:

%Aumentar mi conocimiento del estándar de OHDSI
%Aumentar mi experiencia en el manejo de datos clínicos
%Aumentar mi conocimeinto en el mundo de análisis de datos

\begin{table}[H]
    \resizebox{\columnwidth}{!}{%
    \centering
    \begin{tabular}{|c|c|c|}
    \hline
    \textbf{ID} & \textbf{Descripción} \\
    \hline
    Obj-Pers-001 & Aumentar mi conocimiento del estándar OHDSI y sus herramientas\\
    Obj-Pers-002 & Aumentar mi conocimiento del mundo del análisis de datos \\
    Obj-Pers-003 & Aumentar mi experiencia en el mundo del análisis de dato \\
    \hline
    \end{tabular}
    }
\caption{Objetivos personales del alumno}
\label{tab:objetivosAlum}
\end{table}

El Obj-Pers-001 consiste en ''Aumentar mi conocimiento del estándar OHDSI y sus herramientas'', pues en origen mi conocimiento sobre la comunidad científica de OHDSI era nulo, y a medida que iba investigando descubría nuevas iniciativas, redes colaborativas y nuevas herramientas de gran utilidad que despertaron un creciente interés sobre la organización, además de la necesaria recopilación de información para estructurar un trabajo coherente y bien fundamentado.

El Obj-Pers-002 consiste en ''Aumentar mi conocimiento del mundo del análisis de datos'', pues si bien durante mis estudios de grado he aprendido y obtenido grandes conocimientos sobre este sector de las ciencias de datos, el conocimiento nunca sobra, por lo que de este trabajo también se espera aumentar en mayor profundidad los conocimientos teóricos, generales y específicos a ATLAS sobre análisis de datos.

El Obj-Pers-003 consiste en ''Aumentar mi experiencia en el mundo del análisis de datos'', pues si bien también durante mis estudios de grado he interaccionado de forma experimental con este sector de las ciencias de grado, la experiencia nunca sobra, por lo que gracias a la realización de la parte práctica de este trabajo, en colaboración con el grupo de Innovación Tecnológica del Hospital, también se espera adquirir experiencia real en el tratamiento de datos clínicos, especificamente usando ATLAS.

\section{Trazabilidad de Objetivos} \label{sec:02trazabilidad}

Los objetivos estipulados se han cumplido al término del desarrollo del Trabajo de Fin de Grado, permitiendo elaborar las siguientes tablas o matrices de trazabilidad para cada objetivo. Cada tabla incluye los capítulos del TFG donde se alcanzan los objetivos (según la numeración del índice dle trabajo) y el tiempo en horas invertido a cada uno (extraído del estudio exhaustivo en el capítulo \ref{cap:03gestión}):

\begin{table}[H]
    %\resizebox{\columnwidth}{!}
\caption{Trazabilidad de objetivos del Trabajo Fin de Grado}
\label{tab:trazabilidadTFG}
\end{table}

El Obj-001 se desarrolla y alcanza a través del Anexo A del TFG ''Manual de instalación, despliegue y configuración de ATLAS Broadsea'', para el que se invierte en total {horas} 

El Obj-002 se desarrolla y alcanza a través de los capítulos 5 ''Marco teórico específico'', 6 ''Documento de requisitos'' y 7 ''Documento de Análisis y Diseño'' y el Anexo A del TFG ''Manual de instalación, despliegue y configuración de ATLAS Broadsea'', para los cuales se invierte {horas}, {horas}, {horas}, respectivamente, sumando un total de {horas}.

El Obj-003 se desarrolla y alcanza a través de el capítulo 8 ''Plan de pruebas'', para el cuál se invierte en total {horas}.

\begin{table}[H]
    %\resizebox{\columnwidth}{!}
\caption{Trazabilidad de objetivos personales de la alumna}
\label{tab:trazabilidadAlum}
\end{table}

El Obj-Pers-001  se desarrolla y alcanza a través de los capítulos 1 ''Introducción, Contexto y Motivación'' y 5 ''Estudio Previo'' y el Anexo A del TFG ''Manual de instalación, despliegue y configuración de ATLAS Broadsea'', para los cuales se invierte {horas}, {horas}, {horas}, respectivamente, sumando un total de {horas}.

El Obj-Pers-002 se desarrolla y alcanza a través de los capítulos 1 ''Introducción, Contexto y Motivación'', 5 ''Estudio Previo'',  6 ''Documento de requisitos'', 9 ''Resultados'' y 10 ''Conclusiones'', para los cuales se invierte {horas}, {horas}, {horas}, respectivamente, sumando un total de {horas}.

El Obj-Pers-003 se desarrolla y alcanza a través de los capítulos  8 ''Plan de pruebas'', 9 ''Resultados'' y 10 ''Conclusiones'', para los cuales se invierte en total {horas}, {horas}, {horas}, respectivamente, sumando un total de {horas}.
