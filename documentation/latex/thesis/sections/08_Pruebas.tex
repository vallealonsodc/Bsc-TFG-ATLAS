\chapter{Caso práctico}\label{cap:08pruebas}

\section{Introducción}

Se va a reproducir un estudio oncológico realizado por los investigadores del HUVR pero utilizando herramienta ATLAS

\section{Estudio del HUVR}

Presentación del estudio oncológico realizado por los investigadores del HUVR  (no utiliza ATLAS)

- Dataset utilizado

- Metodología si fuese relevante

- Objetivos del estudio

- Resultados

%\subsection{Comprobación calidad datos}

%- Datos omopizados por TFG Paco 
%- Calidad previa y post comprobada tb en TFG Paco

\section{Reproducción del estudio}
%Caso práctico del TFG

%Check de los casos de uso/requisitos en el estudio real.

\subsection{Datos}

Se han obtenido el dataset que se utilizó en ese estudio.

- Los datos se convirtieron a OMOP (TFG Paco)

- Se analizó la calidad de los datos eficiente (TFG Paco)

\subsection{Metodología}

\subsubsection{Reporte del dataset}

\subsubsection{Definición de la cohorte}

Necesario para utilzar ATLAS

Cohorte = Personas con cancer de pulmon

\subsubsection{Caracterización de la cohorte}

Para conocer los pacientes que tenemos en el cohorte

\subsubsection{Estimación a nivel de población}

Hacer un estudio sobre los efectos adversos que sufrirá la población del cohorte

\subsubsection{Predicción a nivel de Paciente}

Para un paci

\subsection{Resultados}


\section{Comparación de resultados}

Comparación de los resultados obtenidos en el estudio del HUVR y el estudio realizado con ATLAS


\section{Conclusiones}

En este capítulo concluimos que...