\chapter{Introducción, conceptos previos y Motivación}\label{cap:introduccion}

Este primer capítulo del Trabajo Fin de Grado (TFG) se divide en cuatro secciones: se presenta en \ref{sec:01Intro} una introducción descriptiva de los contenidos del documento, en \ref{sec:01Contexto} el contexto teórico general, en \ref{sec:01EstadoArte} el estado del arte actual y en \ref{sec:01Motivacion} la motivación personal que trasciende a la realización del trabajo.

\section{Introducción} \label{sec:01Intro}

Este Trabajo Fin de Grado ha sido realizado por María del Valle Alonso de Caso Ortiz, alumna del grado de Ingeniería de la Salud por la Universidad de Sevilla (US), de la promoción 2020-2024 y bajo la tutela de D. Julián A. García García y Da. Maria J. Escalona Cuaresma, ambos pertenecientes al departamento de Lenguajes y Sistemas Informáticos de la Escuela Técnica Superior de Ingeniería Informática (ETSII) de la misma universidad. Además se realiza en conjunto con el Departamento de Innovación Tecnológica del Hospital Universitario Virgen del Rocío, mediante un convenio de prácticas curriculares de 337 horas, donde han ejercido la tutela D. Silvia Rodríguez Mejías y D. Carlos Luis Parra Calderón. Toda la información técnica relativa al desarrollo del TFG como los Objetivos del proyecto, la Gestión del proyecto y la Metodología empleada se presentan en los capítulos \ref{cap:02objetivos}, \ref{cap:03gestión} y \ref{cap:04metodologia}, respectivamente.

Por otra parte, el trabajo abarca una perspectiva teórica muy amplia, de la que se presenta, en primer lugar, en la secciones \ref{sec:01Contexto} y \ref{sec:01EstadoArte} el paradigma tecnológico y sanitario actual, con sus características, desafíos principales e iniciativas reales que se están llevando a cabo actualmente. Posteriormente, en el capítulo \ref{cap:05EstudioPrevio} se realiza una exposición teórica profunda y extensa sobre las herramientas y estándares de la organización Observational Health Data Sciences and Informatics (OHDSI), que es el foco central del trabajo, por lo que se pretende que el lector conozca en profundidad los aspectos importantes de la misma.

Una vez se presentan los conocimientos teóricos generales sobre la organización OHDSI y las herramientas más relevantes que ofrece, el trabajo se especializa en el uso de la herramienta ATLAS en \textcolor{red}{la reproducción de un estudio de datos clínicos con la misma}. Concretamente, la herramienta se despliega e implementa a través de la iniciativa \textit{Broadsea}, que origina la denominación \textit{ATLAS Broadsea} de la herramienta a lo largo del TFG. La información teórica sobre la herramienta y el sistema se presenta en \ref{cap:05EstudioPrevio}, \ref{cap:06requisitos} y \ref{cap:07diseño}.

En tercer lugar, debido a la naturaleza práctica del TFG, por haber sido desarrollado en colaboración con el HUVR, se realiza \textcolor{red}{una reproducción de un estudio realizado por los investiagdores del hospital el estudio es...} (véase \ref{cap:08pruebas}). Además, se desarrolla el Anexo A \ref{anexo:manual} como un documento de gran extensión y relevancia que describe en profundidad el proceso de instalación, despliegue y configuración de ATLAS Broadsea. Este Anexo es de gran interés porque no existe en internet ninguna guía completa de estas características sobre la herramienta, aportando un contenido de gran valor a la comunidad científica y a mis compañeros del departamento de Innovación Tecnológica del hospital.

Por último, los últimos dos capítulos \ref{cap:09resultados} y \ref{cap:10conclusiones} presentan una recopilación de resultados y conclusiones, respectivamente, obtenidos al término del desarrollo del TFG. También se adjunta el Anexo B, que consiste en un Glosario de Términos técnicos relevantes para la comprensión del trabajo. 

Adicionalmente, por su naturaleza informática, este TFG se ha desarrollado paralelamente a un repositorio de github del proyecto \cite{vallealonsodc}, que ha servido como controlador de versiones y como administrador de archivos en la nube, permitiendo almacenar y compartir con el lector archivos relevantes del TFG, ya sean archivos necesarios para el despliegue de la herramienta, archivos producidos durante el análisis o los propios documentos en sí mismos.

\section{Contexto} \label{sec:01Contexto} 
%Intención: dar a conocer al lector los aspectos y características fundamentales del panorama sanitario y tecnológico actual, necesario para comprender en profundidad el trabajo.

En esta sección de contexto se presentan los aspectos y características fundamentales del panorama tecnológico-sanitario emergente. Se pretende que el lector conozca los conceptos teóricos que le permitan comprender las necesidades actuales que subyacen al Trabajo Fin de Grado.
 
%El contexto en el que se desarrolla el TFG se encuentra notablemente influido por el surgimiento de la Industria 4.0 y las nuevas tecnologías que la acompañan, así como por su impacto significativo y transformador en el sector sanitario que ha generado nuevas necesidades a nivel mundial y europeo, más concretamente en el tratamiento de los datos de salud. %que ha desembocado en Sevilla para ser el tópico trascendental de este trabajo.

En primer lugar se presenta el origen de la Industria 4.0 y su influencia en el sector sanitario, denominado Sanidad o Salud 4.0. 
%se realiza a continuación un recorrido por los conceptos y características más relevantes del panorama tecnológico-sanitario emergente, denominado Sanidad 4.0 (de \textit{Industria 4.0}). 
Posteriormente se describen tres características principales de la Salud 4.0 que son el motor del cambio de paradigma y se destacan dos necesidades fundamentales: la interoperabilidad y la estandarización de las tecnologías sanitarias, especialmente en el tratamiento de los datos de salud. Finalmente se presentan algunos de los desafíos actuales de esta disciplina.

%Frente a ellas se presentan desafíos y soluciones ampliamente aceptadas a nivel global, destacando en particular la creciente importancia de la organización \textit{OHDSI}, que constituye el foco central del trabajo.
\subsubsection{Introducción: Industria 4.0}

La Industria 4.0, o cuarta revolución industrial, fue un concepto concebido por el gobierno alemán en noviembre de 2011 como una estrategia tecnológica para abordar el crecimiento industrial proyectado para 2020. Su uso internacional se popularizó en abril de 2013 durante la feria industrial de Hannover \textit{Hannover Messe}). Este concepto representa la cuarta fase de la industrialización, sucediendo a la mecanización, electrificación e informatización, y destaca la integración digital de tecnologías avanzadas \cite{lasi2014industry}.
Se centra principalmente en la digitalización y la necesaria convergencia entre los sistemas físicos y cibernéticos (\textit{Cyber-Physical Systems, CPS}). Esta integración se busca a través de nuevas tecnologías de la información y telecomunicación (TICs), como el internet de las cosas (\textit{Internet of Things, IoT}), la generación y análisis de datos masivos (\textit{Big Data \& Big Data Analytics}), la computación en la nube (\textit{Cloud Computing}) y el auge de la Inteligencia Artificial (IA) \cite{lasi2014industry}\cite{chen2020times}\cite{tortorella2020healthcare}

\subsubsection{Características de la Sanidad 4.0}

La integración de los principios y tecnologías de la Industria 4.0 en el sector sanitario originó el concepto de Salud o Sanidad 4.0 (del inglés, \textit{Healthcare 4.0})\cite{tortorella2020healthcare}\cite{tortorella2021impacts}.  %
En este contexto, este nuevo término se presenta como un complejo desafío  destinado a abordar los nuevos escenarios generados por la creciente demanda de dispositivos y sistemas médicos más eficaces y alineados con las nuevas TICs y los avances ininterrumpidos en ciencias como la biotecnología y la ingeniería genética. \cite{martin2021ehealth}. 

La Sanidad 4.0 origina un nuevo ecosistema interseccional del que se destacan tres  características principales: (1) la provisión continua de cuidado sanitario, (2) la orientación de la medicina hacia el paciente y (3) la prevención y predicción de enfermedades.


%Provisión continua de cuidado sanitario - Telemedicina, salud digital

\begin{enumerate}

\item \textbf{Provisión continua de cuidado sanitario}. La provisión continua del cuidado sanitario se basa en el cuidado continuo (\textit{continuum of care}) \cite{kouroubali2019new}. Gracias a las nuevas tecnologías de la Industria 4.0, mayoritariamente a las TICs y al IoT, la sociedad se encuentra estrechamente comunicada de forma prácticamente ininterrumpida. En el ámbito sanitario, a raíz de la pandemia del COVID-19 ha habido un creciente auge en el desarrollo de la telemedicina y la salud digital (\textit{e-Health}) \cite{martin2021ehealth} con el fin de monitorear al paciente dentro y fuera del complejo hospitalario. Para ello se ha potenciado el desarrollo de programas informáticos y dispositivos portátiles o \textit{wearables} de monitoreo de actividad como pulseras, relojes, sensores corporales, implantes inteligentes... Estos dispositivos generan enormes cantidades de datos médicos que junto a los registros clínicos de los hospitales generan una amplísima variedad de datos de pacientes reales. A estos conjuntos de datos se les denomina 'Datos del mundo real' (\textit{Real World Data}) y presentan un gran desafío actual en el tratamiento de los datos clínicos debido a su naturaleza de distintas índoles y que, además, cada organización recoge con distintos propósitos y estructura, lo que conlleva que frecuentemente se presenten grandísimas cantidades de datos inconsistentes, incoherentes o inaccesibles entre sí, produciéndose registros electrónicos de salud muy extensos y dispares. \cite{kouroubali2019new}.

%Patient-centred - Modelos de datos más amplios. Medicina de precisión.
\item \textbf{Medicina centrada en el paciente}. La orientación de la medicina hacia el paciente se refiere a la priorización del paciente como objeto central de la provisión de salud  \cite{tortorella2020healthcare}. La atención sanitaria cada vez es más específica para cada individuo, gracias al seguimiento remoto de su actividad diaria y al auge de la medicina de precisión \cite{ruiz2023inteligencia}. La posición del foco de la salud en el paciente, fomentado por la Unión Europea,  implica reestructurar el sistema sanitario alrededor del mismo, pues el paciente debe ser el cliente final, juez y recibidor de todos los servicios y aplicaciones de la salud digital \cite{ntafi2022legal} \cite{katehakis2019framework}. En términos informáticos esto implica la reconfiguración de los sistemas médicos de modo que se recoja de manera central para cada individuo su historial clínico electrónico (HCE) completo, que incluya tanto datos médicos, como farmaceúticos y otros datos de interés.  

 %privacidad de datos, espacio de datos internacional, consentimiento en el uso secundario, acceso a los hce...

%Preventiva y predictiva - Herramientas de big data, IA
\item \textbf{Preventiva y predictiva}. La última característica es que sea preventiva y predictiva en lugar de meramente reactiva. Esto quiere que decir, que a diferencia del enfoque tradicional en el que la medicina es curativa, se debe transicionar hacia la provisión de salud de manera previa a la aparición de una enfermedad, de modo que esta pueda ser (i) predecida a través del análisis del HCE del paciente y/o exhaustivos análisis de precisión, y (ii) prevenida a través de monitorearización y provisión de tratamientos preventivos en el cuidado continuo de la salud \cite{ruiz2023inteligencia}. En esta línea el análisis del historial clínico de un paciente genera un desafío muy complejo y la prevención y la predicción se alcanza gracias al constante desarrollo de técnicas y algoritmos cada vez más sofisticados de inteligencia artificial y aprendizaje automático y herramientas cada vez más poderosas de ciencia y análisis de datos.
\end{enumerate}

\subsubsection{Interoperabilidad}

La Sanidad 4.0 se edifica sobre un principio fundamental de creciente interés internacional: la interoperabilidad de los sistemas médicos.
% Ambos conceptos están relacionados entre sí mediante una relación causa-consecuencia, según el Institute of Electrical and Electronics Engineers (IEEE, 2013), "la interoperabilidad se hace posible mediante la implementación de estándares" \cite{berryman2013data}.


\begin{enumerate}[label=\alph*.]

    \item \textbf{Estandarización}. La implementación de estándares o estandarización consiste principalmente en establecer acuerdos entre las grandes organizaciones de la salud para definir marcos específicos a través de los que estructurar los registros clínicos electrónicos de manera única, reduciendo el desorden y la disparidad de los datos y permitiendo el intercambio de mensajes entre sistemas pertenecientes a distintas organizaciones. La estandarización es un requisito fundamental para alcanzar la interoperabilidad \cite{katehakis2019framework}. Actualmente existen muchos estándares reconocidos y utilizados internacionalmente, tales como HL7 (Health Level Seven), DICOM (Digital Imaging and Communications in Medicine), SNOMED CT (Systematized Nomenclature of Medicine - Clinical Terms) o IHE (Integrating the Healthcare Enterprise). Con los estándares nace también un concepto importante: el código abierto o \textit{Open Source}. Sin ir más lejos, HL7, la mayor de las organizaciones entre las anteriores comenzó ofreciendo sus servicios de manera privada hasta 2012, cuando se decidió a promover el código abierto liberando la mayor parte de su propiedad intelectual para que pudiera ser accesible de forma gratuita. Esto potenció y promovió la adopción de estándares y la consecuente interoperabilidad entre las organizaciones sanitarias \cite{berryman2013data}.

    \item  \textbf{Interoperabilidad}. La interoperabilidad entre sistemas y datos es el objetivo final de la revolución industrial, tecnológica y sanitaria actual. La necesidad de interoperabilidad es una realidad a la que se enfrentan todos los sectores y sistemas de información de las organizaciones públicas y privadas. La Comisión Europe identificó esta necesidad ya a principios de siglo \cite{CEU1999ida} y durante los años ha ido adquiriendo cada vez mayor relevancia. En 2013, el IEEE definió el concepto de interoperabilidad como 
    "la habilidad de los sistemas de intercambiar información y utilizar dicha información intercambiada de forma efectiva" (IEE, 2013). Actualmente, el nuevo Marco de Interoperabilidad Europea (\textit{new EIF}) es el órgano encargado de ofrecer recomendaciones, modelos y guianza a fin de mejorar la calidad de los servicios públicos europeos en cuanto a interopeabilidad, pues se dice que "la falta de interoperabilidad es el mayor obstáculo para progresar" \cite{kouroubali2019new}.
    
    %en la administración ya había sido identificada desde principios de siglo por la Comisión Europea \cite{CEU1999ida} aunque no fue hasta 2010 que verdaderamante comenzaron las iniciativas para poner en práctica estrategias interoperables. Este mismo año se adoptó el primer Marco Europeo de Interoperabilidad (\textit{European Interoperability Framework, EIF}) junto a los programas Soluciones de interoperabilidad para las administraciones públicas europeas (ISA y ISA\textsuperscript{2}) y tres años más tarde, en 2013,  el IEEE definió rigurosamente el concepto como "la habilidad de los sistemas de intercambiar información y utilizar dicha información intercambiada de forma efectiva" \cite{berryman2013data}. 
    
    %Recientemente, en 2017 la Unión Europea adoptó el nuevo Marco de Interoperabilidad Europea (\textit{new EIF})  a través del cual ofrecer recomendaciones, modelos y guianza a fin de mejorar la calidad de los servicios públicos europeos alegando que "la falta de interoperabilidad es el mayor obstáculo para progresar" \cite{kouroubali2019new}. También, en la Comisión Europea del mismo año, se actualizó la definición de interoperabilidad como "la habilidad de las organizaciones de interactuar hacia objetivos mutamente beneficiosos, involucrando el intercambio de información y conocimiento entre dichas organizaciones a través de los procesos empresariales que soportan" otorgando una importancia cada vez mayor al concepto \cite{katehakis2019framework}\cite{CEU2017eif} \cite{casiano2022towards}. 

\end{enumerate}

\subsubsection{Desafíos en el tratamiento de los datos}

No obstante, aún con tantas iniciativas a nivel global y europeo, la transición hacia la interoperabilidad y la estandarización sigue siendo muy dificultosa, debido a la gran complejidad y sensibilidad de los sistemas de información en salud. El tratamiento de datos sanitarios requiere de gestiones muy precisas, con protocolos de ciberseguridad muy estrictos y leyes sobre pivacidad y confidencialidad muy bien definidas, que dificultan la implementación coordinada en diferentes regiones. A continuación se presentan algunos de los desafíos en el tratamiento de los datos clínicos expuestos en el Foro de Seguridad y Protección de Datos organizado por la SEIS en 2024 \cite{SEIS2024tercera} \cite{SEIS2024octava}:

\begin{enumerate}[label=\roman*.]
    \item \textbf{Ciberseguridad del sistema}. La ciberseguridad de los datos clínicos representa un desafío crítico. El creciente auge de amenazas cibernéticas constantes, requiere de actualizaciones y mejoras en las medidas de protección de la información médica. Las instituciones de salud deben estar a la vanguardia en la implementación de tecnologías de seguridad robustas para salvaguardar la integridad y la confidencialidad de los datos.
    \item \textbf{Confidencialidad y privacidad}.   La confidencialidad y privacidad de los datos clínicos también conforma un desafío relevante. Garantizar que solo las partes autorizadas tengan acceso a la información médica de los pacientes requiere no solo de protocolos tecnológicos sólidos, sino también de una cultura organizacional comprometida con el cumplimiento de las regulaciones de protección de datos y la ética médica. Para ello, además se necesitan protocolos de anonimización y pseudoanonimización de las bases de datos, que garanticen la privacidad de la información personal de los pacientes.
    \item \textbf{El uso secundario}. El consentimiento para el uso secundario de datos clínicos es otro aspecto crucial a considerar. A medida que se exploran nuevas formas de aprovechar los datos para la investigación y la mejora de la atención médica, es fundamental asegurar que los pacientes comprendan y otorguen su consentimiento informado para cualquier uso adicional de su información médica, respetando siempre su autonomía y derechos individuales.
    \item \textbf{Infraestructura tecnológica}. Por último, la infraestructura tecnológica adecuada es un requisito fundamental para el manejo eficiente de los datos clínicos. La arquitectura de los datos cada vez es más compleja y requiere infraestructuras tecnológicas muy potentes y costosas. Además, la falta de interoperabilidad entre sistemas, la obsolescencia de la tecnología y las limitaciones presupuestarias pueden obstaculizar los esfuerzos para la prestación de servicios TIC de salud.
    
\end{enumerate}

Todos estos desafíos son los puntos débiles de la actual Sanidad 4.0 pero también son los puntos de mayor atención, pues incidiendo se forma especial en ellos se podrá alcanzar una solución global que facilite la provisión de salud y el aprovechamiento de la información clínica.

\section{Estado del Arte} \label{sec:01EstadoArte} 
%Intención: cuáles son las inicitivas que hay actualmente en el sector y cuál es la presencia real de OHDSI en el mismo.

En esta sección se presentan las iniciativas que se están llevando a cabo más recientemente para afrontar los desafíos en el análisis de datos clínicos a nivel global, europeo y nacional. La intención es que el lector se familiarice con iniciativas reales actuales.

Actualmente a nivel global, la interoperabilidad es el foco de atención de las grandes potencias y compañías del sector sanitario. Algunas propuestas ejemplares para favorecer su desarrollo se llevan a cabo en, en Estados Unidos, el IEEE ha desempeñado un papel crucial en el desarrollo de estándares para la interoperabilidad de datos en salud, con iniciativas como el estándar IEEE 11073 para dispositivos médicos interoperables. Además, el National Institutes of Health (NIH) y la organización HL7 lideran esfuerzos para promover la colaboración y el intercambio de datos. Otra organización de gran popularidad y presencia en Estados Unidos es OHDSI, que además que se está convirtiendo en un referente a nivel global en el campo de la ciencia de datos en salud.

En China, otra de las grandes potencias, el gobierno ha lanzado múltiples programas y proyectos para mejorar la interoperabilidad de los datos de salud, como el China Health Information Interoperability Project (CHIIP), que busca establecer estándares y protocolos para la integración de datos de salud en todo el país.

%A nivel global, las organizaciones con mayor relevancia en el mundo de la ciencia de datos en salud  son HL7 y OHDSI entre otros grandes estándares ya mencionados como LOINC, SNOMED, el NIH.

En Europa, en noviembre de 2018 se lanzó la Red Europea de Datos y Evidencia en Salud (\textit{European Health Data \& Evidence Network, EHDEN}) con el objetivo de ''abordar los desafíos actuales en la generación de conocimientos y evidencia a partir de datos clínicos del mundo real a escala, para ayudar a los pacientes, médicos, pagadores, reguladores, gobiernos y la industria'' \cite{ehden}.

Desde marzo de 2020 EHDEN colabora con OHDSI para proporcionar un espacio de datos interoperables y estandarizados. La colaboración comenzó con el fin de realizar estudios sobre Covid-19 aunque su relación se mantiene en la actualidad, ejemplo de ello fue la participación de los socios de EHDEN en el Simposio Europeo de OHDSI en junio de 2022. Ese mismo año OHDSI mostraba también su interés en el proyecto, que dió comienzo en 2021, DARWIN EU (\textit{Data Analysis and Real World Interrogation Network European Unión}) \cite{OHDSI2023Darwin} para proporcionar evidencia del mundo real de toda Europa sobre enfermedades, poblaciones y los usos y rendimiento de medicamentos \cite{Darwin2023website}. Otra iniciativa más reciente, comenzada en 2023, es el proyecto de EUCAIM (\textit{Cancer Image Europe}) que pretende establecer una red federada interoperable de compartición de imágenes oncológicas. Para la selección del estandar que debe seguir la federación se está considerando la participación de HL7 FHIR o de OHDSI \cite{Kalokyri2023Early}.

Por otro lado, la Infraestructura de Servicios Digitales de eSalud (eHDSI) \cite{DHE2023eHDSI} representa un hito crucial en el impulso de la interoperabilidad y la integración de los sistemas de información sanitaria en Europa. Este marco establece estándares y protocolos para facilitar el intercambio seguro y eficiente de datos de salud entre los Estados miembros de la Unión Europea, con el objetivo de mejorar la calidad de la atención médica y promover la movilidad de los pacientes en el espacio europeo de salud digital \cite{EU2023Servicios}. También el proyecto European Genomic Data Infrastructure (GDI) \cite{GDI2022GDI} busca establecer una infraestructura unificada para gestionar y compartir datos genómicos en Europa, abordando desafíos de interoperabilidad y ética. Su objetivo es promover la colaboración y la innovación en genómica, posicionando a Europa como líder en el uso responsable de datos genómicos para mejorar la salud.

A nivel estatal, España colabora en muchas de las iniciativas europeas como EUCAIM o eHDSI, y conforma uno de los nodos de colaboración con OHDSI más grandes de Europa. Muchas organizaciones a lo largo del territorio español ya están colaborando con el estándar de OHDSI como la Agencia Española de Medicamentos y Productos Sanitarios (AEMPS) o Quirónsalud entre otros \cite{ohdsiSpain}. En Sevilla, especialmente, la colaboración con OHDSI la llevan a cabo el IBIS (Instituto de Biomedicina de Sevilla), la fundación FISEVI (Fundación para la Gestión de la Investigación en Salud en Sevilla) y los hospitales universitarios Virgen Macarena y Virgen del Rocío, con participación muy importante en el proyecto de EUCAIM y la comunidad de OHDSI. El pasado octubre de 2023 el hospital Macarena celebró el 'Innodata 2023' \cite{HUVM2023INNODATA}, un congreso nacional sobre investigación de datos en salud, en la que se presentó una ponencia que trató las herramientas y experiencias de OHDSI. 

Por otra parte, el Hospital Virgen del Rocio también está participando en estos proyectos innovadores a cargo del departamento de Innovación Tecnológica, siendo esta la sede del estudio práctico que ha acompañado al desarrollo del TFG, que tratará de aquí en adelante la importancia de la organización OHDSI, su estándar y herramientas.

%%---------------------------------------------------------------------------

\section{Motivación} \label{sec:01Motivacion}

%Mi motivación personal de entrar en el mundo del %análisis de datos clínicos utilizando  esta %herramienta prometedora..

La principal motivación para realizar este Trabajo Fin de Grado ha sido mi curiosidad e interés por el mundo de la ciencia de datos a lo largo de mis años de formación universitaria. El origen se sitúa en el primer año de carrera, allá en el 2020, cuando por primera vez el profesor de estadística nos habló a mi y a mis compañeros sobre el 'Big Data' como una disciplina emergente de gran interés a nivel laboral. Esta primera toma de contacto, fue la que me llevó a continuar investigando sobre dicha disciplina y todo lo relacionado con ella. En tercero de carrera tuve la oportunidad de realizar el programa de movilidad ERASMUS al Politecnico di Milano, una de las mejores universidades de ingeniería del mundo \cite{QSPolimi}, por lo que opté a seleccionar el mayor número de asignaturas de Data Science que mi convenio de estudios me permitió. Este año de estudio en Milán confirmó que, lo que había nacido como una mera curiosidad, se había convertido en una pasión, por lo que a mi regreso del Erasmus me decidí a orientar mi carrera profesional y mi TFG hacia el mundo del análisis de datos clínicos, hasta el día de hoy en que este trabajo es escrito.

También ha sido de gran importancia la motivación de mis profesores y tutores de la Escuela Técnica Superior de Ingeniería Informática de la Universidad de Sevilla y la colaboración, mediante el convenio de prácticas, del grupo científico del Departamento de Innovación Tecnológica del Hospital Universitario Virgen del Rocío, quienes confiando en mi me han apoyado, motivado y dado las herramientas y conocimientos necesarios para completar mi formación sobre ATLAS y OHDSI y la informática clínica en general.
