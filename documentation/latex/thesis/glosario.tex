\chapter{Glosario}\label{anexo:glosario}

%%TENER EN CUENTA----------------------------------------------------

%% Todos losTérminos indexados según su traducción al español y asociados a su término inglés

%¿¿¿¿¿¿¿¿ Hacer un Glossary con los términos indexados directamente según el término en inglés? ??????

%%--------------------------------------------------------------

%%A

%%B

\textbf{Datos masivos (\textit{Big Data}):} Conjunto de datos extremadamente grandes y complejos que requieren tecnologías especializadas para su almacenamiento, procesamiento y análisis, con el objetivo de extraer información significativa y tomar decisiones informadas.

%%C

\textbf{Computación en la Nube (\textit{Cloud Computing}):} Modelo de prestación de servicios de computación a través de internet, donde los recursos como almacenamiento, servidores y aplicaciones son proporcionados y gestionados por proveedores externos, permitiendo un acceso flexible y escalable según la demanda del usuario.



%%D

%%E

%%F

%%G

%%H

%%I
\textbf{Industria 4.0 (\textit{Industry 4.0}):} Concepto acuñado por el gobierno alemán en 2011 para referirse a la emergente cuarta revolución industrial basada fundamentalmente en la integración de los sistemas físicos con Internet a través de herramientas como Internet de las cosas, Big Data, Cloud Computing o Inteligencia Artficial.

\textbf{Internet de las cosas (\textit{Internet of Things, IoT}):} Red de dispositivos, sistemas y servicios que incorporan sensores, software y otras tecnologías que permiten la conectividad avanzada y el intercambio de datos entre sí a través de Internet u otras redes de comunicación \cite{oracleiot}, \cite{wikipediaiot}.

\textbf{Inteligencia Artificial (\textit{Artificial Intelligence, AI}):} Disciplina científica que se ocupa de crear programas informáticos que ejecutan operaciones comparables a las que realiza la mente humana, como el aprendizaje o el razonamiento lógico. \cite{rae}

%%J

%%K

%%L

%%M

%%N

%%Ñ

%%O

%%P

%%Q

%%R

%%S

\textbf{Sistemas ciber-físicos (\textit{Cyber-Physical Systems, CPS}}: Sistemas que integran componentes físicos y computacionales, conectados a través de redes, para monitorear y controlar procesos físicos en tiempo real, utilizando tecnologías como sensores, actuadores, y sistemas de información y comunicación.

\textbf{Sanidad 4.0 (\textit{Healthcare 4.0})}: También conocido como Salud 4.0, es la aplicación de tecnologías digitales como inteligencia artificial, Internet de las cosas y big data en el sector de la salud para mejorar la atención médica, la gestión de datos y la experiencia del paciente.

%%T

%%U

%%V

%%W

%%X

%%Y

%%Z

