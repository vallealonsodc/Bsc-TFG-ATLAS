\chapter{Introducción} \label{cap:01Introduccion}

Este manual ha sido redactado con motivo de complementar como anexo al Trabajo de Fin de Grado %
\textbf{Extrayendo conocimiento a partir de análisis clínico de datos CDM usando la herramienta Atlas} %
de la misma autora, María del Valle Alonso de Caso Ortiz.

Esta herramienta, 

No obstante, debido al fin último de este manual de acompañar al Trabajo Fin de Grado desarrollado en compañía del grupo de Innovación Tecnológica del Hospital Universitario Virgen del Rocío,  frente a la multitud de posibilidades de configuración de distintos aspectos de la herramienta, aunque el manual presenta todos ellos, en númerosas ocasiones solo se centra en los procedimientos que aplican a las necesidades del TFG y del departamento del Hospital.

Es decir, el manual se desarrolla en un contexto de implementación de ATLAS Broadsea bajo los requisitos de una organización pública, el HUVR, con fines de ayudar y dar soporte a las investigaciones realizadas en el mismo, y bajo la supervisión continua de Da. Silvia Rodríguez Mejías y Dr. Carlos Parra Calderón.

%\subsubsection{Información adicional.}

Toda la información que se ha generado durante y tras la redacción de este documento, y del TFG, se encuentra en un repositorio de github público, para permitir el acceso de cualquier usuario a comprobar archivos generados de variables, logs, scripts de código, etcétera. Para mayor información consultar el link del repositorio \cite{vallealonsodc}.






