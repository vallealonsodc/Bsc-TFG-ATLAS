\begin{table}[H]
\centering
\small % Tamaño del texto reducido
\begin{tabular}{|l|p{5cm}|p{5cm}|} % Reducción del ancho de las columnas
%\begin{tabular}{|l|p{6cm}|p{6cm}|}
\hline
\textbf{Característica} & \textbf{Metodologías Tradicionales} & \textbf{Metodologías Ágiles} \\
\hline
Planificación & Planificación detallada y rígida al inicio del proyecto. & Planificación adaptable y flexible, se adapta a cambios constantes. \\
\hline
Entrega de valor & Entregas al final del proyecto. & Entregas frecuentes de funcionalidades, permitiendo feedback temprano. \\
\hline
%Roles & Roles definidos y fijos. & Equipos multifuncionales con roles flexibles. \\
%\hline
Cambio & Cambios difíciles de gestionar, conllevan retrasos y costos adicionales. & Cambios bienvenidos y gestionados de manera eficiente, se incorporan fácilmente al proyecto. \\
\hline
%Calidad & Pruebas al final del ciclo de desarrollo. & Pruebas continuas e integradas durante todo el proceso. \\
%\hline
Cliente & Interacción limitada con el cliente. & Colaboración estrecha con el cliente, involucrado en todo el proceso. \\
\hline
\end{tabular}
\caption{Comparación de características entre metodologías tradicionales y ágiles en proyectos informáticos.}
\label{tab:metodologias}
\end{table}
