\chapter*{Resumen}

Este Trabajo Fin de Grado presenta el estudio y aplicación de la herramienta de análisis de datos ATLAS perteneciente a la organización \textit{Observational Health Data Sciences and Informatics} (OHDSI). El proyecto, que aborda la necesidad de estandarización de los sistemas informáticos y datos de salud en el panorama nacional e internacional, se ha desarrollado en colaboración con el Grupo de Informática de la Salud Computacional (GIS) del Hospital Universitario Virgen del Rocío (HUVR).

En este marco, la organización OHDSI se alza como la comunidad de investigadores más importante con la finalidad común de unificar la forma de conducir estudios observacionales, a través del Modelo de Datos Común de OMOP y un complejo ecosistema de herramientas de procesamiento y análisis de datos, entre las que destaca ATLAS, eje central de este trabajo. 

La herramienta de análisis de datos \textit{low-code}, permite ejecutar varios análisis dependiendo del caso de uso concreto siguiendo una metodología común. Finalmente, este TFG se completa con un caso práctico en el que mediante el despliegue de la herramienta ATLAS a través de Broadsea, se estandariza un estudio sobre posibles efectos adversos del tratamiento radioterápico en pacientes de cáncer de pulmón realizado por el HUVR. El proyecto en su totalidad, confirma la relevancia de la organización OHDSI en el sector de la informática clínica y los beneficios de la aplicación de la herramienta ATLAS y el Modelo de Datos Común de OMOP en la estandarización de datos de salud e investigación observacional.


\vspace{.5cm}

\textbf{Palabras clave:} ATLAS, Broadsea, Modelo de Datos Común, Observational Health Data Sciences and Informatics (OHDSI), Observational Medical Outcomes Partnership (OMOP), Oncología.