\chapter*{Resumen}

Este trabajo consiste en el estudio y aplicación de la herramienta de análisis de datos ATLAS perteneciente a la organización \textit{Observational Health Data Sciences and Informatics} (OHDSI) con la finalidad de promover la estandarización de la investigación observacional con datos de salud. El proyecto se ha desarrollado en colaboración con el grupo de Informática de la Salud Computacional del Hospital Universitario Virgen del Rocío.

La necesidad de estandarización entre los sistemas informáticos y datos de salud es un aspecto de cada vez mayor relevancia a nivel mundial. En este aspecto, OHDSI se alza como una comunidad de investigadores con la finalidad común de unificar la forma de conducir estudios observacionales, a través del Modelo de Datos Común de OMOP y un complejo ecosistema de herramientas de procesamiento y análisis de datos, entre las que destaca ATLAS, una herramienta de análisis de datos \textit{low-code} que permite ejecutar análisis siguiendo una metodología común.

Por la relevancia de la organización, el proyecto consta de una primera parte en la que se recopila información teórica sobre OHDSI, sus estándares y herramientas. Posteriormente, la investigación se complementa con un caso práctico en el que se reproduce con ATLAS un estudio realizado por el grupo de investigadores del hospital sobre posibles efectos adversos del tratamiento radioterápico en pacientes de cáncer de pulmón.

El proyecto en su totalidad confirma la relevancia de OHDSI en el sector de la informática clínica y los beneficios de la aplicación de la herramienta ATLAS y el Modelo de Datos Común de OMOP en la estandarización de la investigación observacional.


\vspace{.5cm}

\textbf{Palabras clave:} Observational Health Data Sciences and Informatics, ATLAS, Modelo de Datos Común de OMOP, Broadsea.