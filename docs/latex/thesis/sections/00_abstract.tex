\chapter*{Abstract}

This work involves the study and application of the data analysis tool ATLAS, which belongs to the organization Observational Health Data Sciences and Informatics (OHDSI), with the aim of promoting the standardization of observational research with health data. The project has been developed in collaboration with the Computational Health Informatics group at the Virgen del Rocío University Hospital.

The need for standardization among health data and information systems is an increasingly important aspect worldwide. In this regard, OHDSI stands out as a community of researchers with the common goal of unifying the conduct of observational studies through the OMOP Common Data Model and a complex ecosystem of data processing and analysis tools, among which ATLAS stands out. ATLAS is a low-code data analysis tool that allows analyses to be performed following a common methodology.

Due to the organization's relevance, the project includes an initial part where theoretical information about OHDSI, its standards, and tools is collected. Subsequently, the research is complemented by a practical case in which a study conducted by the hospital's research group on possible adverse effects of radiotherapy treatment in lung cancer patients is reproduced using ATLAS.

The entire project confirms the relevance of OHDSI in the clinical informatics sector and the benefits of applying the ATLAS tool and the OMOP Common Data Model in the standardization of observational research.



\vspace{.5cm}

\textbf{Keywords:} Observational Health Data Sciences and Informatics, ATLAS, Broadsea, OMOP Common Data Model 