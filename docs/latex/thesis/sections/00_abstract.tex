\chapter*{Abstract}

**Translation:**

This Bachelor's Thesis presents the study and application of the data analysis tool ATLAS, belonging to the organization \textit{Observational Health Data Sciences and Informatics} (OHDSI). The project, which addresses the need for standardization of health information systems and data at the national and international levels, has been developed in collaboration with the Computational Health Informatics Group (GIS) of the Virgen del Rocío University Hospital (HUVR).

In this context, the OHDSI organization stands out as the most important community of researchers with the common goal of unifying the way observational studies are conducted, through the OMOP Common Data Model and a complex ecosystem of data processing and analysis tools, among which ATLAS, the central focus of this work, stands out.

The \textit{low-code} data analysis tool allows for the execution of various analyses depending on the specific use case following a common methodology. Finally, this thesis is completed with a practical case in which, through the deployment of the ATLAS tool via Broadsea, a study on possible adverse effects of radiotherapy treatment in lung cancer patients conducted by HUVR is standardized. The project as a whole confirms the relevance of the OHDSI organization in the clinical informatics sector and the benefits of applying the ATLAS tool and the OMOP Common Data Model in the standardization of health data and observational research.


\vspace{.5cm}

\textbf{Keywords:} ATLAS, Broadsea, Common Data Model, Observational Health Data Sciences and Informatics (OHDSI), Observational Medical Outcomes Partnership (OMOP), Oncology.