\chapter{Conclusiones}\label{cap:11conclusiones}

El análisis de datos clínicos está adquiriendo una importancia cada vez mayor a nivel mundial, especialmente mediante estudios observacionales a gran escala que utilizan datos del mundo real. Estos estudios son fundamentales para generar evidencia real que pueda mejorar la toma de decisiones en el ámbito sanitario. Una de las necesidades cruciales en este contexto es la interoperabilidad entre los diferentes sistemas de información. La búsqueda de estándares que faciliten esta interoperabilidad es esencial, pero también conlleva desafíos significativos.

En este escenario, la iniciativa OHDSI (Observational Health Data Sciences and Informatics) juega un papel crucial. OHDSI se dedica a la estandarización de la investigación clínica y tiene una presencia destacada en numerosos proyectos europeos como EHDEN, IMPaCT-Data, y EUCAIM. Estas colaboraciones subrayan la importancia de OHDSI en el esfuerzo por crear un marco común para el análisis de datos clínicos, que permita a los investigadores de todo el mundo colaborar y compartir conocimientos de manera más eficiente.

Uno de los aspectos más destacados de OHDSI es su modelo de datos OMOP (Observational Medical Outcomes Partnership), que proporciona una estructura estandarizada para la recopilación y análisis de datos de salud. El modelo OMOP permite la armonización de datos provenientes de diversas fuentes, lo que es crucial para realizar estudios multicéntricos y comparativos. %A través de este proyecto, se ha podido experimentar de primera mano las ventajas que ofrece este modelo, como la reducción de la redundancia de datos y la mejora en la calidad y precisión de los análisis. La capacidad de integrar datos de diferentes sistemas y fuentes en un formato común facilita no solo la investigación, sino también la toma de decisiones informadas en la práctica clínica.
Por otro lado, la herramienta ATLAS, realiza una tarea muy importante en en la exploración y análisis de datos clínicos a través de su interfaz \textit{low-code}, de gran relevancia para facilitar la realización de estos análisis de forma más sencilla, eficiente y precisa. 

%A lo largo del proyecto, también se enfrentaron desafíos significativos, como las limitaciones temporales y las dificultades asociadas al autoaprendizaje de las interfaces y herramientas. El periodo limitado para completar el estudio impuso restricciones en la profundidad y alcance de los análisis, requiriendo una planificación meticulosa y una priorización de las tareas más críticas. Además, la necesidad de aprender de forma autodidacta sobre el uso de herramientas como ATLAS y la implementación de entornos virtuales con Docker presentó obstáculos adicionales. No obstante, estos desafíos también brindaron oportunidades para desarrollar habilidades autodidactas y resiliencia, fomentando una mayor comprensión técnica y autonomía en el manejo de tecnologías avanzadas.

La colaboración en redes como OHDSI, así como en espacios de trabajo más limitados como el grupo de Informática de la Salud del Hospital Universitario Virgen del Rocío es fundamental para el intercambio de conocimientos y experiencias, y fomenta la innovación y mejoran la calidad de los resultados de la investigación. Trabajar en equipo y compartir responsabilidades dentro de un entorno colaborativo es esencial para enfrentar los desafíos complejos del análisis de datos clínicos y para avanzar en la investigación sanitaria.

En conclusión, la implementación y uso del estándar OHDSI y sus herramientas asociadas han demostrado ser altamente beneficiosos para la investigación clínica y la práctica médica. La capacidad de estandarizar y analizar datos de salud de manera eficiente y coherente tiene un impacto significativo en la mejora de la calidad de la atención sanitaria y en el avance de la investigación médica. La creciente adopción de OHDSI en proyectos europeos subraya su relevancia y potencial para transformar el panorama de la salud pública y la investigación biomédica. Este proyecto ha proporcionado una valiosa experiencia práctica y ha resaltado la importancia de continuar promoviendo la estandarización y la interoperabilidad en los sistemas de salud.
