\chapter{Conclusiones}\label{cap:11conclusiones}

Como se evidencia en este Trabajo Fin de Grado, el análisis de datos clínicos está adquiriendo una importancia cada vez mayor a nivel mundial, especialmente en estudios observacionales a gran escala que utilizan datos del mundo real. Estos estudios son fundamentales para generar evidencia real que pueda mejorar la toma de decisiones en el ámbito sanitario. En este contexto, uno de los grandes desafíos que se presenta es la necesidad de mejora en la interoperabilidad entre los diferentes sistemas de información a través de la búsqueda de estándares que faciliten esta interoperabilidad, especialmente en el área de los datos de salud.

La iniciativa de estandarización de datos e investigación de OHDSI (Observational Health Data Sciences and Informatics) juega un papel fundamental, teniendo una presencia destacada en numerosos proyectos de investigación europeos como EHDEN, IMPaCT-Data, y EUCAIM. Estas colaboraciones subrayan la importancia de OHDSI en su esfuerzo por crear un marco común para el análisis de datos sanitarios, que permita a los investigadores de todo el mundo colaborar y compartir conocimientos de manera más eficiente.

En definitiva, uno de los aspectos más destacados de OHDSI es el Modelo de Datos Común de OMOP (Observational Medical Outcomes Partnership), que proporciona una estructura estandarizada para la recopilación y análisis de datos de salud. Este modelo permite la armonización de datos provenientes de diversas fuentes, facilitando la realización de estudios multicéntricos y comparativos. %A través de este proyecto, se ha podido experimentar de primera mano las ventajas que ofrece este modelo, como la reducción de la redundancia de datos y la mejora en la calidad y precisión de los análisis. La capacidad de integrar datos de diferentes sistemas y fuentes en un formato común facilita no solo la investigación, sino también la toma de decisiones informadas en la práctica clínica.
Por otro lado, la herramienta ATLAS, realiza una tarea muy importante en el campo de la exploración y análisis de datos clínicos a través de su interfaz \textit{low-code}, de gran relevancia para simplificar estos análisis y facilitar su reproducibilidad mediante la reutilziación y exportación del código subyacente al análisis, proporcionado en este proyecto a través del repositorio de github del Trabajo Fin de Grado.

%A lo largo del proyecto, también se enfrentaron desafíos significativos, como las limitaciones temporales y las dificultades asociadas al autoaprendizaje de las interfaces y herramientas. El periodo limitado para completar el estudio impuso restricciones en la profundidad y alcance de los análisis, requiriendo una planificación meticulosa y una priorización de las tareas más críticas. Además, la necesidad de aprender de forma autodidacta sobre el uso de herramientas como ATLAS y la implementación de entornos virtuales con Docker presentó obstáculos adicionales. No obstante, estos desafíos también brindaron oportunidades para desarrollar habilidades autodidactas y resiliencia, fomentando una mayor comprensión técnica y autonomía en el manejo de tecnologías avanzadas.

La colaboración con la red de investigadores de OHDSI, así como con el Grupo de Informática de la Salud (GIS) del Hospital Universitario Virgen del Rocío (HUVR) ha sido fundamental para el intercambio de conocimientos y experiencias, fomentar la innovación y mejorar la calidad de los resultados de la investigación. La experiencia adquida mediante el trabajo en el GIS evidencia que el trabajo  en equipo y la compartición de responsabilidades es esencial para enfrentar desafíos complejos concretamente en el campo del análisis de datos clínicos para lograr avanzar en la investigación sanitaria.


%La implementación y uso del estándar OHDSI y sus herramientas asociadas han demostrado ser altamente beneficiosos para la investigación clínica y la práctica médica. La capacidad de estandarizar y analizar datos de salud de manera eficiente y coherente tiene un impacto significativo en la mejora de la calidad de la atención sanitaria y en el avance de la investigación médica. La creciente adopción de OHDSI en proyectos europeos subraya su relevancia y potencial para transformar el panorama de la salud pública y la investigación biomédica. 
Por tanto, este proyecto ha proporcionado,  una valiosa experiencia teórica y práctica, resaltando la importancia de continuar promoviendo la estandarización y la interoperabilidad en los sistemas de salud utilizando la herramienta ATLAS, desplegada a través Broadsea y documentada en el Anexo \ref{anexo:manual} ''Manual de instalación, despliegue y configuración de ATLAS Broadsea'' con la finalidad de estandarizar el estudio oncológico sobre efectos adversos del tratamiento radioterápico llevado a cabo por el HUVR.
