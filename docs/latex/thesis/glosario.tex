\chapter{Glosario}\label{anexo:glosario}


\textbf{Aprendizaje automático (\textit{Machine Learning, ML}):} Campo de la inteligencia artificial que desarrolla algoritmos y modelos que permiten a las máquinas aprender a partir de datos, identificar patrones y tomar decisiones sin necesidad de ser programadas explícitamente para cada tarea específica.

\textbf{ATLAS}: Herramienta de código abierto desarrollada por la colaboración Observational Health Data Sciences and Informatics (OHDSI), diseñada para la visualización, exploración y análisis de datos de salud provenientes de diferentes fuentes y estándares, facilitando la investigación en salud pública y la toma de decisiones clínicas basadas en evidencia.


%%B

%\textbf{\textit{Backend}}: Parte de un sistema de software que se encarga del procesamiento y la gestión de datos, así como de la lógica de negocio que no es visible para el usuario final. Incluye el servidor, la base de datos y la lógica de aplicación que realiza las operaciones detrás de escena para proporcionar funcionalidades al frontend.

%%C

\textbf{Código abierto \textit{(Open source)}}: Modelo de desarrollo de software que promueve el acceso abierto al código fuente de un programa, permitiendo su estudio, modificación y distribución por parte de la comunidad de desarrolladores, lo que fomenta la colaboración, la transparencia y la innovación en el desarrollo de software.

\textbf{Cohorte \textit{(Cohort)}}: Grupo de individuos que comparten una característica común o que han sido seleccionados para participar en un estudio de investigación, con el fin de observar y analizar los resultados de un evento o exposición específica durante un período de tiempo determinado.

\textbf{Computación en la Nube (\textit{Cloud Computing}):} Modelo de prestación de servicios de computación a través de internet, donde los recursos como almacenamiento, servidores y aplicaciones son proporcionados y gestionados por proveedores externos, permitiendo un acceso flexible y escalable según la demanda del usuario.

\textbf{Contenedor Docker \textit{(Docker container)}}: Tecnología de virtualización que permite empaquetar y ejecutar aplicaciones y sus dependencias en entornos aislados, proporcionando portabilidad, rapidez y consistencia en el despliegue de aplicaciones en diferentes sistemas operativos y entornos de ejecución.



%%D

\textbf{Datos del mundo real \textit{(Real World Data, RWD)}}: Información sobre la salud y los resultados de atención médica recopilada de fuentes del mundo real, como registros médicos electrónicos, reclamaciones de seguros y dispositivos portátiles, utilizada para complementar los datos de ensayos clínicos y proporcionar información sobre la efectividad y seguridad de tratamientos en condiciones reales fuera del entorno controlado de un estudio clínico.

\textbf{Datos masivos (\textit{Big Data}):} Conjunto de datos extremadamente grandes y complejos que requieren tecnologías especializadas para su almacenamiento, procesamiento y análisis, con el objetivo de extraer información significativa y tomar decisiones informadas.


%%E

\textbf{\textit{European Health Data \& Evidence Network (EHDEN)}}: Consorcio europeo que tiene como objetivo establecer una infraestructura escalable y sostenible para el análisis de datos de salud del mundo real en Europa. EHDEN promueve la estandarización de datos y el uso de herramientas y métodos avanzados para facilitar la investigación clínica y epidemiológica.

%%F

%\textbf{\textit{Frontend}}: Nivel de un sistema de software o una aplicación que interactúa directamente con el usuario final. Incluye la interfaz de usuario, que permite a los usuarios interactuar con el sistema, y cualquier elemento visible o interactivo en la pantalla, como botones, formularios y gráficos. 

%%G

%%H

\textbf{Historial Clínico Electrónico (HCE)}: Registro digitalizado y centralizado de toda la información médica de un paciente, que incluye datos como diagnósticos, tratamientos, resultados de pruebas, alergias y antecedentes médicos, accesible por profesionales de la salud autorizados para mejorar la coordinación de la atención, la precisión diagnóstica y la seguridad del paciente.




%%I
\textbf{Industria 4.0 (\textit{Industry 4.0}):} Concepto acuñado por el gobierno alemán en 2011 para referirse a la emergente cuarta revolución industrial basada fundamentalmente en la integración de los sistemas físicos con Internet a través de herramientas como Internet de las cosas, Big Data, Cloud Computing o Inteligencia Artificial.

\textbf{Inteligencia Artificial (\textit{Artificial Intelligence, AI}):} Disciplina científica que se ocupa de crear programas informáticos que ejecutan operaciones comparables a las que realiza la mente humana, como el aprendizaje o el razonamiento lógico.

\textbf{Internet de las cosas (\textit{Internet of Things, IoT}):} Red de dispositivos, sistemas y servicios que incorporan sensores, software y otras tecnologías que permiten la conectividad avanzada y el intercambio de datos entre sí a través de Internet u otras redes de comunicación.


\textbf{Interoperabilidad}: Capacidad de sistemas, dispositivos o aplicaciones para intercambiar datos y trabajar juntos de manera efectiva, garantizando que la información sea comprensible y utilizada de manera consistente entre diferentes plataformas, organizaciones o entornos. Se puede clasificar en tres grupos: semántica, técnica y organizacional.

%%J

%%K

%%L


\textbf{\textit{Low-code}}: Enfoque de desarrollo de software que utiliza herramientas visuales y abstracciones de código para permitir a los usuarios crear aplicaciones de manera rápida y con menos necesidad de programación manual, acelerando el proceso de desarrollo y permitiendo a usuarios con menos experiencia técnica participar en la creación de aplicaciones.

%%M

\textbf{Modelo de Datos Común de OMOP \textit{(OMOP Common Data Model, OMOP CDM)}}: Estructura estandarizada de base de datos desarrollada por la colaboración Observational Medical Outcomes Partnership (OMOP), diseñada para representar datos de salud de manera uniforme y compatible, facilitando el análisis comparativo de datos clínicos y epidemiológicos provenientes de diferentes fuentes y sistemas de salud.


%%N

%%Ñ

%%O
\textbf{\textit{Observational Health Data Sciences and Informatics (OHDSI)}}: Organización internacional que desarrolla y aplica métodos de análisis de datos de salud para generar evidencia a partir de datos del mundo real, con el objetivo de mejorar la toma de decisiones en salud pública y clínica, promoviendo el uso de estándares y herramientas abiertas para el intercambio y análisis de datos.

\textit{\textbf{Observational Medical Outcomes Partnership (OMOP)}}: Iniciativa colaborativa entre la industria, académicos y reguladores para mejorar la evaluación de medicamentos a través del análisis de datos de salud del mundo real. OMOP desarrolla métodos y estándares para el análisis de datos de salud, incluido el Modelo de Datos Común (CDM), que permite la armonización de datos para la investigación.

\textbf{OMOPizar}: Proceso de transformar datos de salud de diferentes fuentes y formatos al Modelo de Datos Común de OMOP (CDM), para estandarizar la representación de los datos y facilitar su análisis comparativo y la generación de evidencia científica en investigación clínica.



%%P

%%Q

%%R

%%S

\textbf{Salud digital \textit{(e-Salud)}}: Utilización de tecnologías de la información y comunicación en el ámbito de la salud para mejorar la eficiencia, accesibilidad, calidad y seguridad de los servicios médicos, así como para fomentar la participación activa de los pacientes en su cuidado y la gestión de su salud.

\textbf{Sanidad 4.0 (\textit{Healthcare 4.0})}: También conocido como Salud 4.0, es la aplicación de tecnologías digitales como inteligencia artificial, Internet de las cosas y big data en el sector de la salud para mejorar la atención médica, la gestión de datos y la experiencia del paciente.

\textbf{Sistemas ciber-físicos (\textit{Cyber-Physical Systems, CPS}}): Sistemas que integran componentes físicos y computacionales, conectados a través de redes, para monitorear y controlar procesos físicos en tiempo real, utilizando tecnologías como sensores, actuadores, y sistemas de información y comunicación.



%%T

\textbf{Tecnologías de la Información y Comunicación (TICs)}: Conjunto de herramientas, recursos y sistemas tecnológicos utilizados para adquirir, almacenar, procesar, transmitir y presentar información de manera digital, facilitando la comunicación y el intercambio de datos entre personas, organizaciones y dispositivos.

\textbf{Telemedicina}: Práctica médica que utiliza tecnologías de la información y comunicación para realizar consultas médicas, diagnósticos, tratamiento y seguimiento de pacientes a distancia, facilitando el acceso a la atención médica y la colaboración entre profesionales de la salud sin necesidad de encuentros físicos.

%%U

%%V

%%W

%%X

%%Y

%%Z

